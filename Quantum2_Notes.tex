% Copyright (c) 2020 Brett MacNeil <brett.macneil@dal.ca>

\documentclass[12pt, titlepage]{article}

\usepackage{fullpage}  %Increases margins
\usepackage{amsmath} % math
\usepackage{amssymb} % symbols
\usepackage{physics} % physics
\usepackage{graphicx} % including figures
\usepackage[labelfont=bf]{caption} % figures
\usepackage{float}  % placement of figures
\usepackage{siunitx} % SI units
\usepackage{mhchem} % chemical symbols
\usepackage{mdframed} % frames
\usepackage{xcolor} % colors
\usepackage{hyperref} % references and links
\hypersetup{colorlinks=false, linkbordercolor=1 1 1}

\newcommand{\com}[2]{\qty[#1 \,,\, #2]}
\newcommand{\exv}[1]{\langle #1 \rangle}

\setlength{\parindent}{0pt} % no auto-indents
\setlength{\parskip}{8pt}

\begin{document}
\title{\huge Quantum Physics Notes \vspace{2cm} \\  \large Adapted from Dr. Jesse Maasen's lecture notes\thanks{As well as \emph{Introduction to Quantum Mechanics}, 3rd ed. by Griffiths and Schroeter.} \\ \vspace{0.5cm} \normalsize for \vspace{0.5cm} \\  PHYC 4151 \vspace{2cm}}
\author{\huge Brett MacNeil \\ \vspace{0.5cm} \normalsize Dalhousie University}
\date{Fall 2020 \\ \vspace{1cm}
	\normalsize Last updated: \today}
\maketitle

\section{Spin}
\subsection{Introduction to spin}
In classical mechanics, a rigid object can have two kinds of angular momentum:

\begin{description}
	\item Orbital: $\vb{L} = \vb{r}\cross\vb{p}$
	\item Spin : $\vb{S} = I\vb*{\omega}$
\end{description}

%% Figure: L and S for Earth orbit

Classically, both are equivalent; spin is simply the sum of the angular momenta of infinitesimal elements over the volume of a non-point mass. In quantum mechanics, we have the same two components of angular momentum, but the distinction is absolutely fundamental.

There is orbital angular momentum, for example associated with the motion of an electron around the nucleus in the hydrogen atom, for example. This is described by the spherical harmonics for a spherically symmetric potential.

The electron also carries another form of angular momentum which has nothing to do with motion in space and is independent of position ($r$, $\phi$, $\theta$) but is somewhat analogous to classical spin. This analogy cannot be pressed too far, as quantum mechanical spin cannot be decomposed into the orbital angular momentum of constituent parts.

We start from the fact that elementary particles carry \emph{intrinsic} angular momentum $\vb{S}$ in addition to their \emph{extrinsic} angular momentum $\vb{L}$. The algebraic theory of spin is similar to the theory of orbital angular momentum.

\subsection{Review of orbital angular momentum}
\subsubsection{Definitions of orbital angular momentum}
Orbital angular momentum is defined as the cross product of the position and linear momentum vector:

\begin{equation}
\vb{L} = \vb{r}\cross\vb{p} \, .
\end{equation}

%% Figure: angular momentum vector

In Cartesian components, this is

\begin{subequations}
	\begin{align}
	L_{x} &= y p_{z} - z p_{y} \\
	L_{y} &= z p_{x} - x p_{x} \\
	L_{z} &= x p_{y} - y p_{x} \, ,
	\end{align}
\end{subequations}

and in quantum mechanics, we use the prescription

\begin{subequations}
	\begin{align}
	p_{x} &= - i \hbar \pdv*{x} \\
	p_{y} &= - i \hbar \pdv*{y} \\
	p_{z} &= - i \hbar \pdv*{z} \, .
	\end{align}
\end{subequations}

An important consequence is that the angular momentum operators do not commute:

\begin{subequations}
	\begin{align}
	\com{L_{x}}{L_{y}} &= i \hbar L_{z} \\
	\com{L_{y}}{L_{z}} &= i \hbar L_{x} \\
	\com{L_{z}}{L_{x}} &= i \hbar L_{y} \, .
	\end{align}
\end{subequations}

The observables $L_{x}$, $L_{y}$ and $L_{z}$ are incompatible and for example, the uncertainty principle says

\begin{equation}
	\sigma_{L_{x}}\sigma_{L_{y}} \geq \frac{\hbar}{2}\left|\exv{L_{z}}\right| \, .
\end{equation}

These  $L_{x}$, $L_{y}$ and $L_{z}$ operators do not share the same eigenfunctions. However, the total angular momentum,

\begin{equation}
	L^2 = L_{x}^{\, 2} + L_{y}^{\, 2} + L_{z}^{\, 2}
\end{equation}

does commute with each of the component angular momentum operators,

\begin{subequations}
	\begin{align}
	\com{L^2}{L_{x}} &= 0 \\
	\com{L^2}{L_{y}} &= 0 \\
	\com{L^2}{L_{z}} &= 0 \, .
	\end{align}
\end{subequations}

 and therefore they share the same eigenfunctions.
 
 \subsubsection{Eigenvalues and Eigenfunctions}
 The eigenvalue equations for the angular momentum operators are 
 
 \begin{align}
 	L^2 Y_{\ell}^{m} &= \hbar^2\ell(\ell+1) Y_{\ell}^{m} \\
 	L_{z} Y_{\ell}^{m} &= \hbar m Y_{\ell}^{m} \, ,
 \end{align}
 
 where $\ell$ is the azimuthal quantum number ($\ell=0,1,2,\ldots$) and $m$ is the magnetic quantum number ($m=-\ell,-\ell+1,\ldots,\ell-1,\ell$). The eigenfunctions $Y_{\ell}^{m}$ are functions of $\phi$ and $\theta$ and are known as the \emph{spherical harmonics}.
 
 %% Figure: Classical picture of l = 2
 
We can also define ladder operators,

\begin{equation}
	L_{\pm} = L_{x} \pm i L_{y}
\end{equation}

%% Figure: Ladder operator

These operators raise or lower the eigenvalues and eigenstates of $L_{z}$ . Of course, applying the raising operator to the top state, or applying the lowering operator to the bottom state gives zero:

\begin{subequations}
\begin{align}
	L_{+}\ket{\ell, m_{\text{max}}} &= 0 \\
	L_{-}\ket{\ell, m_{\text{min}}} &= 0 \, .
\end{align}
\end{subequations}

The ladder operators are not hermitian and are thus not observable quantities.
 
 \subsection{Theory of spin}
 We begin with the fundamental commutation relations for spin, which is analogous to those of orbital angular momentum:
 
 \begin{subequations}
 	\begin{align}
 	\com{S_{x}}{S_{y}} &= i \hbar S_{z} \\
 	\com{S_{y}}{S_{z}} &= i \hbar S_{x} \\
 	\com{S_{z}}{S_{x}} &= i \hbar S_{y} \, .
 	\end{align}
 \end{subequations}

We will take these as postulates for the theory of spin. The eigenvectors of $S^2$ and $S_{z}$ satisfy:

\begin{align}
	S^2 \ket{s, m} &= \hbar^2 s(s+1)\ket{s, m} \\
	S_{z} \ket{s, m} &= \hbar m \ket{s, m}
\end{align}

where $\ket{s, m}$ is an eigenvector of both $S^2$ and $S_z$, which has no dependence on $\phi$ or $\theta$. Since the eigenstates of spin are not functions, we represent the eigenvector in Dirac notation rather than writing $Y_{s}^{m}$ as we did for the spherical harmonics.

%% Figure: Classical picture for s = 1

We also have spin ladder operators

\begin{equation}
	S_{\pm} = S_{x} \pm i S_{y} \,,
\end{equation}

where

\begin{equation}
	S_{\pm}\ket{s, m} = \hbar\sqrt{s(s+1)-m(m\pm1)}\ket{s, m\pm1} \, .
\end{equation}

The spin ladder operators change $m$ by $\pm 1$.

%% Figure: Spin ladder operators

In general, $s$ can be an integer or half integer, whereas for orbital angular momentum, $\ell$ could only be an integer.

\begin{equation}
	\left\{
	\begin{array}{l}
	s = 0, \, 1/2, \, 1, \, 3/2, \, \ldots \\
	m = -s, \, -s+1, \, \ldots, \, s-1, \, s
	\end{array}
	\right.
\end{equation}

Every particle has a specific and immutable value of $s$, which we call spin. By contrast, the orbital angular momentum quantum number $\ell$ can take on any integer value, and may change when the system is perturbed. \textbf{Spin is fixed}!

\subsection{Spin 1/2}
The case where $s=1/2$ is the most important to study, since ordinary matter (electrons, protons, neutrons) have this spin. There are only two eigenstates.

\begin{description}
	\item Spin up ($\uparrow$): $s=\frac{1}{2}$, $m=+\frac{1}{2}$
	\item Spin down ($\downarrow$): $s=\frac{1}{2}$, $m=-\frac{1}{2}$
\end{description}

Using these two eigenstates as a basis, we can express the general state of a spin-$1/2$ particle as a two-component vector, known as a \textbf{spinor}:

\begin{equation}
	\chi = \begin{bmatrix} a \\ b \end{bmatrix} = a\chi_+ + b\chi_- 
\end{equation}

where the basis states are

\begin{equation}
	\chi_+ = \begin{bmatrix} 1 \\ 0 \end{bmatrix} \quad \text{(spin up)} \qquad;\qquad \chi_- = \begin{bmatrix} 0 \\ 1 \end{bmatrix} \quad \text{(spin down)} \,.
\end{equation}

This means that the spin operators are $2 \times 2$ matrices and we can determine their form by examining their effect on the basis spinors. Since $S^2\ket{s, m} = \hbar^2s(s+1)\ket{s, m}$, it follows that for $s=1/2$,

\begin{equation}
	S^2\chi_+ = \frac{3}{4}\hbar^2\chi_+ \quad\text{and}\quad S^2\chi_- = \frac{3}{4}\hbar^2\chi_- \,.
\end{equation}

If we consider a general matrix $S^2 = \begin{bmatrix} s_{11} & s_{12} \\ s_{21} & s_{22}\end{bmatrix}$ then we find

\begin{align*}
	S^2\chi_+ = \begin{bmatrix} s_{11} & s_{12} \\ s_{21} & s_{22}\end{bmatrix}\begin{bmatrix} 1 \\ 0\end{bmatrix} = \frac{3}{4}\hbar^2\begin{bmatrix} 1 \\ 0\end{bmatrix} \quad\implies\quad \begin{bmatrix} s_{11} \\ s_{21}\end{bmatrix}=\begin{bmatrix} 3\hbar^2/4 \\ 0\end{bmatrix}\\[8pt]
	S^2\chi_- = \begin{bmatrix} s_{11} & s_{12} \\ s_{21} & s_{22}\end{bmatrix}\begin{bmatrix} 0 \\ 1\end{bmatrix} = \frac{3}{4}\hbar^2\begin{bmatrix} 0 \\ 1\end{bmatrix} \quad\implies\quad \begin{bmatrix} s_{12} \\ s_{22}\end{bmatrix}=\begin{bmatrix} 0 \\ 3\hbar^2/4 \end{bmatrix}
\end{align*}

which means that

\begin{equation}
	S^2 = \frac{3}{4}\hbar^2\begin{bmatrix} 1 & 0 \\ 0 & 1\end{bmatrix} \,.
\end{equation}

Additionally, since $S_z\ket{s, m} = \hbar m \ket{s, m}$ we have

\begin{equation}
S_z\chi_+ = \frac{1}{2}\hbar\chi_+ \quad\text{and}\quad S_z\chi_- = -\frac{1}{2}\hbar\chi_- \,.
\end{equation}

By similar analysis as above, we find that the matrix representation of the $S_z$ operator is

\begin{equation}
	S_z = \frac{\hbar}{2}\begin{bmatrix}1 & 0 \\ 0 & -1\end{bmatrix} \,.
\end{equation}

Meanwhile, the eigenvalue equations for the spin ladder operators are

\begin{equation*}
	S_+\chi_- = \hbar\chi_+ \quad;\quad S_-\chi_+ = \hbar\chi_- \quad;\quad S_+\chi_+ = S_-\chi_- = 0
\end{equation*}

and therefore it can be shown that

\begin{equation}
	S_+ = \hbar\begin{bmatrix} 0 & 1 \\ 0 & 0\end{bmatrix} \quad\text{and}\quad S_- = \hbar\begin{bmatrix} 0 & 0 \\ 1 & 0\end{bmatrix} \,.
\end{equation}

Now since $S_\pm = S_x \pm iS_y$, we find $S_x = (S_+ + S_-)/2$ and $S_y = (S_+ - S_-)/2i$. This gives

\begin{equation}
	S_x = \frac{\hbar}{2}\begin{bmatrix} 0 & 1 \\ 1 & 0\end{bmatrix} \quad\text{and}\quad S_y = \frac{\hbar}{2}\begin{bmatrix} 0 & -i \\ i & 0\end{bmatrix} \,.
\end{equation}

Since the spin operators all carry a factor of $\hbar/2$, it is tidier to write

\begin{equation}
	\vb{S} = \frac{\hbar}{2}\vb*{\sigma}
\end{equation}

where $\vb*{\sigma}$ is given by the \textbf{Pauli spin matrices}:

\begin{equation}
\boxed{
	\sigma_x = \begin{bmatrix} 0 & 1 \\ 1 & 0\end{bmatrix} \quad,\quad
	\sigma_y = \begin{bmatrix} 0 & -i \\ i & 0\end{bmatrix} \quad,\quad
	\sigma_z = \begin{bmatrix} 1 & 0 \\ 0 & -1\end{bmatrix} }
\end{equation}

The $S_x$, $S_y$, and $S_z$ operators are Hermitian, since they represent observables. The ladder operators, on the other hand, are not Hermitian as they do not represent observables. 

The eigenspinors of $S_z$ are, by definition, $\chi_+$ and $\chi_-$. If we measure $S_z$ on a particle in some general state $\chi = a\chi_+ + b\chi_-$ then we will measure $=\hbar/2$ with probability $|a|^2$ or $-\hbar/2$ with probability $|b|^2$. Those are the only two choices, so the normalization condition imposes

\begin{equation}
	|a|^2 + |b|^2 = 1
\end{equation}

since $\bra{\chi}\ket{\chi} = 1$. 

\begin{mdframed}
\paragraph*{Note:} The probability of observing a state $\ket{\chi_\pm}$ after measuring a particle in superposition state $\ket{\chi}$ is given by the norm of the projection of the measured state onto the particle state. In other words,

\begin{equation}
	P_\pm = \qty|\bra{\chi_\pm}\ket{\chi}|^2
\end{equation}

where the inner product for these spinors involves matrix multiplication, as the basis is discrete.
\end{mdframed}

What if we choose to measure $S_x$? What are the possible results, and the associated probabilities? According to the generalized statistical interpretation, we must know the eigenvalues and eigenspinors of $S_x$. The characteristic equation is:

\begin{equation*}
	S_x\chi = \lambda\chi \quad\implies\quad \det(S_x-\lambda\mathbb{I}) = 0 \quad\implies \begin{vmatrix}-\lambda & \hbar/2 \\ \hbar/2 & -\lambda \end{vmatrix} = 0
\end{equation*}

which yields eigenvalues of $\lambda \pm\hbar/2$, as expected. The eigenspinors are then calculated by:

\begin{equation*}
	\frac{\hbar}{2}\begin{bmatrix}0 & 1 \\ 1 & 0 \end{bmatrix}\begin{bmatrix} \alpha \\ \beta \end{bmatrix} = \pm\frac{\hbar}{2}\begin{bmatrix} \alpha \\ \beta \end{bmatrix} \quad\implies \begin{bmatrix} \beta \\ \alpha \end{bmatrix} = \pm\begin{bmatrix} \alpha \\ \beta \end{bmatrix} \quad\implies\quad \beta = \pm\alpha
\end{equation*}

so that we can write $\chi_{\pm}^{(x)} = \alpha \begin{bmatrix} 1 \\ \pm 1\end{bmatrix}$ and applying the normalization condition gives $\alpha = 1/\sqrt{2}$. This gives

\begin{equation}
	\chi_+^{(x)} = \frac{1}{\sqrt{2}}\begin{bmatrix} 1 \\ 1\end{bmatrix} \quad;\quad \chi_-^{(x)} = \frac{1}{\sqrt{2}}\begin{bmatrix} 1 \\ -1\end{bmatrix}
\end{equation}

Recall that the eigenvectors of a Hermitian matrix span the space and form a complete basis. We can use a linear combination of them to express a generic spinor. If we change the basis to the eigenspinors of $S_x$, we get

\begin{equation}
	\chi = \bra{\chi_+^{(x)}}\ket{\chi}\chi_+^{(x)} + \bra{\chi_-^{(x)}}\ket{\chi}\chi_-^{(x)}
\end{equation}

which gives

\begin{equation}
	\chi = \qty(\frac{a+b}{\sqrt{2}})\chi_+^{(x)} + \qty(\frac{a-b}{\sqrt{2}})\chi_-^{(x)} \,.
\end{equation}
Measuring $S_x$ will yield $+\hbar/2$ with probability $|a+b|^2/2$ and $-\hbar/2$ with probability $|a-b|^2/2$.
\clearpage

\begin{mdframed}[backgroundcolor=gray!20]
\paragraph*{Example:}
Suppose a spin-1/2 particle is in the state:

\begin{equation*}
	\chi = \frac{1}{\sqrt{6}} \begin{bmatrix} 1+i \\ 2 \end{bmatrix} \,.
\end{equation*}

What are the probabilities of obtaining $+\hbar/2$ and $-\hbar/2$ when measuring $S_z$ ans $S_x$?

\paragraph*{Solution:}
Here, $a = (1+i)/\sqrt{6}$ and $b = 2/\sqrt{6}$. 

For $S_z$, the probability of measuring $+\hbar/2$ is 

\begin{equation*}
	|a|^2 = \qty|\frac{1+i}{\sqrt{6}}|^2 = 1/3 \,.
\end{equation*}

The probability of measuring $-\hbar/2$ is

\begin{equation*}
	|b|^2 = \qty|\frac{2}{\sqrt{6}}|^2 = 2/3 \,.
\end{equation*}

For $S_x$, the probability of measuring $+\hbar/2$ is 

\begin{equation*}
	\frac{|a+b|^2}{2} = \frac{|(3+i)/\sqrt{6}|^2}{2} = 5/6 \,.
\end{equation*}

The probability of measuring $-\hbar/2$ is

\begin{equation*}
	\frac{|a-b|^2}{2} = \frac{|(-1+i)/\sqrt{6}|^2}{2} = 1/6 \,.
\end{equation*}
\end{mdframed}

\begin{mdframed}[backgroundcolor=gray!20]
\paragraph*{Problem:} (a) Show that the eigenspinors of $S_y$ are 

\begin{equation}
\chi_+^{(y)} = \frac{1}{\sqrt{2}}\begin{bmatrix} 1 \\ i\end{bmatrix} \quad;\quad \chi_-^{(x)} = \frac{1}{\sqrt{2}}\begin{bmatrix} 1 \\ -i \end{bmatrix} \,.
\end{equation}

(b) If you measure $S_y$ on a particle in a general state $\chi$, what values might you get, and what is the probability of each?
\end{mdframed}
\end{document}