% Copyright (c) 2020 Brett MacNeil <brett.macneil@dal.ca>

\documentclass[12pt, titlepage]{article}

\usepackage{fullpage}  %Increases margins
\usepackage{amsmath} % math
\usepackage{amssymb} % symbols
\usepackage{physics} % physics
\usepackage{graphicx} % including figures
\usepackage[labelfont=bf]{caption} % figures
\usepackage{float}  % placement of figures
\usepackage{siunitx} % SI units
\usepackage{mhchem} % chemical symbols
\usepackage{mdframed} % frames
\usepackage{xcolor} % colors
\usepackage{cancel} % cancelto
\usepackage{hyperref} % references and links
\hypersetup{colorlinks=false, linkbordercolor=1 1 1}

\newcommand{\exv}[1]{\left\langle #1 \right\rangle}
\newcommand{\Sx}{\begin{bmatrix} 0 & 1 \\ 1 & 0\end{bmatrix}}
\newcommand{\Sy}{\begin{bmatrix} 0 & -i \\ i & 0\end{bmatrix}}
\newcommand{\Sz}{\begin{bmatrix} 1 & 0 \\ 0 & -1\end{bmatrix}}

\setlength{\parindent}{0pt} % no auto-indents
\setlength{\parskip}{8pt}

\begin{document}
\title{\huge Quantum Physics Notes \vspace{2cm} \\  \large Adapted from Dr. Jesse Maasen's lecture notes\thanks{As well as \emph{Introduction to Quantum Mechanics}, 3rd ed. by Griffiths and Schroeter.} \\ \vspace{0.5cm} \normalsize for \vspace{0.5cm} \\  PHYC 4151 \vspace{2cm}}
\author{\huge Brett MacNeil \\ \vspace{0.5cm} \normalsize Dalhousie University}
\date{Fall 2020 \\ \vspace{1cm}
	\normalsize Last updated: \today}
\maketitle

\section{Spin}
\subsection{Introduction to spin}
In classical mechanics, a rigid object can have two kinds of angular momentum:

\begin{description}
	\item Orbital: $\vb{L} = \vb{r}\cross\vb{p}$
	\item Spin : $\vb{S} = I\vb*{\omega}$
\end{description}

%% Figure: L and S for Earth orbit

Classically, both are equivalent; spin is simply the sum of the angular momenta of infinitesimal elements over the volume of a non-point mass. In quantum mechanics, we have the same two components of angular momentum, but the distinction is absolutely fundamental.

There is orbital angular momentum, for example associated with the motion of an electron around the nucleus in the hydrogen atom, for example. This is described by the spherical harmonics for a spherically symmetric potential.

The electron also carries another form of angular momentum which has nothing to do with motion in space and is independent of position ($r$, $\phi$, $\theta$) but is somewhat analogous to classical spin. This analogy cannot be pressed too far, as quantum mechanical spin cannot be decomposed into the orbital angular momentum of constituent parts.

We start from the fact that elementary particles carry \emph{intrinsic} angular momentum $\vb{S}$ in addition to their \emph{extrinsic} angular momentum $\vb{L}$. The algebraic theory of spin is similar to the theory of orbital angular momentum.

\subsection{Review of orbital angular momentum}
\subsubsection{Definitions of orbital angular momentum}
Orbital angular momentum is defined as the cross product of the position and linear momentum vector:

\begin{equation}
\vb{L} = \vb{r}\cross\vb{p} \, .
\end{equation}

%% Figure: angular momentum vector

In Cartesian components, this is

\begin{subequations}
	\begin{align}
	L_{x} &= y p_{z} - z p_{y} \\
	L_{y} &= z p_{x} - x p_{x} \\
	L_{z} &= x p_{y} - y p_{x} \, ,
	\end{align}
\end{subequations}

and in quantum mechanics, we use the prescription

\begin{subequations}
	\begin{align}
	p_{x} &= - i \hbar \pdv*{x} \\
	p_{y} &= - i \hbar \pdv*{y} \\
	p_{z} &= - i \hbar \pdv*{z} \, .
	\end{align}
\end{subequations}

An important consequence is that the angular momentum operators do not commute:

\begin{subequations}
	\begin{align}
	\comm{L_{x}}{L_{y}} &= i \hbar L_{z} \\
	\comm{L_{y}}{L_{z}} &= i \hbar L_{x} \\
	\comm{L_{z}}{L_{x}} &= i \hbar L_{y} \, .
	\end{align}
\end{subequations}

The observables $L_{x}$, $L_{y}$ and $L_{z}$ are incompatible and for example, the uncertainty principle says

\begin{equation}
	\sigma_{L_{x}}\sigma_{L_{y}} \geq \frac{\hbar}{2}\left|\exv{L_{z}}\right| \, .
\end{equation}

These  $L_{x}$, $L_{y}$ and $L_{z}$ operators do not share the same eigenfunctions. However, the total angular momentum,

\begin{equation}
	L^2 = L_{x}^{\, 2} + L_{y}^{\, 2} + L_{z}^{\, 2}
\end{equation}

does commute with each of the component angular momentum operators,

\begin{subequations}
	\begin{align}
	\comm{L^2}{L_{x}} &= 0 \\
	\comm{L^2}{L_{y}} &= 0 \\
	\comm{L^2}{L_{z}} &= 0 \, .
	\end{align}
\end{subequations}

 and therefore they share the same eigenfunctions.
 
 \subsubsection{Eigenvalues and Eigenfunctions}
 The eigenvalue equations for the angular momentum operators are 
 
 \begin{align}
 	L^2 Y_{\ell}^{m} &= \hbar^2\ell(\ell+1) Y_{\ell}^{m} \\
 	L_{z} Y_{\ell}^{m} &= \hbar m Y_{\ell}^{m} \, ,
 \end{align}
 
 where $\ell$ is the azimuthal quantum number ($\ell=0,1,2,\ldots$) and $m$ is the magnetic quantum number ($m=-\ell,-\ell+1,\ldots,\ell-1,\ell$). The eigenfunctions $Y_{\ell}^{m}$ are functions of $\phi$ and $\theta$ and are known as the \emph{spherical harmonics}.
 
 %% Figure: Classical picture of l = 2
 
We can also define ladder operators,

\begin{equation}
	L_{\pm} = L_{x} \pm i L_{y}
\end{equation}

%% Figure: Ladder operator

These operators raise or lower the eigenvalues and eigenstates of $L_{z}$ . Of course, applying the raising operator to the top state, or applying the lowering operator to the bottom state gives zero:

\begin{subequations}
\begin{align}
	L_{+}\ket{\ell, m_{\text{max}}} &= 0 \\
	L_{-}\ket{\ell, m_{\text{min}}} &= 0 \, .
\end{align}
\end{subequations}

The ladder operators are not hermitian and are thus not observable quantities.
\subsection{Theory of spin}
 We begin with the fundamental commutation relations for spin, which is analogous to those of orbital angular momentum:
 
 \begin{subequations}
 	\begin{align}
 	\comm{S_{x}}{S_{y}} &= i \hbar S_{z} \\
 	\comm{S_{y}}{S_{z}} &= i \hbar S_{x} \\
 	\comm{S_{z}}{S_{x}} &= i \hbar S_{y} \, .
 	\end{align}
 \end{subequations}

We will take these as postulates for the theory of spin. The eigenvectors of $S^2$ and $S_{z}$ satisfy:

\begin{align}
	S^2 \ket{s, m} &= \hbar^2 s(s+1)\ket{s, m} \\
	S_{z} \ket{s, m} &= \hbar m \ket{s, m}
\end{align}

where $\ket{s, m}$ is an eigenvector of both $S^2$ and $S_z$, which has no dependence on $\phi$ or $\theta$. Since the eigenstates of spin are not functions, we represent the eigenvector in Dirac notation rather than writing $Y_{s}^{m}$ as we did for the spherical harmonics.

%% Figure: Classical picture for s = 1

We also have spin ladder operators

\begin{equation}
	S_{\pm} = S_{x} \pm i S_{y} \,,
\end{equation}

where

\begin{equation}
	S_{\pm}\ket{s, m} = \hbar\sqrt{s(s+1)-m(m\pm1)}\ket{s, m\pm1} \, .
\end{equation}

The spin ladder operators change $m$ by $\pm 1$.

%% Figure: Spin ladder operators

In general, $s$ can be an integer or half integer, whereas for orbital angular momentum, $\ell$ could only be an integer.

\begin{equation}
	\left\{
	\begin{array}{l}
	s = 0, \, 1/2, \, 1, \, 3/2, \, \ldots \\
	m = -s, \, -s+1, \, \ldots, \, s-1, \, s
	\end{array}
	\right.
\end{equation}

Every particle has a specific and immutable value of $s$, which we call spin. By contrast, the orbital angular momentum quantum number $\ell$ can take on any integer value, and may change when the system is perturbed. \textbf{Spin is fixed}!

\subsection{Spin 1/2}
The case where $s=1/2$ is the most important to study, since ordinary matter (electrons, protons, neutrons) have this spin. There are only two eigenstates.

\begin{description}
	\item Spin up ($\uparrow$): $s=\frac{1}{2}$, $m=+\frac{1}{2}$
	\item Spin down ($\downarrow$): $s=\frac{1}{2}$, $m=-\frac{1}{2}$
\end{description}

Using these two eigenstates as a basis, we can express the general state of a spin-$1/2$ particle as a two-component vector, known as a \textbf{spinor}:

\begin{equation}
	\chi = \begin{bmatrix} a \\ b \end{bmatrix} = a\chi_+ + b\chi_- 
\end{equation}

where the basis states are

\begin{equation}
	\chi_+ = \begin{bmatrix} 1 \\ 0 \end{bmatrix} \quad \text{(spin up)} \qquad;\qquad \chi_- = \begin{bmatrix} 0 \\ 1 \end{bmatrix} \quad \text{(spin down)} \,.
\end{equation}

This means that the spin operators are $2 \times 2$ matrices and we can determine their form by examining their effect on the basis spinors. Since $S^2\ket{s, m} = \hbar^2s(s+1)\ket{s, m}$, it follows that for $s=1/2$,

\begin{equation}
	S^2\chi_+ = \frac{3}{4}\hbar^2\chi_+ \quad\text{and}\quad S^2\chi_- = \frac{3}{4}\hbar^2\chi_- \,.
\end{equation}

If we consider a general matrix $S^2 = \begin{bmatrix} s_{11} & s_{12} \\ s_{21} & s_{22}\end{bmatrix}$ then we find

\begin{align*}
	S^2\chi_+ = \begin{bmatrix} s_{11} & s_{12} \\ s_{21} & s_{22}\end{bmatrix}\begin{bmatrix} 1 \\ 0\end{bmatrix} = \frac{3}{4}\hbar^2\begin{bmatrix} 1 \\ 0\end{bmatrix} \quad\implies\quad \begin{bmatrix} s_{11} \\ s_{21}\end{bmatrix}=\begin{bmatrix} 3\hbar^2/4 \\ 0\end{bmatrix}\\[8pt]
	S^2\chi_- = \begin{bmatrix} s_{11} & s_{12} \\ s_{21} & s_{22}\end{bmatrix}\begin{bmatrix} 0 \\ 1\end{bmatrix} = \frac{3}{4}\hbar^2\begin{bmatrix} 0 \\ 1\end{bmatrix} \quad\implies\quad \begin{bmatrix} s_{12} \\ s_{22}\end{bmatrix}=\begin{bmatrix} 0 \\ 3\hbar^2/4 \end{bmatrix}
\end{align*}

which means that

\begin{equation}
	S^2 = \frac{3}{4}\hbar^2\begin{bmatrix} 1 & 0 \\ 0 & 1\end{bmatrix} \,.
\end{equation}

Additionally, since $S_z\ket{s, m} = \hbar m \ket{s, m}$ we have

\begin{equation}
S_z\chi_+ = \frac{1}{2}\hbar\chi_+ \quad\text{and}\quad S_z\chi_- = -\frac{1}{2}\hbar\chi_- \,.
\end{equation}

By similar analysis as above, we find that the matrix representation of the $S_z$ operator is

\begin{equation}
	S_z = \frac{\hbar}{2}\Sz \,.
\end{equation}

Meanwhile, the eigenvalue equations for the spin ladder operators are

\begin{equation*}
	S_+\chi_- = \hbar\chi_+ \quad;\quad S_-\chi_+ = \hbar\chi_- \quad;\quad S_+\chi_+ = S_-\chi_- = 0
\end{equation*}

and therefore it can be shown that

\begin{equation}
	S_+ = \hbar\begin{bmatrix} 0 & 1 \\ 0 & 0\end{bmatrix} \quad\text{and}\quad S_- = \hbar\begin{bmatrix} 0 & 0 \\ 1 & 0\end{bmatrix} \,.
\end{equation}

Now since $S_\pm = S_x \pm iS_y$, we find $S_x = (S_+ + S_-)/2$ and $S_y = (S_+ - S_-)/2i$. This gives

\begin{equation}
	S_x = \frac{\hbar}{2}\Sx \quad\text{and}\quad S_y = \frac{\hbar}{2}\Sy \,.
\end{equation}

Since the spin operators all carry a factor of $\hbar/2$, it is tidier to write

\begin{equation}
	\vb{S} = \frac{\hbar}{2}\vb*{\sigma}
\end{equation}

where $\vb*{\sigma}$ is given by the \textbf{Pauli spin matrices}:

\begin{equation}
\boxed{
	\sigma_x = \Sx \quad,\quad
	\sigma_y = \Sy \quad,\quad
	\sigma_z = \Sz }
\end{equation}

The $S_x$, $S_y$, and $S_z$ operators are Hermitian, since they represent observables. The ladder operators, on the other hand, are not Hermitian as they do not represent observables. 

The eigenspinors of $S_z$ are, by definition, $\chi_+$ and $\chi_-$. If we measure $S_z$ on a particle in some general state $\chi = a\chi_+ + b\chi_-$ then we will measure $=\hbar/2$ with probability $|a|^2$ or $-\hbar/2$ with probability $|b|^2$. Those are the only two choices, so the normalization condition imposes

\begin{equation}
	|a|^2 + |b|^2 = 1
\end{equation}

since $\bra{\chi}\ket{\chi} = 1$. 

\begin{mdframed}
\paragraph*{Note:} The probability of observing a state $\ket{\chi_\pm}$ after measuring a particle in superposition state $\ket{\chi}$ is given by the norm of the projection of the measured state onto the particle state. In other words,

\begin{equation}
	P_\pm = \qty|\bra{\chi_\pm}\ket{\chi}|^2
\end{equation}

where the inner product for these spinors involves matrix multiplication, as the basis is discrete.
\end{mdframed}

What if we choose to measure $S_x$? What are the possible results, and the associated probabilities? According to the generalized statistical interpretation, we must know the eigenvalues and eigenspinors of $S_x$. The characteristic equation is:

\begin{equation*}
	S_x\chi = \lambda\chi \quad\implies\quad \det(S_x-\lambda\mathbb{I}) = 0 \quad\implies \begin{vmatrix}-\lambda & \hbar/2 \\ \hbar/2 & -\lambda \end{vmatrix} = 0
\end{equation*}

which yields eigenvalues of $\lambda \pm\hbar/2$, as expected. The eigenspinors are then calculated by:

\begin{equation*}
	\frac{\hbar}{2}\begin{bmatrix}0 & 1 \\ 1 & 0 \end{bmatrix}\begin{bmatrix} \alpha \\ \beta \end{bmatrix} = \pm\frac{\hbar}{2}\begin{bmatrix} \alpha \\ \beta \end{bmatrix} \quad\implies \begin{bmatrix} \beta \\ \alpha \end{bmatrix} = \pm\begin{bmatrix} \alpha \\ \beta \end{bmatrix} \quad\implies\quad \beta = \pm\alpha
\end{equation*}

so that we can write $\chi_{\pm}^{(x)} = \alpha \begin{bmatrix} 1 \\ \pm 1\end{bmatrix}$ and applying the normalization condition gives $\alpha = 1/\sqrt{2}$. This gives

\begin{equation}
	\chi_+^{(x)} = \frac{1}{\sqrt{2}}\begin{bmatrix} 1 \\ 1\end{bmatrix} \quad;\quad \chi_-^{(x)} = \frac{1}{\sqrt{2}}\begin{bmatrix} 1 \\ -1\end{bmatrix}
\end{equation}

Recall that the eigenvectors of a Hermitian matrix span the space and form a complete basis. We can use a linear combination of them to express a generic spinor. If we change the basis to the eigenspinors of $S_x$, we get

\begin{equation}
	\chi = \bra{\chi_+^{(x)}}\ket{\chi}\chi_+^{(x)} + \bra{\chi_-^{(x)}}\ket{\chi}\chi_-^{(x)}
\end{equation}

which gives

\begin{equation}
	\chi = \qty(\frac{a+b}{\sqrt{2}})\chi_+^{(x)} + \qty(\frac{a-b}{\sqrt{2}})\chi_-^{(x)} \,.
\end{equation}
Measuring $S_x$ will yield $+\hbar/2$ with probability $|a+b|^2/2$ and $-\hbar/2$ with probability $|a-b|^2/2$.
\clearpage

\begin{mdframed}[backgroundcolor=gray!20]
\paragraph*{Example:}
Suppose a spin-1/2 particle is in the state:

\begin{equation*}
	\chi = \frac{1}{\sqrt{6}} \begin{bmatrix} 1+i \\ 2 \end{bmatrix} \,.
\end{equation*}

What are the probabilities of obtaining $+\hbar/2$ and $-\hbar/2$ when measuring $S_z$ ans $S_x$?

\paragraph*{Solution:}
Here, $a = (1+i)/\sqrt{6}$ and $b = 2/\sqrt{6}$. 

For $S_z$, the probability of measuring $+\hbar/2$ is 

\begin{equation*}
	|a|^2 = \qty|\frac{1+i}{\sqrt{6}}|^2 = 1/3 \,.
\end{equation*}

The probability of measuring $-\hbar/2$ is

\begin{equation*}
	|b|^2 = \qty|\frac{2}{\sqrt{6}}|^2 = 2/3 \,.
\end{equation*}

For $S_x$, the probability of measuring $+\hbar/2$ is 

\begin{equation*}
	\frac{|a+b|^2}{2} = \frac{|(3+i)/\sqrt{6}|^2}{2} = 5/6 \,.
\end{equation*}

The probability of measuring $-\hbar/2$ is

\begin{equation*}
	\frac{|a-b|^2}{2} = \frac{|(-1+i)/\sqrt{6}|^2}{2} = 1/6 \,.
\end{equation*}
\end{mdframed}

\begin{mdframed}[backgroundcolor=gray!20]
\paragraph*{Problem:} (a) Show that the eigenspinors of $S_y$ are 

\begin{equation}
\chi_+^{(y)} = \frac{1}{\sqrt{2}}\begin{bmatrix} 1 \\ i\end{bmatrix} \quad;\quad \chi_-^{(x)} = \frac{1}{\sqrt{2}}\begin{bmatrix} 1 \\ -i \end{bmatrix} \,.
\end{equation}

(b) If you measure $S_y$ on a particle in a general state $\chi$, what values might you get, and what is the probability of each?
\end{mdframed}

\subsection{Electron in a magnetic field}
A charged particle with spin constitutes a magnetic dipole. The \textbf{magnetic dipole moment}, of a particle, denoted by $\vb*{\mu}$, is proportional to its spin angular momentum $\vb{S}$:

\begin{equation}
    \vb*{\mu} = \gamma \vb{S} \,,
\end{equation}

where the proportionality constant $\gamma$ is the \textbf{gyromagnetic ratio}. When a magnetic dipole is placed in a magnetic field $\vb{B}$, it experiences a torque $\vb{\Gamma} = \vb*{\mu}\cross\vb{B}$ which tends to line it up parallel to the field. The associated energy is known as the Zeeman energy,

\begin{equation}
	H = -\vb*{\mu}\vdot\vb{B} \,,
\end{equation}

so that the Hamiltonian of a charged particle with nonzero spin in a magnetic field is

\begin{equation}
	\mathcal{H} = -\gamma\vb{B}\vdot\vb{S} \,,
\end{equation}

where $\vb{S}$ is the appropriate spin matrix. 

\subsubsection{Larmor precession}
Imagine a particle of spin $1/2$, which points in the $z-$ direction,

\begin{equation*}
	\vb{B} = B_0\vu{z},.
\end{equation*}

The Hamiltonian is therefore

\begin{equation}
	\mathcal{H} = -\gamma B_0 S_z = -\frac{\gamma B_0 \hbar}{2} \begin{bmatrix} 1 & 0 \\ 0 & -1 \end{bmatrix} \,.
\end{equation}

Since the Hamiltonian is a scalar multiple of the $S_z$ operator, it is trivial to show that they commute. Therefore, the eigenstates of $\mathcal{H}$ are the same as those of $S_z$:

\begin{equation}
    \left\{
    \begin{array}{cc}
    \chi_+ \,, & \text{with energy} E_+ = -\gamma B_0\hbar/2 \\[6pt]
    \chi_- \,, & \text{with energy} E_- = \gamma B_0\hbar/2
    \end{array}
    \right.
\end{equation}

The energy is lowest when the dipole moment is parallel to the field, just as it would be classically. Since the Hamiltonian is time-independent, the general solution to the time-dependent Schrödinger equation $\mathcal{H}\chi = i\hbar\pdv*{\chi}{t}$ can be expressed in terms of stationary states:

\begin{equation*}
	\chi(t) = a\chi_+\exp(-i E_+ t/\hbar) + b\chi_-\exp(-i E_-t/\hbar) = \begin{bmatrix} a\exp(i\gamma B_0 t/2) \\[4pt] b\exp(-i\gamma B_0 t/2) \end{bmatrix}
\end{equation*}

where the constants $a$ and $b$ are determined by initial conditions

\begin{equation*}
\chi(0) = \begin{bmatrix} a \\ b \end{bmatrix}
\end{equation*}

and due to normalization, $|a|^2 + |b|^2 = 1$. With no loss of generality we may assume that $a$ and $b$ are real, and write $a = \cos(\alpha/2)$ and $b = \sin(\alpha/2)$, where $\alpha$ is a fixed angle. This gives

\begin{equation}
	\chi(t) = \begin{bmatrix} \cos(\alpha/2) \exp(-i\gamma B_0 t/2) \\[4pt] \sin(\alpha/2) \exp(-i\gamma B_0 t/2) \end{bmatrix} \,.
\end{equation}

To see what is happening, we can calculate the expectation value of $\vb{S}$ as a function of time:

\begin{align*}
	\exv{S_x} &= \chi^\dagger S_x \chi \\[4pt]
	&= \begin{bmatrix} \cos(\alpha/2)\exp(-i\gamma B_0 t/2) \\[4pt] \sin(\alpha/2)\exp(i\gamma B_0 t/2) \end{bmatrix}^{\text{T}} \frac{\hbar}{2} \Sx \begin{bmatrix} \cos(\alpha/2)\exp(i\gamma B_0 t/2) \\[4pt] \sin(\alpha/2)\exp(-i\gamma B_0 t/2) \end{bmatrix} \\[4pt]
	&= \frac{\hbar}{2}\sin\alpha\cos(\gamma B_0 t)
\end{align*}

Similarly,

\begin{equation*}
	\exv{S_y} = -\frac{\hbar}{2}\sin\alpha\sin(\gamma B_0 t) \quad\text{and}\quad \exv{S_z} = \frac{\hbar}{2}\cos\alpha \,.
\end{equation*}

Thus $\exv{\vb{S}}$ is tilted at a constant angle $\alpha$ to the $z$-axis, and precesses about the field at \textbf{Larmor frequency}

\begin{equation}
	\omega = \gamma B_0 \,,
\end{equation}

just as a classical dipole would. 
\clearpage

\begin{mdframed}
\paragraph*{Note:} This is a consequence of Ehrenfest's theorem, which relates the time derivative of an operator to its commutator with the Hamiltonian:

\begin{equation}
	\dv{t}\exv{\mathcal{A}} = \frac{1}{i\hbar}\exv{\comm{\mathcal{A}}{\mathcal{H}}} + \exv{\pdv{\mathcal{A}}{t}}
\end{equation}

Using this relation, one can obtain relations of the same form as classical mechanics:

\begin{align}
	\dv{\exv{\vb{p}}}{t} &= \exv{-\grad{V}} \,; \\[4pt]
	\dv{\exv{\vb{J}}}{t} &= \exv{\vb{\Gamma}} = \exv{\vb{r}\cross(-\grad{V})} \,,
\end{align}

where $\exv{-\grad{V}}$ is analogous to a force, and $\vb{J} = \vb{L} + \vb{S}$ is the total angular momentum. 
\end{mdframed}

\begin{mdframed}[backgroundcolor=gray!20]
\paragraph*{Problem:} Consider a charged particle with spin $1/2$ in a uniform magnetic field of magnitude $B_0$ along the $z$-direction. 

(a) What is the probability, as a function of time, of measuring the $x$-component of the spin angular momentum to be $+\hbar/2$?

(b) Repeat (a), for the $y$-component.

(c) Repeat (a) again, for the $z$-component.
\end{mdframed}

\subsubsection{The Stern-Gerlach experiment}
In an \textit{inhomogeneous} magnetic field, there is also a force on a magnetic dipole. From electrodynamics:

\begin{equation}
	\vb{F} = \grad(\vb*{\mu}\vdot\vb{B}) \,.
\end{equation}

This force can be used to separate out particles with a particular spin orientation. WE consider a beam of heavy neutral atoms, so we can avoid relativistic effects and large deflection from the Lorentz force. 

%% Figure: Stern-Gerlach experiment

The particles travel in the $y$-direction through a region of static but inhomogeneous magnetic field given by

\begin{equation}
	\vb{B}(x, y, z) = -\alpha x \vu{x} + (B_0 + \alpha z)\vu{z} \,,
\end{equation}

where $B_0$ is a large uniform field and $\alpha$ is a small deviation. Ideally, the field would just be along $\vu{z}$, however an $x$-component is required to ensure that $\div{\vb{B}} = 0$. There is a net force on the beam in the $z$-direction and the beam will be deflected up or down depending on $S_z$. Classically, one would observe a continuum of deflections, however since spin is quantized, there will be $2s+1$ separate streams. This was demonstrated by Stern and Gerlach using silver atoms. The inner elections are paired such that their orbital and spin angular momenta cancel. The net atomic spin comes from one unpaired valence electron, giving the overall atom $s=1/2$.

From the frame of reference of the silver atoms, the magnetic Hamiltonian is initially zero and turns on for some time $T$ as they pass through the magnet. Afterward, the Hamiltonian is again zero. This means

\begin{equation}
	\mathcal{H} = \left\{
	\begin{array}{ll}
	0 \,, & t<0 \\
	-\gamma(B_0 +\alpha z)S_z \,, & 0 \leq t \leq T \\
	0 \,, & t > T 
	\end{array}
	\right. \, .
\end{equation}

We ignore the $x$-component of the field as Larmor precession causes the expectation value $\exv{S_x}$ to average to zero since the precession frequency is large. If the atoms start in state $\chi = a\chi_+ + b\chi_-$ for $t<0$, the spin state will evolve while the Hamiltonian acts,

\begin{equation*}
	\chi(t) = a\chi_+\exp(-iE_+ t/\hbar) + b\chi_-\exp(-iE_- t/\hbar) \quad,\quad E_\pm = \mp\frac{\gamma\hbar}{2}(B_0+\alpha z) \quad\text{and}\quad 0 \leq t \leq T \,.
\end{equation*}

Hence, it emerges from the magnetic field in state

\begin{equation*}
\chi(t) = \qty(a\exp(i\gamma  T B_0/2)\chi_+)\exp(i\alpha\gamma T/2) + \qty(b\exp(-i\gamma  T B_0/2)\chi_-)\exp(-i\alpha\gamma T/2)
\end{equation*}

and is frozen in this state for $t>T$. Applying the momentum operator, we can see that the spin-up and spin-down particles have different momentum,

\begin{equation}
	\left\{
	\begin{array}{l}
	P_{z}^{\uparrow} = \alpha \gamma T \hbar/2 \\
	P_{z}^{\downarrow} = -\alpha \gamma T \hbar/2
	\end{array}
	\right.
\end{equation}

and the beam splits in two.

In this problem, we assumed that the initial state of a spin system is known and quantum mechanics tells us how it \textit{evolves}. To prepare an ensemble of atoms in a particular spin state, a beam can be passed through a Stern-Gerlach magnet and one particular beam can be selected.

\subsection{Addition of angular momentum}
Suppose now that we have two particles of spins $s_1$ and $s_2$. The first is in state $\ket{s_1, m_1}$ and the second is in state $\ket{s_2, m_2}$. We can denote the composite state as $\ket{s_1s_2,m_1m_2}$, where

\begin{align*}
	{S^{(1)}}^2\ket{s_1s_2,m_1m_2} &= s_1(s_1+1)\hbar^2\ket{s_1s_2,m_1m_2} \\
	{S^{(2)}}^2\ket{s_1s_2,m_1m_2} &= s_2(s_2+1)\hbar^2\ket{s_1s_2,m_1m_2} \\
	S^{(1)}_{z}\ket{s_1s_2,m_1m_2} &= m_1\hbar\ket{s_1s_2,m_1m_2} \\
	S^{(2)}_{z}\ket{s_1s_2,m_1m_2} &= m_2\hbar\ket{s_1s_2,m_1m_2} \,,
\end{align*}

as expected. The total $z$-component of the spin,

\begin{align*}
	S_z\ket{s_1s_2,m_1m_2} &= (S^{(1)}_{z} + S^{(2)}_{z})\ket{s_1s_2,m_1m_2} \\
	&= S^{(1)}_{z}\ket{s_1s_2,m_1m_2} + S^{(2)}_{z}\ket{s_1s_2,m_1m_2} \\
	&= m_1\hbar\ket{s_1s_2,m_1m_2} + m_2\ket{s_1s_2,m_1m_2} \\
	&= (m_1+m_2)\ket{s_1s_2,m_1m_2} \,,
\end{align*}

is simply the sum of the two particles. However, the net spin, $s$ is much less subtle. 

\subsubsection{Two spin-1/2 particles}
We consider the case of two spin-1/2 particles, such as the proton and electron in the hydrogen atom. Each can be either spin-up or spin-down so there are four possibilities in all:

\begin{align*}
	\ket{\uparrow\uparrow} &= \ket{\frac{1}{2}\frac{1}{2},\frac{1}{2}\frac{1}{2}} &&m=1 \\
	\ket{\uparrow\downarrow} &= \ket{\frac{1}{2}\frac{1}{2},\frac{1}{2}\frac{-1}{2}} &&m=0 \\
	\ket{\downarrow\uparrow} &= \ket{\frac{1}{2}\frac{1}{2},\frac{-1}{2}\frac{1}{2}} &&m=0 \\
	\ket{\downarrow\downarrow} &= \ket{\frac{1}{2}\frac{1}{2},\frac{-1}{2}\frac{-1}{2}}  &&m=-1 \,.
\end{align*}

Since $m=-s,-s+,\ldots,s-1,s$ it appears that in this case $s=1$ however there is an extra $m=0$ state. To gain insight into this, we can apply the lowering operator, $S_- = S_{-}^{(1)} + S_{-}^{(2)}$ to the state $\ket{\uparrow\uparrow}$:

\begin{align*}
	S_- \ket{\uparrow\uparrow} &= \qty(S_{-}^{(1)}\ket{\uparrow})\ket{\uparrow} + \ket{\uparrow}\qty(S_{-}^{(2)}\ket{\uparrow}) \\
	&= \hbar\ket{\downarrow}\ket{\uparrow} + \ket{\uparrow}\hbar\ket{\downarrow} \\
	&= \hbar(\ket{\downarrow\uparrow} - \ket{\uparrow\downarrow}) \,.
\end{align*}

Applying the lowering operator again to this state gives:

\begin{align*}
	S_-(\ket{\downarrow\uparrow} - \ket{\uparrow\downarrow}) &= S_{-}^{(1)}\ket{\uparrow\downarrow} + S_{-}^{(2)}\ket{\uparrow\downarrow} + S_{-}^{(1)}\ket{\downarrow\uparrow} + S_{-}^{(2)}\ket{\downarrow\uparrow} \\
	&= \qty(S_{-}\ket{\uparrow})\ket{\downarrow} + \ket{\uparrow}\cancelto{0}{\qty(S_{-}\ket{\downarrow})} + \cancelto{0}{\qty(S_{-}\ket{\downarrow})\ket{\uparrow}} + \ket{\downarrow}\qty(S_{-}\ket{\uparrow}) \\
	&= 2\hbar\ket{\downarrow\downarrow}
\end{align*}

and applying the lowering operator again gives zero. This means that we have three states with $s=1$:

\begin{equation}
	\left\{
	\begin{array}{l}
	\ket{1,1} = \ket{\uparrow\uparrow} \\[2pt]
	\ket{1,0} = \frac{1}{\sqrt{2}}(\ket{\uparrow\downarrow} + \ket{\downarrow\uparrow}) \\[2pt]
	\ket{1, -1} = \ket{\downarrow\downarrow}
	\end{array}
	\right. 
	\qquad \boxed{s=1}
\end{equation}

This is called the \textbf{triplet} combination.

\begin{mdframed}
\paragraph*{Note:} The state $\chi = \ket{\uparrow\downarrow} + \ket{\downarrow\uparrow}$ is not normalized. We must determine the inner product of $\chi$ with itself in order to normalize it:

\begin{align*}
	\bra{\chi}\ket{\chi} &= \bigl(\ket{\uparrow\downarrow} + \ket{\downarrow\uparrow}\bigr)^\dagger\bigl(\ket{\uparrow\downarrow} + \ket{\downarrow\uparrow}\bigr) \\
	&= \bigl(\bra{\uparrow\downarrow} + \bra{\downarrow\uparrow}\bigr)\bigl(\ket{\uparrow\downarrow} + \ket{\downarrow\uparrow}\bigr) \\
	&= \bra{\uparrow\downarrow}\ket{\uparrow\downarrow} + \bra{\uparrow\downarrow}\ket{\downarrow\uparrow} + \bra{\downarrow\uparrow}\ket{\uparrow\downarrow} + \bra{\downarrow\uparrow}\ket{\downarrow\uparrow} \\
	&= \bra{\uparrow}\ket{\uparrow}\bra{\downarrow}\ket{\downarrow} + \bra{\uparrow}\ket{\downarrow}\bra{\downarrow}\ket{\uparrow} + \bra{\downarrow}\ket{\uparrow}\bra{\uparrow}\ket{\downarrow} + \bra{\downarrow}\ket{\downarrow}\bra{\uparrow}\ket{\uparrow} \\
	&= (1)(1) + 0 + 0 + (1)(1) \\
	&= 2
\end{align*}

Therefore, we write $\chi = \frac{1}{\sqrt{2}}(\ket{\uparrow\downarrow} + \ket{\downarrow\uparrow})$ as the normalized spinor.
\end{mdframed}
\clearpage

The leftover orthogonal state with $m=0$ carries $s=0$. It is normalized in a similar manner as demonstrated previously. 

\begin{equation}
	\left\{
	\begin{array}{l}
	\ket{0,0} = \frac{1}{\sqrt{2}}(\ket{\uparrow\downarrow} - \ket{\downarrow\uparrow})
	\end{array}
	\right.
	\qquad \boxed{s=0}
\end{equation}

This is called the \textbf{singlet} state and applying both ladder operators will give zero.

It appears that the \textit{combination} of the two spin=1/2 particles can carry a spin of 1 or 0, depending on whether they occupy the singlet or triplet configuration. To confirm this, we must show that the triplet states are eigenvectors of $S^2$ with eigenvalue $2\hbar^2$ and applying $S^2$ to the singlet state gives zero.

The $S^2$ operator is

\begin{align}
	S^2 &= \qty(S^{(1)} + S^{(2)})\vdot\qty(S^{(1)} + S^{(2)}) \nonumber \\
	&= \qty(S^{(1)})^2 + \qty(S^{(2)})^2 + 2 S^{(1)}\vdot S^{(2)} \nonumber \\
	&= \qty(S^{(1)})^2 + \qty(S^{(2)})^2 + 2\qty(S_{x}^{(1)}S_{x}^{(2)} + S_{y}^{(1)}S_{y}^{(2)} + S_{z}^{(1)}S_{z}^{(2)}) \,.
\end{align}

For the $\ket{1,1}$ state, we have:

\begin{align*}
	S^2\ket{1,1} &= \qty(\qty(S^{(1)})^2 + \qty(S^{(2)})^2 + 2S^{(1)}\vdot S^{(2)})\ket{\uparrow\uparrow} \\
	&= (S^2\ket{\uparrow})\ket{\uparrow} + \ket{\uparrow}(S^2)\ket{\uparrow} + 2\qty((S_x\ket{\uparrow})(S_x\ket{\uparrow}) + (S_y\ket{\uparrow})(S_y\ket{\uparrow}) + (S_z\ket{\uparrow})(S_z\ket{\uparrow})) \\
	&= \frac{3\hbar^2}{4}\ket{\uparrow} + \frac{3\hbar^2}{4}\ket{\uparrow} + 2\qty(\frac{\hbar}{2}\ket{\downarrow}\frac{\hbar}{2}\ket{\downarrow} + \frac{i\hbar}{2}\ket{\downarrow}\frac{i\hbar}{2}\ket{\downarrow} + \frac{\hbar}{2}\ket{\uparrow}\frac{\hbar}{2}\ket{\uparrow}) \\
	&= 2\hbar^2\ket{\uparrow\uparrow} \\
	&= 2\hbar^2\ket{1,1} \,,
\end{align*}

as required. The proof for the $\ket{1, 0}$ and $\ket{1, -1}$ states follow the same approach. For the singlet state,

\begin{align*}
	S^2\ket{0,0} &= 0 + 0 + \qty(2S^{(1)}\vdot S^{(2)})\frac{1}{\sqrt{2}}(\ket{\uparrow\downarrow}-\ket{\downarrow\uparrow}) \\
	&= \sqrt{2}\bigl(S_x\ket{\uparrow}S_x\ket{\downarrow} + S_y\ket{\uparrow}S_y\ket{\downarrow} + S_z\ket{\uparrow}S_z\ket{\downarrow} \bigr. \\
	& \qquad\qquad \bigl.- S_x\ket{\downarrow}S_x\ket{\uparrow} + S_y\ket{\downarrow}S_y\ket{\uparrow} + S_z\ket{\downarrow}S_z\ket{\uparrow}\bigr) \\
	&= 0
\end{align*}

as required. 

We have just studied what happens when we combine two spin-1/2 particles, finding that we obtain spin-1 and spin-0 states. This case is part of a larger problem, where we combine spin-$s_1$ with spin-$s_2$. In this case, the possible spin states are:

\begin{equation}
	s = (s_1+s_2),\, (s_1+s_2-1),\, (s_1+s_2-2),\, \ldots,\, |s_1-s_2| \,.
\end{equation}

For example, combining a particle of spin 3/2 with another of spin 2, the two particles could take on spin 7/2, 5/2, 3/2 or 1/2.

\subsubsection{Arbitrary combinations}
The combined state $\ket{s, m}$ with total spin $s$ and $z$-component $m$ will be some linear combination of the composite states $\ket{s_1, m_1}\ket{s_2, m_2}$:

\begin{equation}
	\ket{s,m} = \sum_{m_1+m_2=m}C_{m_1m_2m}^{s_1s_2s}\ket{s_1s_2,m_1m_2} \,.
\end{equation}

Because the $z$-components add, the only composite states that contribute are those for which $m_1+m_2=m$. The constants $C_{m_1m_2m}^{s_1s_2s}$ are called the \textbf{Clebsch-Gordan coefficients}, which are often presented in table form.

%% Figure: Clebsch-Gordan coefficients

Consider the combination of a spin-2 particle with a spin-1 particle. The table says that

\begin{equation*}
	\ket{3,0} = \frac{1}{\sqrt{5}}\ket{2,1}\ket{1,-1} + \sqrt{\frac{3}{5}}\ket{2,0}\ket{1,0} + \frac{1}{\sqrt{5}}\ket{2,-1}\ket{1,1} \,.
\end{equation*}

If two particles of spin 2 and spin 1 are at rest in a box, and the \emph{total} spin is 3, and its $z$-component is zero, them a measurement of $S_z^{(1)}$ could return the value $\hbar$ (with probability 1/5) or zero (with probability 3/5) or $-\hbar$ (with probability 1/5).

These tables also work the other way around:

\begin{equation}
	\ket{s_1s_2,m_1m_2} = \sum_{s}C_{m_1m_2m}^{s_1s_2s}\ket{s,m_1+m_2} \,.
\end{equation}

Now consider a spin-3/2 particle along with a spin-1 particle. The table reads:

\begin{equation}
	\ket{\frac{3}{2}\frac{1}{2}, \frac{1}{2}0} = \sqrt{\frac{3}{5}}\ket{\frac{5}{2},\frac{1}{2}} + \frac{1}{\sqrt{15}}\ket{\frac{3}{2},\frac{1}{2}} - \frac{1}{\sqrt{3}}\ket{\frac{1}{2},\frac{1}{2}} \,.
\end{equation}

If particles of spin 3/2 and spin 1 are at rest in a boc, and it is known that the first has $m_1=1/2$ and the second has $m_2=0$ (so $m=1/2$), the total spin $s$ could be measured as 5/2 (with probability 3/5), 3/2 (with probability 1/15) or 1/2 (with probability 1/3).

\section{Identical particles}
\subsection{Two-particle systems}
A two particle system has a wavefunction that is a function of the coordinates of both particles, and time: $\Psi_{2\text{-particle}} = \Psi(\vb{r}_1,\vb{r}_2,t)$. The Schrödinger equation remains the same, $\mathcal{H}\Psi = i\hbar\pdv*{\Psi}{t}$, where $\mathcal{H}$ is the Hamiltonian for the whole system, 

\begin{equation}
	\mathcal{H} = -\frac{\hbar^2}{2m_1}\laplacian{}_{1} - \frac{\hbar^2}{2m_2}\laplacian{}_2 + V(\vb{r}_1, \vb{r}_2, t)\,,
\end{equation}

where $\grad{}_{j}$ only applies to the coordinates of particle $j$. The probability of finding particle 1 in volume $\dd[3]{r_1}$ \textbf{and} particle 2 in volume $\dd[3]{r_2}$ is

\begin{equation}
	|\Psi(\vb{r}_1, \vb{r}_2, t)|^2\dd[3]{r_1}\dd[3]{r_2} \,,
\end{equation}

which is the same statistical interpretation as the single-particle case. The wavefunction must also be normalized such that 

\begin{equation}
\iint|\Psi(\vb{r}_1, \vb{r}_2, t)|^2\dd[3]{r_1}\dd[3]{r_2} = 1 \,.
\end{equation}

For a time independent potential, we can obtain a complete set of solutions using separation of variables, yielding a spatial component of the wavefunction comprised of stationary states:

\begin{equation}
	\Psi(\vb{r}_1,\vb{r}_2,t) = \psi(\vb{r}_1, \vb{r}_2)\exp(-iEt/\hbar) \,,
\end{equation}

where $E$ is the total energy of the system and $\psi$ obeys the time independent Schrödinger equation.

When considering multi-particle systems, distinguishability is an important concept with important consequences. There are two ways to distinguish particles:

\begin{enumerate}
	\item Using differences in intrinsic physical properties (like mass, charge or spin).
	\item Track the trajectory of each particle.
\end{enumerate}

The problem with the second point is that quantum mechanics forbids us from having infinite prevision on the trajectory of a particle. Thus, according to quantum mechanics, identical particles are indistinguishable. This impacts the possible form of multi-particle wavefunctions.

\subsection{Bosons and fermions}
Suppose particle 1 is in the state $\psi_a(\vb{r}_1)$ and particle 2 is in the state $\psi(\vb{r}_2)$. In this case, the two-particle wavefunction is the product of the two single-particle wavefunctions:

\begin{equation}
\psi(\vb{r}_1,\vb{r}_2) = \psi(\vb{r}_1)\psi(\vb{r}_2) \,,
\end{equation}

assuming that the different particles can identified. However, for two identical particles, it is impossible to determine which particle is in state $\psi_a$ or $\psi_b$. All that can be known is that one particle is in one state and the second particle is in the other. 

To accommodate indistinguishable particles, we construct a wavefunction that is a superposition of the two possible states:

\begin{equation}
\psi_\pm(\vb{r}_1,\vb{r}_2) = A\Bigl[ \psi_a(\vb{r}_1)\psi_b(\vb{r}_2) \pm \psi_b(\vb{r}_1)\psi_a(\vb{r}_2) \Bigr] \,.
\end{equation}

The theory permits two kinds of identical particles:

\begin{itemize}
	\item  Bosons (+): have a symmetric wavefunction and are characterized by integer spin values.
	\item Fermions ($-$): have an anti-symmetric wavefunction and are characterized by half-integer spin values.
\end{itemize}

\begin{mdframed}
	\paragraph*{Note}:
	A consequence of this is that two identical fermions may not occupy the same state. Consider two identical fermions in states $\psi_a = \psi_b$. Then,
	
	\begin{equation*}
	\psi_-(\vb{r}_1,\vb{r}_2) =  A\Bigl[ \psi_a(\vb{r}_1)\psi_a(\vb{r}_2) - \psi_a(\vb{r}_1)\psi_a(\vb{r}_2) \Bigr] = 0 \,.
	\end{equation*}
	
	In this case the total particle wavefunction must be zero and therefore two identical fermions will never occupy the same state. This is the famous \textbf{Pauli exclusion principle}.	
\end{mdframed}

Let us define the exchange operator $\mathcal{P}$, which interchanges the two particles:

\begin{equation}
\mathcal{P}\ket{(1,2)} = \ket{(2,1)} \,.
\end{equation}

Clearly, applying this operator twice does not change the state ($\mathcal{P}^2 = \mathbb{I}$) and we also note that the eigenvalues of $\mathcal{P}$ are $\pm1$. If the two particles are identical, the Hamiltonian must treat them the same, since $m_1 = m_2$ and $V(\vb{r}_1,\vb{r}_2) = V(\vb{r}_2,\vb{r}_1)$. It then follows that $\mathcal{P}$ and $\mathcal{H}$ are compatible observables, 

\begin{equation}
\comm{\mathcal{P}}{\mathcal{H}} = 0 \,.
\end{equation}

Ehrenfest's theorem also ensures that $\dv*{\exv{\mathcal{P}}}{t} = 0$ and the expectation value of the exchange operator is conserved. As a result, we can find solutions to the Schrödinger equation that are either symmetric (eigenvalue $+1$) or anti-symmetric (eigenvalue $-1$) under exchange:

\begin{equation}
\boxed{ \psi(\vb{r}_1,\vb{r}_2) = \pm\psi(\vb{r}_2,\vb{r}_1) \,.}
\end{equation}

The wavefunction for identical particles is required to obey the systematization rule (+: bosons, $-$: fermions). This is true for $N$-particle systems with any two particles exchanged.

\begin{mdframed}[backgroundcolor=gray!20]
	\paragraph*{Example:}
	Suppose we have two non-interacting particles in a one-dimensional infinite square well. The one-particle states are given by:
	
	\begin{equation*}
	\psi_n(x) = \sqrt{\frac{2}{a}}\sin\qty(\frac{n\pi}{a}x) \,, \quad E_n = \frac{n^2\pi^2\hbar^2}{2ma^2} \,.
	\end{equation*}
	
	If the particles are distinguishable with the first in state $n_1$ and the second in state $n_2$, the composite wavefunction is:
	
	\begin{equation*}
	\psi_{n_{1}n_{2}}(x) = \psi_{n_{1}}(x_1)\psi_{n_{2}}(x_2) \,, \qquad E_{n_{1n_{2}}} = \frac{({n_1}^2 + {n_2}^2)\pi^2\hbar^2}{2ma^2} \,.
	\end{equation*}
	
	The ground state is 
	
	\begin{equation*}
	\psi_{1,1} = \frac{2}{a}\sin\qty(\frac{\pi}{a}x_1)\sin\qty(\frac{\pi}{a}x_2) \, \qquad E_{1,1} = \frac{\pi^2\hbar^2}{ma^2} \,.
	\end{equation*}
	
	The first excited state is doubly degenerate:
	
	\begin{equation*}
	\left\{
	\begin{array}{ll}
	\psi_{1,2} = \frac{2}{a}\sin\qty(\frac{2\pi}{a}x_2) & E_{1,2} = \frac{2\pi^2\hbar^2}{2ma^2} \\[6pt]
	\psi_{2,1} = \frac{2}{a}\sin\qty(\frac{2\pi}{a}x_1)\sin\qty(\frac{\pi}{a}x_2) & E_{2,1} = \frac{5\pi^2\hbar^2}{2ma^2}
	\end{array}
	\right. \,.
	\end{equation*}
	
	If the particles are identical bosons, the ground state is unchanged but the first excited state is non-degenerate:
	
	\begin{equation*}
	\frac{\sqrt{2}}{a}\qty[\sin\qty(\frac{\pi}{a}x_1)\sin\qty(\frac{2\pi}{a}x_2) + \sin\qty(\frac{2\pi}{a}x_1)\sin\qty(\frac{2\pi}{a}x_2)] \,,\qquad E = \frac{5\pi^2\hbar^2}{2ma^2} \,.
	\end{equation*}
	
	If the particles are identical fermions, they cannot both have the same quantum number $n$, so there is no state with energy $E = \pi^2\hbar^2/ma^2$. Instead, the ground state is
	
	\begin{equation*}
	\frac{\sqrt{2}}{a}\qty[\sin\qty(\frac{\pi}{a}x_1)\sin\qty(\frac{2\pi}{a}x_2) - \sin\qty(\frac{2\pi}{a}x_1)\sin\qty(\frac{2\pi}{a}x_2)] \,,\qquad E = \frac{5\pi^2\hbar^2}{2ma^2} \,.
	\end{equation*}
	
	%% Plots of wavefunctions
\end{mdframed}

\subsection{Exchange forces}
The symmetrization requirement of multi-particle wavefunctions has profound implications on the chemistry and physics of materials. Let us consider a one-dimensional example of two particles. Suppose one particle is in state $\psi_a(x)$ and the other is in $\psi_b(x)$. We also assume that these states are orthonormal.

If both particles are distinguishable, the combined wavefunction is

\begin{equation}
	\psi(x_1,x_2) = \psi_a(x_2)\psi_b(x_2) \,.
\end{equation}

If they are identical bosons, the (normalized) composite wavefunction is

\begin{equation}
	\psi_+(x_1,x_2) = \frac{1}{\sqrt{2}}\Bigl[\psi_a(x_1)\psi_b(x_2) + \psi_b(x_1)\psi_a(x_1)\Bigr] \,.
\end{equation}

If, on the other hand, they are identical fermions,

\begin{equation}
\psi_+(x_1,x_2) = \frac{1}{\sqrt{2}}\Bigl[\psi_a(x_1)\psi_b(x_2) - \psi_b(x_1)\psi_a(x_1)\Bigr] \,.
\end{equation}

We wish to compute the average squared distance between the particles,

\begin{equation}
	\exv{(x_1-x_2)^2} = \exv{{x_1}^2} + \exv{{x_2}^2} - 2\exv{x_1x_2} \,.
\end{equation}

\paragraph*{Case 1: Distinguishable particles.} Using the distinguishable particle wavefunction, we find that:

\begin{align*}
	\exv{{x_1}^2} &= \int{x_1}^2|\psi_a(x_1)|^2 \dd{x_1}\int|\psi_b(x_2)|^2\dd{x_2} = \exv{x^2}_a \\[4pt]
	\exv{{x_2}^2} &= \int|\psi_a(x_1)|^2\dd{x_1}\int{x_2}^2|\psi_b(x_2)|^2 \dd{x_2} = \exv{x^2}_b \\[4pt]
	\exv{x_1x_2} &= \int x_1|\psi_a(x_1)|^2 \dd{x_1} \int x_2|\psi_b(x_2)|^2\dd{x_2} = \exv{x}_a\exv{x}_b \,,
\end{align*}

so in this case we find that:

\begin{equation}
		\exv{(x_1-x_2)^2} = \exv{x^2}_a + \exv{x^2}_b - 2\exv{x}_a\exv{x}_b \,.
\end{equation}

This answer would be the same if particle 1 had been in state $\psi_b$ and particle 2 had been in state $\psi_b$.

\paragraph*{Case 2: Identical particles.} Using the bosonic and fermionic wavefunctions, we can repeat the same process. We find:

\begin{align*}
	\exv{{x_1}^2} &= \frac{1}{2}\Biggl[\Biggl. \int{x_1}^2|\psi_a(x_1)|^2\dd{x_1}\int|\psi_b(x_2)|^2\dd{x_2} \\ &\qquad\qquad + \int{x_1}^2|\psi_b(x_1)|^2\dd{x_1}\int|\psi_a(x_2)|^2\dd{x_2} \\
	&\qquad\qquad\pm \int{x_1}^2{\psi_a}^*(x_1)\psi_b(x_2)\dd{x_1}\int{\psi_b}^*(x_2)\psi_a(x_2) \dd{x_2} \\ 
	& \qquad\qquad\pm \int{x_1}^2{\psi_b}^*(x_1)\psi_a(x_2)\dd{x_1}\int{\psi_a}^*(x_2)\psi_b(x_2) \dd{x_2} \Biggl.\Biggr] \\[4pt]
	&= \frac{1}{2}\qty[\exv{x^2}_a + \exv{x^2}_b \pm 0 \pm 0] \\[4pt]
	&= \frac{1}{2}\qty(\exv{x^2}_a + \exv{x^2}_b) \,.
\end{align*}

In a similar fashion, it can be shown that $\exv{{x_2}^2} = \exv{{x_1}^2}$ since the particles are indistinguishable. Next,

\begin{align*}
	\exv{x_1x_2} &= \frac{1}{2}\Biggl[\Biggr. \int x_1|\psi_a(x_1)|^2\dd{x_1}\int x_2|\psi_b(x_2)|^2\dd{x_2} \\
	&\qquad\qquad + \int x_1|\psi_b(x_1)|^2\dd{x_1}\int x_2|\psi_a(x_2)|^2\dd{x_2} \\
	&\qquad\qquad \pm \int x_1{\psi_a}^*(x_1)\psi_b(x_1)\dd{x_1}\int x_2{\psi_b}^*(x_2)\psi_a(x_2)\dd{x_2} \\
	&\qquad\qquad \pm \int x_1{\psi_a}^*(x_1)\psi_b(x_1)\dd{x_1}\int x_2{\psi_b}^*(x_2)\psi_a(x_2)\dd{x_2} \Biggl.\Biggr] \\[4pt]
	&= \frac{1}{2}\Bigl[\exv{x}_a\exv{x}_b + \exv{x}_b\exv{x}_a \pm \exv{x}_{ab}\exv{x}_{ba} \pm \exv{x}_{ba}\exv{x}_{ab}\Bigr] \\[4pt]
	&= \exv{x}_a\exv{x}_b \pm |\exv{x}_{ab}|^2 \,,
\end{align*}

where

\begin{equation*}
	\exv{x}_{ab} \equiv \int x {\psi_a}^*(x)\psi_b(x)\dd{x}
\end{equation*}

is known as the exchange integral. Putting this together, we see that:

\begin{equation}
	\exv{(x_1-x_2)^2}_\pm = \exv{x^2}_a + \exv{x^2}_b - 2\exv{x}_a\exv{x}_b \mp 2|\exv{x}_{ab}| \,.
\end{equation}

Comparing the distinguishable and identical results, we see that the difference comes from the final term:

\begin{equation}
	\exv{\Delta x^2}_\pm = \exv{\Delta x^2}_{\text{distinguishable}} \mp 2|\exv{x}_{ab}|^2
\end{equation}

\begin{mdframed}
\paragraph*{Conclusion:}
Identical bosons tend to be closer together, and identical fermions tend to be farther apart, compared to distinguishable.

Note that the exchange integral $\exv{x}_{ab}$ vanishes unless the particles' wavefunctions overlap. Thus when the distance between particles is large, indistinguishability is not an issue.
\end{mdframed}

When there is overlap in wavefunctions, the system behaves as if there is a force of attraction (for bosons) or repulsion (for fermions). This is referred as exchange force (\textit{although this is not really a force}), and is a consequence of the symmetrization requirement. It is strictly a quantum mechanical effect with no classical analogue.

The complete state of a particle includes not only its position wavefunction, but also a spinor describing its spin orientation. It is the total state function, not just the position component, that must obey the symmetrization rule. For an electron, we require

\begin{equation}
	\psi(\vb{r}_1,\vb{r}_2)\chi(1,2) = -\psi(\vb{r}_2,\vb{r}_2)\chi(2,1) \,.
\end{equation}

Recalling the composite spin states for two particles, we see that the singlet state is anti-symmetric whereas the triplet states are symmetric. Two electrons occupying the singlet state must have a symmetric spatial wavefunction whereas two electrons occupying a triplet state must have an anti-symmetric spatial wavefunction.

The singlet state leads to a bonding state, and the the triplet states lead to anti-bonding states.

\subsection{Atoms}
A neutral atom of atomic number $Z$ consists of a heavy nucleus with electric charge $Ze$ surrounded by $Z$ electrons (of mass $m$ and charge $-e$). The Hamiltonian for this system is

\begin{equation}
	\mathcal{H} = \sum_{j=1}^{Z}\qty(-\frac{\hbar^2}{2m}\laplacian{}_j - \frac{1}{4\pi\epsilon_0}\frac{Ze^2}{r_j}) + \frac{1}{2}\frac{1}{4\pi\epsilon_0}\sum_{j\neq}^{Z}\frac{e^2}{|\vb{r}_j - \vb{r}_k|}
\end{equation}

which is composed of kinetic and potential energy terms, as usual, in addition to an additional potential energy term accounting for the electron-electron repulsion. The problem is to solve the Schrödinger equation, $\mathcal{H}\psi = E\psi$ for this situation. Because the electrons are identical, the acceptable states, denoted

\begin{equation}
	\Psi = \psi(\vb{r}_1, \vb{r}_2,\ldots,\vb{r}_Z)\chi(\vb{s}_1, \vb{s}_2,\ldots,\vb{s}_Z)
\end{equation}

must be anti-symmetric with respect to exchange of any two electrons. In particular, no two electrons can occupy the same state. Unfortunately, the general case has not been solved exactly except for the hydrogen atom. We must resort to approximate methods, which we will see in later sections.

After hydrogen, helium is next simplest atom, with $Z=2$. The Hamiltonian is

\begin{equation}
	\mathcal{H} = \qty(-\frac{\hbar^2}{2m}\laplacian{}_{1} - \frac{1}{4\pi\epsilon_0}\frac{2e^2}{r_1}) + \qty(-\frac{\hbar^2}{2m}\laplacian{}_{2} - \frac{1}{4\pi\epsilon_0}\frac{2e^2}{r_2}) + \frac{1}{4\pi\epsilon_0}\frac{e^2}{|\vb{r}_1-\vb{r}_2|}
\end{equation}

which consists of two hydrogenic Hamiltonians and a term for the  repulsion of the two electrons. This repulsion term makes it difficult to solve for the system state, so we will ignore it, and see what result we obtain. In this case, the Hamiltonian separates and we can write the solution as

\begin{equation}
	\psi(\vb{r}_1,\vb{2}_2) = \psi_{n\ell m}(\vb{r}_1)\psi_{n^\prime\ell^\prime m^\prime}(\vb{r}_2)
\end{equation} 

where the single-particle solutions are of the same form as hydrogen, except the Bohr radius is halved and the Bohr energies are increased by a factor of 4. The total energy of the system is

\begin{equation}
	E = 4\qty(-\frac{\SI{13.6}{\eV}}{n^2} - \frac{\SI{13.6}{\eV}}{{n^\prime}^2}) \,.
\end{equation}

The ground state would be 

\begin{equation}
	\psi_{0}(\vb{r}_1,\vb{r}_2) = \psi_{100}(\vb{r}_1)\psi_{100}(\vb{r}_2) = \frac{8\pi}{{a_0}^2}\exp(-\frac{2(r_1+r_2)}{a_0})
\end{equation}

with energy $E = \SI{-109}{\eV}$. Because this spatial wavefunction is symmetric, the spinor must be anti-symmetric and the two electrons must have their spins oppositely aligned.

The experimentally determined ground state of \ce{He} is a singlet state, but the energy is $\SI{-78.975}{\eV}$. There is poor agreement because we have ignored the electron-electron repulsion. This repulsion energy will be positive, and accounts for the higher energy. The excited states will be $\psi_{n\ell m}\psi_{100}$. Exciting both electrons will cause one to drop back to the ground state, releasing enough energy to ionize the atom.

We can construct either symmetric or anti-symmetric spatial wavefunctions which require either anti-symmetric (singlet) or symmetric (triplet) spin states. When a helium atom has a symmetric wavefunction and anti-symmetric spinor, it is known as \textbf{parahelium}. Alternatively, when helium has an anti-symmetric wavefunction and occupies a symmetric spin state, it is known as \textbf{orthohelium}. The ground state is necessarily parahelium, while excited states can take on either configuration.

Symmetric spatial states bring electrons closer together and raises the energy through electrostatic repulsion. This has been verified experimentally.

%% Figure: energy levels of helium

\end{document}