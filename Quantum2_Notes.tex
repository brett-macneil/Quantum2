% Copyright (c) 2020 Brett MacNeil <brett.macneil@dal.ca>

\documentclass[12pt, titlepage]{article}

\usepackage{fullpage}  %Increases margins
\usepackage{amsmath} % math
\usepackage{amssymb} % symbols
\usepackage{physics} % physics
\usepackage{graphicx} % including figures
\usepackage[labelfont=bf]{caption} % figures
\usepackage{float}  % placement of figures
\usepackage{siunitx} % SI units
\usepackage{mhchem} % chemical symbols
\usepackage{mdframed} % frames
\usepackage{xcolor} % colors
\usepackage{cancel} % cancelto
\usepackage{hyperref} % references and links
\usepackage{mathrsfs} % mathscr
\hypersetup{colorlinks=false, linkbordercolor=1 1 1}

\newcommand{\exv}[1]{\left\langle #1 \right\rangle}
\newcommand{\Sx}{\begin{bmatrix} 0 & 1 \\ 1 & 0\end{bmatrix}}
\newcommand{\Sy}{\begin{bmatrix} 0 & -i \\ i & 0\end{bmatrix}}
\newcommand{\Sz}{\begin{bmatrix} 1 & 0 \\ 0 & -1\end{bmatrix}}

\DeclareMathOperator{\Ai}{Ai} % Airy function
\DeclareMathOperator{\Bi}{Bi} % Airy function

\setlength{\parindent}{0pt} % no auto-indents
\setlength{\parskip}{8pt}

\begin{document}
\title{\huge Quantum Physics Notes \vspace{2cm} \\  \large Adapted from Dr. Jesse Maasen's lecture notes\thanks{As well as \emph{Introduction to Quantum Mechanics}, 3rd ed. by Griffiths and Schroeter.} \\ \vspace{0.5cm} \normalsize for \vspace{0.5cm} \\  PHYC 4151 \vspace{2cm}}
\author{\huge Brett MacNeil \\ \vspace{0.5cm} \normalsize Dalhousie University}
\date{Fall 2020 \\ \vspace{1cm}
	\normalsize Last updated: \today}
\maketitle

\section{Spin}
\subsection{Introduction to spin}
In classical mechanics, a rigid object can have two kinds of angular momentum:

\begin{description}
	\item Orbital: $\vb{L} = \vb{r}\cross\vb{p}$
	\item Spin : $\vb{S} = I\vb*{\omega}$
\end{description}

%% Figure: L and S for Earth orbit

Classically, both are equivalent; spin is simply the sum of the angular momenta of infinitesimal elements over the volume of a non-point mass. In quantum mechanics, we have the same two components of angular momentum, but the distinction is absolutely fundamental.

There is orbital angular momentum, for example associated with the motion of an electron around the nucleus in the hydrogen atom, for example. This is described by the spherical harmonics for a spherically symmetric potential.

The electron also carries another form of angular momentum which has nothing to do with motion in space and is independent of position ($r$, $\phi$, $\theta$) but is somewhat analogous to classical spin. This analogy cannot be pressed too far, as quantum mechanical spin cannot be decomposed into the orbital angular momentum of constituent parts.

We start from the fact that elementary particles carry \emph{intrinsic} angular momentum $\vb{S}$ in addition to their \emph{extrinsic} angular momentum $\vb{L}$. The algebraic theory of spin is similar to the theory of orbital angular momentum.

\subsection{Review of orbital angular momentum}
\subsubsection{Definitions of orbital angular momentum}
Orbital angular momentum is defined as the cross product of the position and linear momentum vector:

\begin{equation}
\vb{L} = \vb{r}\cross\vb{p} \, .
\end{equation}

%% Figure: angular momentum vector

In Cartesian components, this is

\begin{subequations}
	\begin{align}
	L_{x} &= y p_{z} - z p_{y} \\
	L_{y} &= z p_{x} - x p_{x} \\
	L_{z} &= x p_{y} - y p_{x} \, ,
	\end{align}
\end{subequations}

and in quantum mechanics, we use the prescription

\begin{subequations}
	\begin{align}
	p_{x} &= - i \hbar \pdv*{x} \\
	p_{y} &= - i \hbar \pdv*{y} \\
	p_{z} &= - i \hbar \pdv*{z} \, .
	\end{align}
\end{subequations}

An important consequence is that the angular momentum operators do not commute:

\begin{subequations}
	\begin{align}
	\comm{L_{x}}{L_{y}} &= i \hbar L_{z} \\
	\comm{L_{y}}{L_{z}} &= i \hbar L_{x} \\
	\comm{L_{z}}{L_{x}} &= i \hbar L_{y} \, .
	\end{align}
\end{subequations}

The observables $L_{x}$, $L_{y}$ and $L_{z}$ are incompatible and for example, the uncertainty principle says

\begin{equation}
	\sigma_{L_{x}}\sigma_{L_{y}} \geq \frac{\hbar}{2}\left|\exv{L_{z}}\right| \, .
\end{equation}

These  $L_{x}$, $L_{y}$ and $L_{z}$ operators do not share the same eigenfunctions. However, the total angular momentum,

\begin{equation}
	L^2 = L_{x}^{\, 2} + L_{y}^{\, 2} + L_{z}^{\, 2}
\end{equation}

does commute with each of the component angular momentum operators,

\begin{subequations}
	\begin{align}
	\comm{L^2}{L_{x}} &= 0 \\
	\comm{L^2}{L_{y}} &= 0 \\
	\comm{L^2}{L_{z}} &= 0 \, .
	\end{align}
\end{subequations}

 and therefore they share the same eigenfunctions.
 
 \subsubsection{Eigenvalues and Eigenfunctions}
 The eigenvalue equations for the angular momentum operators are 
 
 \begin{align}
 	L^2 Y_{\ell}^{m} &= \hbar^2\ell(\ell+1) Y_{\ell}^{m} \\
 	L_{z} Y_{\ell}^{m} &= \hbar m Y_{\ell}^{m} \, ,
 \end{align}
 
 where $\ell$ is the azimuthal quantum number ($\ell=0,1,2,\ldots$) and $m$ is the magnetic quantum number ($m=-\ell,-\ell+1,\ldots,\ell-1,\ell$). The eigenfunctions $Y_{\ell}^{m}$ are functions of $\phi$ and $\theta$ and are known as the \emph{spherical harmonics}.
 
 %% Figure: Classical picture of l = 2
 
We can also define ladder operators,

\begin{equation}
	L_{\pm} = L_{x} \pm i L_{y}
\end{equation}

%% Figure: Ladder operator

These operators raise or lower the eigenvalues and eigenstates of $L_{z}$ . Of course, applying the raising operator to the top state, or applying the lowering operator to the bottom state gives zero:

\begin{subequations}
\begin{align}
	L_{+}\ket{\ell, m_{\text{max}}} &= 0 \\
	L_{-}\ket{\ell, m_{\text{min}}} &= 0 \, .
\end{align}
\end{subequations}

The ladder operators are not hermitian and are thus not observable quantities.
\subsection{Theory of spin}
 We begin with the fundamental commutation relations for spin, which is analogous to those of orbital angular momentum:
 
 \begin{subequations}
 	\begin{align}
 	\comm{S_{x}}{S_{y}} &= i \hbar S_{z} \\
 	\comm{S_{y}}{S_{z}} &= i \hbar S_{x} \\
 	\comm{S_{z}}{S_{x}} &= i \hbar S_{y} \, .
 	\end{align}
 \end{subequations}

We will take these as postulates for the theory of spin. The eigenvectors of $S^2$ and $S_{z}$ satisfy:

\begin{align}
	S^2 \ket{s, m} &= \hbar^2 s(s+1)\ket{s, m} \\
	S_{z} \ket{s, m} &= \hbar m \ket{s, m}
\end{align}

where $\ket{s, m}$ is an eigenvector of both $S^2$ and $S_z$, which has no dependence on $\phi$ or $\theta$. Since the eigenstates of spin are not functions, we represent the eigenvector in Dirac notation rather than writing $Y_{s}^{m}$ as we did for the spherical harmonics.

%% Figure: Classical picture for s = 1

We also have spin ladder operators

\begin{equation}
	S_{\pm} = S_{x} \pm i S_{y} \,,
\end{equation}

where

\begin{equation}
	S_{\pm}\ket{s, m} = \hbar\sqrt{s(s+1)-m(m\pm1)}\ket{s, m\pm1} \, .
\end{equation}

The spin ladder operators change $m$ by $\pm 1$.

%% Figure: Spin ladder operators

In general, $s$ can be an integer or half integer, whereas for orbital angular momentum, $\ell$ could only be an integer.

\begin{equation}
	\left\{
	\begin{array}{l}
	s = 0, \, 1/2, \, 1, \, 3/2, \, \ldots \\
	m = -s, \, -s+1, \, \ldots, \, s-1, \, s
	\end{array}
	\right.
\end{equation}

Every particle has a specific and immutable value of $s$, which we call spin. By contrast, the orbital angular momentum quantum number $\ell$ can take on any integer value, and may change when the system is perturbed. \textbf{Spin is fixed}!

\subsection{Spin 1/2}
The case where $s=1/2$ is the most important to study, since ordinary matter (electrons, protons, neutrons) have this spin. There are only two eigenstates.

\begin{description}
	\item Spin up ($\uparrow$): $s=\frac{1}{2}$, $m=+\frac{1}{2}$
	\item Spin down ($\downarrow$): $s=\frac{1}{2}$, $m=-\frac{1}{2}$
\end{description}

Using these two eigenstates as a basis, we can express the general state of a spin-$1/2$ particle as a two-component vector, known as a \textbf{spinor}:

\begin{equation}
	\chi = \begin{bmatrix} a \\ b \end{bmatrix} = a\chi_+ + b\chi_- 
\end{equation}

where the basis states are

\begin{equation}
	\chi_+ = \begin{bmatrix} 1 \\ 0 \end{bmatrix} \quad \text{(spin up)} \qquad;\qquad \chi_- = \begin{bmatrix} 0 \\ 1 \end{bmatrix} \quad \text{(spin down)} \,.
\end{equation}

This means that the spin operators are $2 \times 2$ matrices and we can determine their form by examining their effect on the basis spinors. Since $S^2\ket{s, m} = \hbar^2s(s+1)\ket{s, m}$, it follows that for $s=1/2$,

\begin{equation}
	S^2\chi_+ = \frac{3}{4}\hbar^2\chi_+ \quad\text{and}\quad S^2\chi_- = \frac{3}{4}\hbar^2\chi_- \,.
\end{equation}

If we consider a general matrix $S^2 = \begin{bmatrix} s_{11} & s_{12} \\ s_{21} & s_{22}\end{bmatrix}$ then we find

\begin{align*}
	S^2\chi_+ = \begin{bmatrix} s_{11} & s_{12} \\ s_{21} & s_{22}\end{bmatrix}\begin{bmatrix} 1 \\ 0\end{bmatrix} = \frac{3}{4}\hbar^2\begin{bmatrix} 1 \\ 0\end{bmatrix} \quad\implies\quad \begin{bmatrix} s_{11} \\ s_{21}\end{bmatrix}=\begin{bmatrix} 3\hbar^2/4 \\ 0\end{bmatrix}\\[8pt]
	S^2\chi_- = \begin{bmatrix} s_{11} & s_{12} \\ s_{21} & s_{22}\end{bmatrix}\begin{bmatrix} 0 \\ 1\end{bmatrix} = \frac{3}{4}\hbar^2\begin{bmatrix} 0 \\ 1\end{bmatrix} \quad\implies\quad \begin{bmatrix} s_{12} \\ s_{22}\end{bmatrix}=\begin{bmatrix} 0 \\ 3\hbar^2/4 \end{bmatrix}
\end{align*}

which means that

\begin{equation}
	S^2 = \frac{3}{4}\hbar^2\begin{bmatrix} 1 & 0 \\ 0 & 1\end{bmatrix} \,.
\end{equation}

Additionally, since $S_z\ket{s, m} = \hbar m \ket{s, m}$ we have

\begin{equation}
S_z\chi_+ = \frac{1}{2}\hbar\chi_+ \quad\text{and}\quad S_z\chi_- = -\frac{1}{2}\hbar\chi_- \,.
\end{equation}

By similar analysis as above, we find that the matrix representation of the $S_z$ operator is

\begin{equation}
	S_z = \frac{\hbar}{2}\Sz \,.
\end{equation}

Meanwhile, the eigenvalue equations for the spin ladder operators are

\begin{equation*}
	S_+\chi_- = \hbar\chi_+ \quad;\quad S_-\chi_+ = \hbar\chi_- \quad;\quad S_+\chi_+ = S_-\chi_- = 0
\end{equation*}

and therefore it can be shown that

\begin{equation}
	S_+ = \hbar\begin{bmatrix} 0 & 1 \\ 0 & 0\end{bmatrix} \quad\text{and}\quad S_- = \hbar\begin{bmatrix} 0 & 0 \\ 1 & 0\end{bmatrix} \,.
\end{equation}

Now since $S_\pm = S_x \pm iS_y$, we find $S_x = (S_+ + S_-)/2$ and $S_y = (S_+ - S_-)/2i$. This gives

\begin{equation}
	S_x = \frac{\hbar}{2}\Sx \quad\text{and}\quad S_y = \frac{\hbar}{2}\Sy \,.
\end{equation}

Since the spin operators all carry a factor of $\hbar/2$, it is tidier to write

\begin{equation}
	\vb{S} = \frac{\hbar}{2}\vb*{\sigma}
\end{equation}

where $\vb*{\sigma}$ is given by the \textbf{Pauli spin matrices}:

\begin{equation}
\boxed{
	\sigma_x = \Sx \quad,\quad
	\sigma_y = \Sy \quad,\quad
	\sigma_z = \Sz }
\end{equation}

The $S_x$, $S_y$, and $S_z$ operators are Hermitian, since they represent observables. The ladder operators, on the other hand, are not Hermitian as they do not represent observables. 

The eigenspinors of $S_z$ are, by definition, $\chi_+$ and $\chi_-$. If we measure $S_z$ on a particle in some general state $\chi = a\chi_+ + b\chi_-$ then we will measure $=\hbar/2$ with probability $|a|^2$ or $-\hbar/2$ with probability $|b|^2$. Those are the only two choices, so the normalization condition imposes

\begin{equation}
	|a|^2 + |b|^2 = 1
\end{equation}

since $\bra{\chi}\ket{\chi} = 1$. 

\begin{mdframed}
\paragraph*{Note:} The probability of observing a state $\ket{\chi_\pm}$ after measuring a particle in superposition state $\ket{\chi}$ is given by the norm of the projection of the measured state onto the particle state. In other words,

\begin{equation}
	P_\pm = \qty|\bra{\chi_\pm}\ket{\chi}|^2
\end{equation}

where the inner product for these spinors involves matrix multiplication, as the basis is discrete.
\end{mdframed}

What if we choose to measure $S_x$? What are the possible results, and the associated probabilities? According to the generalized statistical interpretation, we must know the eigenvalues and eigenspinors of $S_x$. The characteristic equation is:

\begin{equation*}
	S_x\chi = \lambda\chi \quad\implies\quad \det(S_x-\lambda\mathbb{I}) = 0 \quad\implies \begin{vmatrix}-\lambda & \hbar/2 \\ \hbar/2 & -\lambda \end{vmatrix} = 0
\end{equation*}

which yields eigenvalues of $\lambda \pm\hbar/2$, as expected. The eigenspinors are then calculated by:

\begin{equation*}
	\frac{\hbar}{2}\begin{bmatrix}0 & 1 \\ 1 & 0 \end{bmatrix}\begin{bmatrix} \alpha \\ \beta \end{bmatrix} = \pm\frac{\hbar}{2}\begin{bmatrix} \alpha \\ \beta \end{bmatrix} \quad\implies \begin{bmatrix} \beta \\ \alpha \end{bmatrix} = \pm\begin{bmatrix} \alpha \\ \beta \end{bmatrix} \quad\implies\quad \beta = \pm\alpha
\end{equation*}

so that we can write $\chi_{\pm}^{(x)} = \alpha \begin{bmatrix} 1 \\ \pm 1\end{bmatrix}$ and applying the normalization condition gives $\alpha = 1/\sqrt{2}$. This gives

\begin{equation}
	\chi_+^{(x)} = \frac{1}{\sqrt{2}}\begin{bmatrix} 1 \\ 1\end{bmatrix} \quad;\quad \chi_-^{(x)} = \frac{1}{\sqrt{2}}\begin{bmatrix} 1 \\ -1\end{bmatrix}
\end{equation}

Recall that the eigenvectors of a Hermitian matrix span the space and form a complete basis. We can use a linear combination of them to express a generic spinor. If we change the basis to the eigenspinors of $S_x$, we get

\begin{equation}
	\chi = \bra{\chi_+^{(x)}}\ket{\chi}\chi_+^{(x)} + \bra{\chi_-^{(x)}}\ket{\chi}\chi_-^{(x)}
\end{equation}

which gives

\begin{equation}
	\chi = \qty(\frac{a+b}{\sqrt{2}})\chi_+^{(x)} + \qty(\frac{a-b}{\sqrt{2}})\chi_-^{(x)} \,.
\end{equation}
Measuring $S_x$ will yield $+\hbar/2$ with probability $|a+b|^2/2$ and $-\hbar/2$ with probability $|a-b|^2/2$.
\clearpage

\begin{mdframed}[backgroundcolor=gray!20]
\paragraph*{Example:}
Suppose a spin-1/2 particle is in the state:

\begin{equation*}
	\chi = \frac{1}{\sqrt{6}} \begin{bmatrix} 1+i \\ 2 \end{bmatrix} \,.
\end{equation*}

What are the probabilities of obtaining $+\hbar/2$ and $-\hbar/2$ when measuring $S_z$ ans $S_x$?

\paragraph*{Solution:}
Here, $a = (1+i)/\sqrt{6}$ and $b = 2/\sqrt{6}$. 

For $S_z$, the probability of measuring $+\hbar/2$ is 

\begin{equation*}
	|a|^2 = \qty|\frac{1+i}{\sqrt{6}}|^2 = 1/3 \,.
\end{equation*}

The probability of measuring $-\hbar/2$ is

\begin{equation*}
	|b|^2 = \qty|\frac{2}{\sqrt{6}}|^2 = 2/3 \,.
\end{equation*}

For $S_x$, the probability of measuring $+\hbar/2$ is 

\begin{equation*}
	\frac{|a+b|^2}{2} = \frac{|(3+i)/\sqrt{6}|^2}{2} = 5/6 \,.
\end{equation*}

The probability of measuring $-\hbar/2$ is

\begin{equation*}
	\frac{|a-b|^2}{2} = \frac{|(-1+i)/\sqrt{6}|^2}{2} = 1/6 \,.
\end{equation*}
\end{mdframed}

\begin{mdframed}[backgroundcolor=gray!20]
\paragraph*{Problem:} (a) Show that the eigenspinors of $S_y$ are 

\begin{equation}
\chi_+^{(y)} = \frac{1}{\sqrt{2}}\begin{bmatrix} 1 \\ i\end{bmatrix} \quad;\quad \chi_-^{(x)} = \frac{1}{\sqrt{2}}\begin{bmatrix} 1 \\ -i \end{bmatrix} \,.
\end{equation}

(b) If you measure $S_y$ on a particle in a general state $\chi$, what values might you get, and what is the probability of each?
\end{mdframed}

\subsection{Electron in a magnetic field}
A charged particle with spin constitutes a magnetic dipole. The \textbf{magnetic dipole moment}, of a particle, denoted by $\vb*{\mu}$, is proportional to its spin angular momentum $\vb{S}$:

\begin{equation}
    \vb*{\mu} = \gamma \vb{S} \,,
\end{equation}

where the proportionality constant $\gamma$ is the \textbf{gyromagnetic ratio}. When a magnetic dipole is placed in a magnetic field $\vb{B}$, it experiences a torque $\vb{\Gamma} = \vb*{\mu}\cross\vb{B}$ which tends to line it up parallel to the field. The associated energy is known as the Zeeman energy,

\begin{equation}
	H = -\vb*{\mu}\vdot\vb{B} \,,
\end{equation}

so that the Hamiltonian of a charged particle with nonzero spin in a magnetic field is

\begin{equation}
	\mathcal{H} = -\gamma\vb{B}\vdot\vb{S} \,,
\end{equation}

where $\vb{S}$ is the appropriate spin matrix. 

\subsubsection{Larmor precession}
Imagine a particle of spin $1/2$, which points in the $z-$ direction,

\begin{equation*}
	\vb{B} = B_0\vu{z},.
\end{equation*}

The Hamiltonian is therefore

\begin{equation}
	\mathcal{H} = -\gamma B_0 S_z = -\frac{\gamma B_0 \hbar}{2} \begin{bmatrix} 1 & 0 \\ 0 & -1 \end{bmatrix} \,.
\end{equation}

Since the Hamiltonian is a scalar multiple of the $S_z$ operator, it is trivial to show that they commute. Therefore, the eigenstates of $\mathcal{H}$ are the same as those of $S_z$:

\begin{equation}
    \left\{
    \begin{array}{cc}
    \chi_+ \,, & \text{with energy} E_+ = -\gamma B_0\hbar/2 \\[6pt]
    \chi_- \,, & \text{with energy} E_- = \gamma B_0\hbar/2
    \end{array}
    \right.
\end{equation}

The energy is lowest when the dipole moment is parallel to the field, just as it would be classically. Since the Hamiltonian is time-independent, the general solution to the time-dependent Schrödinger equation $\mathcal{H}\chi = i\hbar\pdv*{\chi}{t}$ can be expressed in terms of stationary states:

\begin{equation*}
	\chi(t) = a\chi_+\exp(-i E_+ t/\hbar) + b\chi_-\exp(-i E_-t/\hbar) = \begin{bmatrix} a\exp(i\gamma B_0 t/2) \\[4pt] b\exp(-i\gamma B_0 t/2) \end{bmatrix}
\end{equation*}

where the constants $a$ and $b$ are determined by initial conditions

\begin{equation*}
\chi(0) = \begin{bmatrix} a \\ b \end{bmatrix}
\end{equation*}

and due to normalization, $|a|^2 + |b|^2 = 1$. With no loss of generality we may assume that $a$ and $b$ are real, and write $a = \cos(\alpha/2)$ and $b = \sin(\alpha/2)$, where $\alpha$ is a fixed angle. This gives

\begin{equation}
	\chi(t) = \begin{bmatrix} \cos(\alpha/2) \exp(-i\gamma B_0 t/2) \\[4pt] \sin(\alpha/2) \exp(-i\gamma B_0 t/2) \end{bmatrix} \,.
\end{equation}

To see what is happening, we can calculate the expectation value of $\vb{S}$ as a function of time:

\begin{align*}
	\exv{S_x} &= \chi^\dagger S_x \chi \\[4pt]
	&= \begin{bmatrix} \cos(\alpha/2)\exp(-i\gamma B_0 t/2) \\[4pt] \sin(\alpha/2)\exp(i\gamma B_0 t/2) \end{bmatrix}^{\text{T}} \frac{\hbar}{2} \Sx \begin{bmatrix} \cos(\alpha/2)\exp(i\gamma B_0 t/2) \\[4pt] \sin(\alpha/2)\exp(-i\gamma B_0 t/2) \end{bmatrix} \\[4pt]
	&= \frac{\hbar}{2}\sin\alpha\cos(\gamma B_0 t)
\end{align*}

Similarly,

\begin{equation*}
	\exv{S_y} = -\frac{\hbar}{2}\sin\alpha\sin(\gamma B_0 t) \quad\text{and}\quad \exv{S_z} = \frac{\hbar}{2}\cos\alpha \,.
\end{equation*}

Thus $\exv{\vb{S}}$ is tilted at a constant angle $\alpha$ to the $z$-axis, and precesses about the field at \textbf{Larmor frequency}

\begin{equation}
	\omega = \gamma B_0 \,,
\end{equation}

just as a classical dipole would. 
\clearpage

\begin{mdframed}
\paragraph*{Note:} This is a consequence of Ehrenfest's theorem, which relates the time derivative of an operator to its commutator with the Hamiltonian:

\begin{equation}
	\dv{t}\exv{\mathcal{A}} = \frac{1}{i\hbar}\exv{\comm{\mathcal{A}}{\mathcal{H}}} + \exv{\pdv{\mathcal{A}}{t}}
\end{equation}

Using this relation, one can obtain relations of the same form as classical mechanics:

\begin{align}
	\dv{\exv{\vb{p}}}{t} &= \exv{-\grad{V}} \,; \\[4pt]
	\dv{\exv{\vb{J}}}{t} &= \exv{\vb{\Gamma}} = \exv{\vb{r}\cross(-\grad{V})} \,,
\end{align}

where $\exv{-\grad{V}}$ is analogous to a force, and $\vb{J} = \vb{L} + \vb{S}$ is the total angular momentum. 
\end{mdframed}

\begin{mdframed}[backgroundcolor=gray!20]
\paragraph*{Problem:} Consider a charged particle with spin $1/2$ in a uniform magnetic field of magnitude $B_0$ along the $z$-direction. 

(a) What is the probability, as a function of time, of measuring the $x$-component of the spin angular momentum to be $+\hbar/2$?

(b) Repeat (a), for the $y$-component.

(c) Repeat (a) again, for the $z$-component.
\end{mdframed}

\subsubsection{The Stern-Gerlach experiment}
In an \textit{inhomogeneous} magnetic field, there is also a force on a magnetic dipole. From electrodynamics:

\begin{equation}
	\vb{F} = \grad(\vb*{\mu}\vdot\vb{B}) \,.
\end{equation}

This force can be used to separate out particles with a particular spin orientation. WE consider a beam of heavy neutral atoms, so we can avoid relativistic effects and large deflection from the Lorentz force. 

%% Figure: Stern-Gerlach experiment

The particles travel in the $y$-direction through a region of static but inhomogeneous magnetic field given by

\begin{equation}
	\vb{B}(x, y, z) = -\alpha x \vu{x} + (B_0 + \alpha z)\vu{z} \,,
\end{equation}

where $B_0$ is a large uniform field and $\alpha$ is a small deviation. Ideally, the field would just be along $\vu{z}$, however an $x$-component is required to ensure that $\div{\vb{B}} = 0$. There is a net force on the beam in the $z$-direction and the beam will be deflected up or down depending on $S_z$. Classically, one would observe a continuum of deflections, however since spin is quantized, there will be $2s+1$ separate streams. This was demonstrated by Stern and Gerlach using silver atoms. The inner elections are paired such that their orbital and spin angular momenta cancel. The net atomic spin comes from one unpaired valence electron, giving the overall atom $s=1/2$.

From the frame of reference of the silver atoms, the magnetic Hamiltonian is initially zero and turns on for some time $T$ as they pass through the magnet. Afterward, the Hamiltonian is again zero. This means

\begin{equation}
	\mathcal{H} = \left\{
	\begin{array}{ll}
	0 \,, & t<0 \\
	-\gamma(B_0 +\alpha z)S_z \,, & 0 \leq t \leq T \\
	0 \,, & t > T 
	\end{array}
	\right. \, .
\end{equation}

We ignore the $x$-component of the field as Larmor precession causes the expectation value $\exv{S_x}$ to average to zero since the precession frequency is large. If the atoms start in state $\chi = a\chi_+ + b\chi_-$ for $t<0$, the spin state will evolve while the Hamiltonian acts,

\begin{equation*}
	\chi(t) = a\chi_+\exp(-iE_+ t/\hbar) + b\chi_-\exp(-iE_- t/\hbar) \quad,\quad E_\pm = \mp\frac{\gamma\hbar}{2}(B_0+\alpha z) \quad\text{and}\quad 0 \leq t \leq T \,.
\end{equation*}

Hence, it emerges from the magnetic field in state

\begin{equation*}
\chi(t) = \qty(a\exp(i\gamma  T B_0/2)\chi_+)\exp(i\alpha\gamma T/2) + \qty(b\exp(-i\gamma  T B_0/2)\chi_-)\exp(-i\alpha\gamma T/2)
\end{equation*}

and is frozen in this state for $t>T$. Applying the momentum operator, we can see that the spin-up and spin-down particles have different momentum,

\begin{equation}
	\left\{
	\begin{array}{l}
	P_{z}^{\uparrow} = \alpha \gamma T \hbar/2 \\
	P_{z}^{\downarrow} = -\alpha \gamma T \hbar/2
	\end{array}
	\right.
\end{equation}

and the beam splits in two.

In this problem, we assumed that the initial state of a spin system is known and quantum mechanics tells us how it \textit{evolves}. To prepare an ensemble of atoms in a particular spin state, a beam can be passed through a Stern-Gerlach magnet and one particular beam can be selected.

\subsection{Addition of angular momentum}
Suppose now that we have two particles of spins $s_1$ and $s_2$. The first is in state $\ket{s_1, m_1}$ and the second is in state $\ket{s_2, m_2}$. We can denote the composite state as $\ket{s_1s_2,m_1m_2}$, where

\begin{align*}
	{S^{(1)}}^2\ket{s_1s_2,m_1m_2} &= s_1(s_1+1)\hbar^2\ket{s_1s_2,m_1m_2} \\
	{S^{(2)}}^2\ket{s_1s_2,m_1m_2} &= s_2(s_2+1)\hbar^2\ket{s_1s_2,m_1m_2} \\
	S^{(1)}_{z}\ket{s_1s_2,m_1m_2} &= m_1\hbar\ket{s_1s_2,m_1m_2} \\
	S^{(2)}_{z}\ket{s_1s_2,m_1m_2} &= m_2\hbar\ket{s_1s_2,m_1m_2} \,,
\end{align*}

as expected. The total $z$-component of the spin,

\begin{align*}
	S_z\ket{s_1s_2,m_1m_2} &= (S^{(1)}_{z} + S^{(2)}_{z})\ket{s_1s_2,m_1m_2} \\
	&= S^{(1)}_{z}\ket{s_1s_2,m_1m_2} + S^{(2)}_{z}\ket{s_1s_2,m_1m_2} \\
	&= m_1\hbar\ket{s_1s_2,m_1m_2} + m_2\ket{s_1s_2,m_1m_2} \\
	&= (m_1+m_2)\ket{s_1s_2,m_1m_2} \,,
\end{align*}

is simply the sum of the two particles. However, the net spin, $s$ is much less subtle. 

\subsubsection{Two spin-1/2 particles}
We consider the case of two spin-1/2 particles, such as the proton and electron in the hydrogen atom. Each can be either spin-up or spin-down so there are four possibilities in all:

\begin{align*}
	\ket{\uparrow\uparrow} &= \ket{\frac{1}{2}\frac{1}{2},\frac{1}{2}\frac{1}{2}} &&m=1 \\
	\ket{\uparrow\downarrow} &= \ket{\frac{1}{2}\frac{1}{2},\frac{1}{2}\frac{-1}{2}} &&m=0 \\
	\ket{\downarrow\uparrow} &= \ket{\frac{1}{2}\frac{1}{2},\frac{-1}{2}\frac{1}{2}} &&m=0 \\
	\ket{\downarrow\downarrow} &= \ket{\frac{1}{2}\frac{1}{2},\frac{-1}{2}\frac{-1}{2}}  &&m=-1 \,.
\end{align*}

Since $m=-s,-s+,\ldots,s-1,s$ it appears that in this case $s=1$ however there is an extra $m=0$ state. To gain insight into this, we can apply the lowering operator, $S_- = S_{-}^{(1)} + S_{-}^{(2)}$ to the state $\ket{\uparrow\uparrow}$:

\begin{align*}
	S_- \ket{\uparrow\uparrow} &= \qty(S_{-}^{(1)}\ket{\uparrow})\ket{\uparrow} + \ket{\uparrow}\qty(S_{-}^{(2)}\ket{\uparrow}) \\
	&= \hbar\ket{\downarrow}\ket{\uparrow} + \ket{\uparrow}\hbar\ket{\downarrow} \\
	&= \hbar(\ket{\downarrow\uparrow} - \ket{\uparrow\downarrow}) \,.
\end{align*}

Applying the lowering operator again to this state gives:

\begin{align*}
	S_-(\ket{\downarrow\uparrow} - \ket{\uparrow\downarrow}) &= S_{-}^{(1)}\ket{\uparrow\downarrow} + S_{-}^{(2)}\ket{\uparrow\downarrow} + S_{-}^{(1)}\ket{\downarrow\uparrow} + S_{-}^{(2)}\ket{\downarrow\uparrow} \\
	&= \qty(S_{-}\ket{\uparrow})\ket{\downarrow} + \ket{\uparrow}\cancelto{0}{\qty(S_{-}\ket{\downarrow})} + \cancelto{0}{\qty(S_{-}\ket{\downarrow})\ket{\uparrow}} + \ket{\downarrow}\qty(S_{-}\ket{\uparrow}) \\
	&= 2\hbar\ket{\downarrow\downarrow}
\end{align*}

and applying the lowering operator again gives zero. This means that we have three states with $s=1$:

\begin{equation}
	\left\{
	\begin{array}{l}
	\ket{1,1} = \ket{\uparrow\uparrow} \\[2pt]
	\ket{1,0} = \frac{1}{\sqrt{2}}(\ket{\uparrow\downarrow} + \ket{\downarrow\uparrow}) \\[2pt]
	\ket{1, -1} = \ket{\downarrow\downarrow}
	\end{array}
	\right. 
	\qquad \boxed{s=1}
\end{equation}

This is called the \textbf{triplet} combination.

\begin{mdframed}
\paragraph*{Note:} The state $\chi = \ket{\uparrow\downarrow} + \ket{\downarrow\uparrow}$ is not normalized. We must determine the inner product of $\chi$ with itself in order to normalize it:

\begin{align*}
	\bra{\chi}\ket{\chi} &= \bigl(\ket{\uparrow\downarrow} + \ket{\downarrow\uparrow}\bigr)^\dagger\bigl(\ket{\uparrow\downarrow} + \ket{\downarrow\uparrow}\bigr) \\
	&= \bigl(\bra{\uparrow\downarrow} + \bra{\downarrow\uparrow}\bigr)\bigl(\ket{\uparrow\downarrow} + \ket{\downarrow\uparrow}\bigr) \\
	&= \bra{\uparrow\downarrow}\ket{\uparrow\downarrow} + \bra{\uparrow\downarrow}\ket{\downarrow\uparrow} + \bra{\downarrow\uparrow}\ket{\uparrow\downarrow} + \bra{\downarrow\uparrow}\ket{\downarrow\uparrow} \\
	&= \bra{\uparrow}\ket{\uparrow}\bra{\downarrow}\ket{\downarrow} + \bra{\uparrow}\ket{\downarrow}\bra{\downarrow}\ket{\uparrow} + \bra{\downarrow}\ket{\uparrow}\bra{\uparrow}\ket{\downarrow} + \bra{\downarrow}\ket{\downarrow}\bra{\uparrow}\ket{\uparrow} \\
	&= (1)(1) + 0 + 0 + (1)(1) \\
	&= 2
\end{align*}

Therefore, we write $\chi = \frac{1}{\sqrt{2}}(\ket{\uparrow\downarrow} + \ket{\downarrow\uparrow})$ as the normalized spinor.
\end{mdframed}
\clearpage

The leftover orthogonal state with $m=0$ carries $s=0$. It is normalized in a similar manner as demonstrated previously. 

\begin{equation}
	\left\{
	\begin{array}{l}
	\ket{0,0} = \frac{1}{\sqrt{2}}(\ket{\uparrow\downarrow} - \ket{\downarrow\uparrow})
	\end{array}
	\right.
	\qquad \boxed{s=0}
\end{equation}

This is called the \textbf{singlet} state and applying both ladder operators will give zero.

It appears that the \textit{combination} of the two spin=1/2 particles can carry a spin of 1 or 0, depending on whether they occupy the singlet or triplet configuration. To confirm this, we must show that the triplet states are eigenvectors of $S^2$ with eigenvalue $2\hbar^2$ and applying $S^2$ to the singlet state gives zero.

The $S^2$ operator is

\begin{align}
	S^2 &= \qty(S^{(1)} + S^{(2)})\vdot\qty(S^{(1)} + S^{(2)}) \nonumber \\
	&= \qty(S^{(1)})^2 + \qty(S^{(2)})^2 + 2 S^{(1)}\vdot S^{(2)} \nonumber \\
	&= \qty(S^{(1)})^2 + \qty(S^{(2)})^2 + 2\qty(S_{x}^{(1)}S_{x}^{(2)} + S_{y}^{(1)}S_{y}^{(2)} + S_{z}^{(1)}S_{z}^{(2)}) \,.
\end{align}

For the $\ket{1,1}$ state, we have:

\begin{align*}
	S^2\ket{1,1} &= \qty(\qty(S^{(1)})^2 + \qty(S^{(2)})^2 + 2S^{(1)}\vdot S^{(2)})\ket{\uparrow\uparrow} \\
	&= (S^2\ket{\uparrow})\ket{\uparrow} + \ket{\uparrow}(S^2)\ket{\uparrow} + 2\qty((S_x\ket{\uparrow})(S_x\ket{\uparrow}) + (S_y\ket{\uparrow})(S_y\ket{\uparrow}) + (S_z\ket{\uparrow})(S_z\ket{\uparrow})) \\
	&= \frac{3\hbar^2}{4}\ket{\uparrow} + \frac{3\hbar^2}{4}\ket{\uparrow} + 2\qty(\frac{\hbar}{2}\ket{\downarrow}\frac{\hbar}{2}\ket{\downarrow} + \frac{i\hbar}{2}\ket{\downarrow}\frac{i\hbar}{2}\ket{\downarrow} + \frac{\hbar}{2}\ket{\uparrow}\frac{\hbar}{2}\ket{\uparrow}) \\
	&= 2\hbar^2\ket{\uparrow\uparrow} \\
	&= 2\hbar^2\ket{1,1} \,,
\end{align*}

as required. The proof for the $\ket{1, 0}$ and $\ket{1, -1}$ states follow the same approach. For the singlet state,

\begin{align*}
	S^2\ket{0,0} &= 0 + 0 + \qty(2S^{(1)}\vdot S^{(2)})\frac{1}{\sqrt{2}}(\ket{\uparrow\downarrow}-\ket{\downarrow\uparrow}) \\
	&= \sqrt{2}\bigl(S_x\ket{\uparrow}S_x\ket{\downarrow} + S_y\ket{\uparrow}S_y\ket{\downarrow} + S_z\ket{\uparrow}S_z\ket{\downarrow} \bigr. \\
	& \qquad\qquad \bigl.- S_x\ket{\downarrow}S_x\ket{\uparrow} + S_y\ket{\downarrow}S_y\ket{\uparrow} + S_z\ket{\downarrow}S_z\ket{\uparrow}\bigr) \\
	&= 0
\end{align*}

as required. 

We have just studied what happens when we combine two spin-1/2 particles, finding that we obtain spin-1 and spin-0 states. This case is part of a larger problem, where we combine spin-$s_1$ with spin-$s_2$. In this case, the possible spin states are:

\begin{equation}
	s = (s_1+s_2),\, (s_1+s_2-1),\, (s_1+s_2-2),\, \ldots,\, |s_1-s_2| \,.
\end{equation}

For example, combining a particle of spin 3/2 with another of spin 2, the two particles could take on spin 7/2, 5/2, 3/2 or 1/2.

\subsubsection{Arbitrary combinations}
The combined state $\ket{s, m}$ with total spin $s$ and $z$-component $m$ will be some linear combination of the composite states $\ket{s_1, m_1}\ket{s_2, m_2}$:

\begin{equation}
	\ket{s,m} = \sum_{m_1+m_2=m}C_{m_1m_2m}^{s_1s_2s}\ket{s_1s_2,m_1m_2} \,.
\end{equation}

Because the $z$-components add, the only composite states that contribute are those for which $m_1+m_2=m$. The constants $C_{m_1m_2m}^{s_1s_2s}$ are called the \textbf{Clebsch-Gordan coefficients}, which are often presented in table form.

%% Figure: Clebsch-Gordan coefficients

Consider the combination of a spin-2 particle with a spin-1 particle. The table says that

\begin{equation*}
	\ket{3,0} = \frac{1}{\sqrt{5}}\ket{2,1}\ket{1,-1} + \sqrt{\frac{3}{5}}\ket{2,0}\ket{1,0} + \frac{1}{\sqrt{5}}\ket{2,-1}\ket{1,1} \,.
\end{equation*}

If two particles of spin 2 and spin 1 are at rest in a box, and the \emph{total} spin is 3, and its $z$-component is zero, them a measurement of $S_z^{(1)}$ could return the value $\hbar$ (with probability 1/5) or zero (with probability 3/5) or $-\hbar$ (with probability 1/5).

These tables also work the other way around:

\begin{equation}
	\ket{s_1s_2,m_1m_2} = \sum_{s}C_{m_1m_2m}^{s_1s_2s}\ket{s,m_1+m_2} \,.
\end{equation}

Now consider a spin-3/2 particle along with a spin-1 particle. The table reads:

\begin{equation}
	\ket{\frac{3}{2}\frac{1}{2}, \frac{1}{2}0} = \sqrt{\frac{3}{5}}\ket{\frac{5}{2},\frac{1}{2}} + \frac{1}{\sqrt{15}}\ket{\frac{3}{2},\frac{1}{2}} - \frac{1}{\sqrt{3}}\ket{\frac{1}{2},\frac{1}{2}} \,.
\end{equation}

If particles of spin 3/2 and spin 1 are at rest in a boc, and it is known that the first has $m_1=1/2$ and the second has $m_2=0$ (so $m=1/2$), the total spin $s$ could be measured as 5/2 (with probability 3/5), 3/2 (with probability 1/15) or 1/2 (with probability 1/3).

\section{Identical particles}
\subsection{Two-particle systems}
A two particle system has a wavefunction that is a function of the coordinates of both particles, and time: $\Psi_{2\text{-particle}} = \Psi(\vb{r}_1,\vb{r}_2,t)$. The Schrödinger equation remains the same, $\mathcal{H}\Psi = i\hbar\pdv*{\Psi}{t}$, where $\mathcal{H}$ is the Hamiltonian for the whole system, 

\begin{equation}
	\mathcal{H} = -\frac{\hbar^2}{2m_1}\laplacian{}_{1} - \frac{\hbar^2}{2m_2}\laplacian{}_2 + V(\vb{r}_1, \vb{r}_2, t)\,,
\end{equation}

where $\grad{}_{j}$ only applies to the coordinates of particle $j$. The probability of finding particle 1 in volume $\dd[3]{r_1}$ \textbf{and} particle 2 in volume $\dd[3]{r_2}$ is

\begin{equation}
	|\Psi(\vb{r}_1, \vb{r}_2, t)|^2\dd[3]{r_1}\dd[3]{r_2} \,,
\end{equation}

which is the same statistical interpretation as the single-particle case. The wavefunction must also be normalized such that 

\begin{equation}
\iint|\Psi(\vb{r}_1, \vb{r}_2, t)|^2\dd[3]{r_1}\dd[3]{r_2} = 1 \,.
\end{equation}

For a time independent potential, we can obtain a complete set of solutions using separation of variables, yielding a spatial component of the wavefunction comprised of stationary states:

\begin{equation}
	\Psi(\vb{r}_1,\vb{r}_2,t) = \psi(\vb{r}_1, \vb{r}_2)\exp(-iEt/\hbar) \,,
\end{equation}

where $E$ is the total energy of the system and $\psi$ obeys the time independent Schrödinger equation.

When considering multi-particle systems, distinguishability is an important concept with important consequences. There are two ways to distinguish particles:

\begin{enumerate}
	\item Using differences in intrinsic physical properties (like mass, charge or spin).
	\item Track the trajectory of each particle.
\end{enumerate}

The problem with the second point is that quantum mechanics forbids us from having infinite prevision on the trajectory of a particle. Thus, according to quantum mechanics, identical particles are indistinguishable. This impacts the possible form of multi-particle wavefunctions.

\subsection{Bosons and fermions}
Suppose particle 1 is in the state $\psi_a(\vb{r}_1)$ and particle 2 is in the state $\psi(\vb{r}_2)$. In this case, the two-particle wavefunction is the product of the two single-particle wavefunctions:

\begin{equation}
\psi(\vb{r}_1,\vb{r}_2) = \psi(\vb{r}_1)\psi(\vb{r}_2) \,,
\end{equation}

assuming that the different particles can identified. However, for two identical particles, it is impossible to determine which particle is in state $\psi_a$ or $\psi_b$. All that can be known is that one particle is in one state and the second particle is in the other. 

To accommodate indistinguishable particles, we construct a wavefunction that is a superposition of the two possible states:

\begin{equation}
\psi_\pm(\vb{r}_1,\vb{r}_2) = A\Bigl[ \psi_a(\vb{r}_1)\psi_b(\vb{r}_2) \pm \psi_b(\vb{r}_1)\psi_a(\vb{r}_2) \Bigr] \,.
\end{equation}

The theory permits two kinds of identical particles:

\begin{itemize}
	\item  Bosons (+): have a symmetric wavefunction and are characterized by integer spin values.
	\item Fermions ($-$): have an anti-symmetric wavefunction and are characterized by half-integer spin values.
\end{itemize}

\begin{mdframed}
	\paragraph*{Note}:
	A consequence of this is that two identical fermions may not occupy the same state. Consider two identical fermions in states $\psi_a = \psi_b$. Then,
	
	\begin{equation*}
	\psi_-(\vb{r}_1,\vb{r}_2) =  A\Bigl[ \psi_a(\vb{r}_1)\psi_a(\vb{r}_2) - \psi_a(\vb{r}_1)\psi_a(\vb{r}_2) \Bigr] = 0 \,.
	\end{equation*}
	
	In this case the total particle wavefunction must be zero and therefore two identical fermions will never occupy the same state. This is the famous \textbf{Pauli exclusion principle}.	
\end{mdframed}

Let us define the exchange operator $\mathcal{P}$, which interchanges the two particles:

\begin{equation}
\mathcal{P}\ket{(1,2)} = \ket{(2,1)} \,.
\end{equation}

Clearly, applying this operator twice does not change the state ($\mathcal{P}^2 = \mathbb{I}$) and we also note that the eigenvalues of $\mathcal{P}$ are $\pm1$. If the two particles are identical, the Hamiltonian must treat them the same, since $m_1 = m_2$ and $V(\vb{r}_1,\vb{r}_2) = V(\vb{r}_2,\vb{r}_1)$. It then follows that $\mathcal{P}$ and $\mathcal{H}$ are compatible observables, 

\begin{equation}
\comm{\mathcal{P}}{\mathcal{H}} = 0 \,.
\end{equation}

Ehrenfest's theorem also ensures that $\dv*{\exv{\mathcal{P}}}{t} = 0$ and the expectation value of the exchange operator is conserved. As a result, we can find solutions to the Schrödinger equation that are either symmetric (eigenvalue $+1$) or anti-symmetric (eigenvalue $-1$) under exchange:

\begin{equation}
\boxed{ \psi(\vb{r}_1,\vb{r}_2) = \pm\psi(\vb{r}_2,\vb{r}_1) \,.}
\end{equation}

The wavefunction for identical particles is required to obey the systematization rule (+: bosons, $-$: fermions). This is true for $N$-particle systems with any two particles exchanged.

\begin{mdframed}[backgroundcolor=gray!20]
	\paragraph*{Example:}
	Suppose we have two non-interacting particles in a one-dimensional infinite square well. The one-particle states are given by:
	
	\begin{equation*}
	\psi_n(x) = \sqrt{\frac{2}{a}}\sin\qty(\frac{n\pi}{a}x) \,, \quad E_n = \frac{n^2\pi^2\hbar^2}{2ma^2} \,.
	\end{equation*}
	
	If the particles are distinguishable with the first in state $n_1$ and the second in state $n_2$, the composite wavefunction is:
	
	\begin{equation*}
	\psi_{n_{1}n_{2}}(x) = \psi_{n_{1}}(x_1)\psi_{n_{2}}(x_2) \,, \qquad E_{n_{1n_{2}}} = \frac{({n_1}^2 + {n_2}^2)\pi^2\hbar^2}{2ma^2} \,.
	\end{equation*}
	
	The ground state is 
	
	\begin{equation*}
	\psi_{1,1} = \frac{2}{a}\sin\qty(\frac{\pi}{a}x_1)\sin\qty(\frac{\pi}{a}x_2) \, \qquad E_{1,1} = \frac{\pi^2\hbar^2}{ma^2} \,.
	\end{equation*}
	
	The first excited state is doubly degenerate:
	
	\begin{equation*}
	\left\{
	\begin{array}{ll}
	\psi_{1,2} = \frac{2}{a}\sin\qty(\frac{2\pi}{a}x_2) & E_{1,2} = \frac{2\pi^2\hbar^2}{2ma^2} \\[6pt]
	\psi_{2,1} = \frac{2}{a}\sin\qty(\frac{2\pi}{a}x_1)\sin\qty(\frac{\pi}{a}x_2) & E_{2,1} = \frac{5\pi^2\hbar^2}{2ma^2}
	\end{array}
	\right. \,.
	\end{equation*}
	
	If the particles are identical bosons, the ground state is unchanged but the first excited state is non-degenerate:
	
	\begin{equation*}
	\frac{\sqrt{2}}{a}\qty[\sin\qty(\frac{\pi}{a}x_1)\sin\qty(\frac{2\pi}{a}x_2) + \sin\qty(\frac{2\pi}{a}x_1)\sin\qty(\frac{2\pi}{a}x_2)] \,,\qquad E = \frac{5\pi^2\hbar^2}{2ma^2} \,.
	\end{equation*}
	
	If the particles are identical fermions, they cannot both have the same quantum number $n$, so there is no state with energy $E = \pi^2\hbar^2/ma^2$. Instead, the ground state is
	
	\begin{equation*}
	\frac{\sqrt{2}}{a}\qty[\sin\qty(\frac{\pi}{a}x_1)\sin\qty(\frac{2\pi}{a}x_2) - \sin\qty(\frac{2\pi}{a}x_1)\sin\qty(\frac{2\pi}{a}x_2)] \,,\qquad E = \frac{5\pi^2\hbar^2}{2ma^2} \,.
	\end{equation*}
	
	%% Plots of wavefunctions
\end{mdframed}

\subsection{Exchange forces}
The symmetrization requirement of multi-particle wavefunctions has profound implications on the chemistry and physics of materials. Let us consider a one-dimensional example of two particles. Suppose one particle is in state $\psi_a(x)$ and the other is in $\psi_b(x)$. We also assume that these states are orthonormal.

If both particles are distinguishable, the combined wavefunction is

\begin{equation}
	\psi(x_1,x_2) = \psi_a(x_2)\psi_b(x_2) \,.
\end{equation}

If they are identical bosons, the (normalized) composite wavefunction is

\begin{equation}
	\psi_+(x_1,x_2) = \frac{1}{\sqrt{2}}\Bigl[\psi_a(x_1)\psi_b(x_2) + \psi_b(x_1)\psi_a(x_1)\Bigr] \,.
\end{equation}

If, on the other hand, they are identical fermions,

\begin{equation}
\psi_+(x_1,x_2) = \frac{1}{\sqrt{2}}\Bigl[\psi_a(x_1)\psi_b(x_2) - \psi_b(x_1)\psi_a(x_1)\Bigr] \,.
\end{equation}

We wish to compute the average squared distance between the particles,

\begin{equation}
	\exv{(x_1-x_2)^2} = \exv{{x_1}^2} + \exv{{x_2}^2} - 2\exv{x_1x_2} \,.
\end{equation}

\paragraph*{Case 1: Distinguishable particles.} Using the distinguishable particle wavefunction, we find that:

\begin{align*}
	\exv{{x_1}^2} &= \int{x_1}^2|\psi_a(x_1)|^2 \dd{x_1}\int|\psi_b(x_2)|^2\dd{x_2} = \exv{x^2}_a \\[4pt]
	\exv{{x_2}^2} &= \int|\psi_a(x_1)|^2\dd{x_1}\int{x_2}^2|\psi_b(x_2)|^2 \dd{x_2} = \exv{x^2}_b \\[4pt]
	\exv{x_1x_2} &= \int x_1|\psi_a(x_1)|^2 \dd{x_1} \int x_2|\psi_b(x_2)|^2\dd{x_2} = \exv{x}_a\exv{x}_b \,,
\end{align*}

so in this case we find that:

\begin{equation}
		\exv{(x_1-x_2)^2} = \exv{x^2}_a + \exv{x^2}_b - 2\exv{x}_a\exv{x}_b \,.
\end{equation}

This answer would be the same if particle 1 had been in state $\psi_b$ and particle 2 had been in state $\psi_b$.

\paragraph*{Case 2: Identical particles.} Using the bosonic and fermionic wavefunctions, we can repeat the same process. We find:

\begin{align*}
	\exv{{x_1}^2} &= \frac{1}{2}\Biggl[\Biggl. \int{x_1}^2|\psi_a(x_1)|^2\dd{x_1}\int|\psi_b(x_2)|^2\dd{x_2} \\ &\qquad\qquad + \int{x_1}^2|\psi_b(x_1)|^2\dd{x_1}\int|\psi_a(x_2)|^2\dd{x_2} \\
	&\qquad\qquad\pm \int{x_1}^2{\psi_a}^*(x_1)\psi_b(x_2)\dd{x_1}\int{\psi_b}^*(x_2)\psi_a(x_2) \dd{x_2} \\ 
	& \qquad\qquad\pm \int{x_1}^2{\psi_b}^*(x_1)\psi_a(x_2)\dd{x_1}\int{\psi_a}^*(x_2)\psi_b(x_2) \dd{x_2} \Biggl.\Biggr] \\[4pt]
	&= \frac{1}{2}\qty[\exv{x^2}_a + \exv{x^2}_b \pm 0 \pm 0] \\[4pt]
	&= \frac{1}{2}\qty(\exv{x^2}_a + \exv{x^2}_b) \,.
\end{align*}

In a similar fashion, it can be shown that $\exv{{x_2}^2} = \exv{{x_1}^2}$ since the particles are indistinguishable. Next,

\begin{align*}
	\exv{x_1x_2} &= \frac{1}{2}\Biggl[\Biggr. \int x_1|\psi_a(x_1)|^2\dd{x_1}\int x_2|\psi_b(x_2)|^2\dd{x_2} \\
	&\qquad\qquad + \int x_1|\psi_b(x_1)|^2\dd{x_1}\int x_2|\psi_a(x_2)|^2\dd{x_2} \\
	&\qquad\qquad \pm \int x_1{\psi_a}^*(x_1)\psi_b(x_1)\dd{x_1}\int x_2{\psi_b}^*(x_2)\psi_a(x_2)\dd{x_2} \\
	&\qquad\qquad \pm \int x_1{\psi_a}^*(x_1)\psi_b(x_1)\dd{x_1}\int x_2{\psi_b}^*(x_2)\psi_a(x_2)\dd{x_2} \Biggl.\Biggr] \\[4pt]
	&= \frac{1}{2}\Bigl[\exv{x}_a\exv{x}_b + \exv{x}_b\exv{x}_a \pm \exv{x}_{ab}\exv{x}_{ba} \pm \exv{x}_{ba}\exv{x}_{ab}\Bigr] \\[4pt]
	&= \exv{x}_a\exv{x}_b \pm |\exv{x}_{ab}|^2 \,,
\end{align*}

where

\begin{equation*}
	\exv{x}_{ab} \equiv \int x {\psi_a}^*(x)\psi_b(x)\dd{x}
\end{equation*}

is known as the exchange integral. Putting this together, we see that:

\begin{equation}
	\exv{(x_1-x_2)^2}_\pm = \exv{x^2}_a + \exv{x^2}_b - 2\exv{x}_a\exv{x}_b \mp 2|\exv{x}_{ab}| \,.
\end{equation}

Comparing the distinguishable and identical results, we see that the difference comes from the final term:

\begin{equation}
	\exv{\Delta x^2}_\pm = \exv{\Delta x^2}_{\text{distinguishable}} \mp 2|\exv{x}_{ab}|^2
\end{equation}

\begin{mdframed}
\paragraph*{Conclusion:}
Identical bosons tend to be closer together, and identical fermions tend to be farther apart, compared to distinguishable.

Note that the exchange integral $\exv{x}_{ab}$ vanishes unless the particles' wavefunctions overlap. Thus when the distance between particles is large, indistinguishability is not an issue.
\end{mdframed}

When there is overlap in wavefunctions, the system behaves as if there is a force of attraction (for bosons) or repulsion (for fermions). This is referred as exchange force (\textit{although this is not really a force}), and is a consequence of the symmetrization requirement. It is strictly a quantum mechanical effect with no classical analogue.

The complete state of a particle includes not only its position wavefunction, but also a spinor describing its spin orientation. It is the total state function, not just the position component, that must obey the symmetrization rule. For an electron, we require

\begin{equation}
	\psi(\vb{r}_1,\vb{r}_2)\chi(1,2) = -\psi(\vb{r}_2,\vb{r}_2)\chi(2,1) \,.
\end{equation}

Recalling the composite spin states for two particles, we see that the singlet state is anti-symmetric whereas the triplet states are symmetric. Two electrons occupying the singlet state must have a symmetric spatial wavefunction whereas two electrons occupying a triplet state must have an anti-symmetric spatial wavefunction.

The singlet state leads to a bonding state, and the the triplet states lead to anti-bonding states.

\subsection{Atoms}
A neutral atom of atomic number $Z$ consists of a heavy nucleus with electric charge $Ze$ surrounded by $Z$ electrons (of mass $m$ and charge $-e$). The Hamiltonian for this system is

\begin{equation}
	\mathcal{H} = \sum_{j=1}^{Z}\qty(-\frac{\hbar^2}{2m}\laplacian{}_j - \frac{1}{4\pi\epsilon_0}\frac{Ze^2}{r_j}) + \frac{1}{2}\frac{1}{4\pi\epsilon_0}\sum_{j\neq}^{Z}\frac{e^2}{|\vb{r}_j - \vb{r}_k|}
\end{equation}

which is composed of kinetic and potential energy terms, as usual, in addition to an additional potential energy term accounting for the electron-electron repulsion. The problem is to solve the Schrödinger equation, $\mathcal{H}\psi = E\psi$ for this situation. Because the electrons are identical, the acceptable states, denoted

\begin{equation}
	\Psi = \psi(\vb{r}_1, \vb{r}_2,\ldots,\vb{r}_Z)\chi(\vb{s}_1, \vb{s}_2,\ldots,\vb{s}_Z)
\end{equation}

must be anti-symmetric with respect to exchange of any two electrons. In particular, no two electrons can occupy the same state. Unfortunately, the general case has not been solved exactly except for the hydrogen atom. We must resort to approximate methods, which we will see in later sections.

After hydrogen, helium is next simplest atom, with $Z=2$. The Hamiltonian is

\begin{equation}
	\mathcal{H} = \qty(-\frac{\hbar^2}{2m}\laplacian{}_{1} - \frac{1}{4\pi\epsilon_0}\frac{2e^2}{r_1}) + \qty(-\frac{\hbar^2}{2m}\laplacian{}_{2} - \frac{1}{4\pi\epsilon_0}\frac{2e^2}{r_2}) + \frac{1}{4\pi\epsilon_0}\frac{e^2}{|\vb{r}_1-\vb{r}_2|}
\end{equation}

which consists of two hydrogenic Hamiltonians and a term for the  repulsion of the two electrons. This repulsion term makes it difficult to solve for the system state, so we will ignore it, and see what result we obtain. In this case, the Hamiltonian separates and we can write the solution as

\begin{equation}
	\psi(\vb{r}_1,\vb{2}_2) = \psi_{n\ell m}(\vb{r}_1)\psi_{n^\prime\ell^\prime m^\prime}(\vb{r}_2)
\end{equation} 

where the single-particle solutions are of the same form as hydrogen, except the Bohr radius is halved and the Bohr energies are increased by a factor of 4. The total energy of the system is

\begin{equation}
	E = 4\qty(-\frac{\SI{13.6}{\eV}}{n^2} - \frac{\SI{13.6}{\eV}}{{n^\prime}^2}) \,.
\end{equation}

The ground state would be 

\begin{equation}
	\psi_{0}(\vb{r}_1,\vb{r}_2) = \psi_{100}(\vb{r}_1)\psi_{100}(\vb{r}_2) = \frac{8\pi}{{a_0}^2}\exp(-\frac{2(r_1+r_2)}{a_0})
\end{equation}

with energy $E = \SI{-109}{\eV}$. Because this spatial wavefunction is symmetric, the spinor must be anti-symmetric and the two electrons must have their spins oppositely aligned.

The experimentally determined ground state of \ce{He} is a singlet state, but the energy is $\SI{-78.975}{\eV}$. There is poor agreement because we have ignored the electron-electron repulsion. This repulsion energy will be positive, and accounts for the higher energy. The excited states will be $\psi_{n\ell m}\psi_{100}$. Exciting both electrons will cause one to drop back to the ground state, releasing enough energy to ionize the atom.

We can construct either symmetric or anti-symmetric spatial wavefunctions which require either anti-symmetric (singlet) or symmetric (triplet) spin states. When a helium atom has a symmetric wavefunction and anti-symmetric spinor, it is known as \textbf{parahelium}. Alternatively, when helium has an anti-symmetric wavefunction and occupies a symmetric spin state, it is known as \textbf{orthohelium}. The ground state is necessarily parahelium, while excited states can take on either configuration.

Symmetric spatial states bring electrons closer together and raises the energy through electrostatic repulsion. This has been verified experimentally.

%% Figure: energy levels of helium

\subsection{Solids}
With solids, a few of the outermost valence electrons detach from their host atoms and are free to move through the material. The core electrons hare tightly bound to the nucleus and have properties like the original atom. The mobile valence electrons are influenced by a potential from the entire crystal lattice.

\subsubsection{Free electron gas}
The electron gas theory (Sommerfeld theory) ignores al forces and treats the mobile electrons as particles in a box. Consider a rectangular solid with dimensions $\ell_x$, $\ell_y$ and $\ell_z$. The potential acting on electrons is

\begin{equation}
	V(x,y,z) = 
	\begin{cases}
	0 & \text{if } x\in[0,\ell_x]\,,y\in[0,\ell_y]\,,z\in[0,\ell_z] \\[4pt]
	\infty & \text{otherwise}
	\end{cases} \,,
\end{equation}

and solving the Schrödinger and applying boundary conditions yields a solution depending on three quantum numbers:

\begin{equation}
	\psi(x,y,z) = \sqrt{\frac{8}{\ell_x\ell_y\ell_z}}\sin\qty(\frac{n_x \pi }{\ell_x}x)\sin\qty(\frac{n_y \pi }{\ell_y}y)\sin\qty(\frac{n_z \pi }{\ell_z}z) \,,
\end{equation} 

with energies given by

\begin{equation}
	E_{n_{x}n_{y}n_{z}} = \frac{\hbar^2\pi^2}{2m}\qty(\frac{{n_x}^2}{{\ell_x}^2} + \frac{{n_y}^2}{{\ell_y}^2} + \frac{{n_z}^2}{{\ell_z}^2}) = \frac{\hbar^2k^2}{2m}
\end{equation}

where $k$ is the magnitude of the wavevector $\vb{k} = k_x\vu{x} + k_y\vu{y} + k_z\vu{z}$. If we imagine a three dimensional $k$-space with axes $k_x$, $k_y$ and $k_z$, the allowable $k$-values setup a uniform discrete grid. 

%% Figure: k-space

The intersection points on the grid are distinct one-particle states. The volume in $k$-space occupied by each state is 

\begin{equation}
	V_{k} = \frac{\pi^3}{\ell_x\ell_y\ell_z} = \frac{\pi^3}{V} \,,
\end{equation}

where $V$ is the volume of the entire solid. Suppose this solid has $N$ atoms, each contributing $d$ free electrons. In a typical solid, $N\sim 10^{22}$ and $d\sim 1$ or $2$. Since electrons are fermions, subject to the Pauli exclusion principle, so only two of them can occupy any given $k$-state, one spin up and one spin down.

Filling states with increasing energy will give one octant a sphere in $k$-space centered around zero. Typically the sample is very large such that $V_k$ is very small and the wavevector $\vb{k}$ is approximated as continuous.

The radius of this spherical octant, known as the \textbf{Fermi wavevector} $k_\text{F}$ is determined by the fact each electron pair fills a volume $\pi^3/V$. This gives

\begin{equation*}
	\frac{1}{8}\qty(\frac{4}{3}\pi {k_\text{F}}^3) = Nd\qty(\frac{\pi^3}{2V})
\end{equation*}

or, after rearranging we obtain

\begin{equation}
	k_\text{F} = (3\rho\pi^2)^{1/3} \,,
\end{equation}

where $\rho = Nq/V$ is the free electron density. The edge of the filled sphere is the \textbf{Fermi surface} and the corresponding energy is the \textbf{Fermi energy}:

\begin{equation}
	E_\text{F} = \frac{\hbar^2}{2m}(3\rho\pi^2)^{2/3} \,.
\end{equation}

The total energy of the electron gas is then calculated by adding the energies of all the electrons in $k$-space:

\begin{align*}
	U &= \int\frac{\hbar^2k^2}{2m}\frac{\dd[3]{k}}{V_k/2} \\[4pt]
	&= \int_{0}^{k_\text{F}}\int_{0}^{\pi/2}\int_{0}^{\pi/2}\frac{\hbar^2 k^2}{2m}\qty(\frac{2V}{\pi^3})k^2\sin\theta\dd{k}\dd{\theta}\dd{\phi} \\[4pt]
	&= \frac{\hbar^2V}{2m\pi^2}\int_{0}^{k_\text{F}}k^4\dd{k} \\[4pt]
	&= V\frac{\hbar^2{k_\text{F}}^5}{10m\pi^2}
\end{align*}

We can write the total energy per unit volume as 

\begin{equation}
	U = \frac{\hbar^2(3\pi^2\rho)^{5/3}}{10\pi^2 m} = \frac{(2m)^{3/2}{E_\text{F}}^{5/2}}{5\pi^2\hbar^3} \,,
\end{equation}

which yields the average energy per electron:

\begin{equation}
	\exv{E} = \frac{3}{5}E_\text{F} \,.
\end{equation}

If we compare this to the average thermal energy of a classical particle, $\exv{E} = 3k_\text{B}T/2$, we can define the \textbf{Fermi temperature} to be 

\begin{equation}
	T_\text{F} = \frac{2}{5}\frac{E_\text{F}}{k_\text{B}} \,,
\end{equation}

which is on the order of $10^4~\si{\kelvin}$ for most solids. Pauli exclusion forces electrons to fill higher energy states than they would classically. This a purely quantum effect known as \textbf{degeneracy pressure} which keeps the Fermi sphere from collapsing.

\subsubsection{Band structure}
In the free electron model, all interactions are neglected, when in reality they feel a force from the positively charged, regularly spaced nuclei. Here will improve on the free electron model by considering a \textit{periodic} potential. The qualitative features of solids can be understood by considering the periodic nature of crystals.

\begin{mdframed}
\paragraph*{Bloch's theorem.}
If a system is described by a periodic potential energy function $V(x) = V(x+a)$ for some constant $a$, and all $x$, then the solutions to Schrödinger's equation are of the form

\begin{equation}
	\psi(x+a) = e^{iqa}\psi(x) 
\end{equation}

or equivalently

\begin{equation}
	\psi(x) = e^{iqx}u(x)
\end{equation}

where $u(x) = u(x+a)$ is some periodic function with the same periodicity as the potential (and of the lattice). In other words, the wavefunction is a plane wave modulated by some periodic envelope function.

\paragraph*{Proof.}
Let us define the translation operator $\hat{T}(a)$ as the operator with the property that 

\begin{equation}
	\hat{T}(a)\psi(x) = \psi(x-a) \,.
\end{equation} 

We can derive an explicit expression for this operator by Taylor expanding the right-hand side of this equation:

\begin{align*}
	\hat{T}(a)\psi(x) &= \sum_{n=0}^{\infty}\frac{1}{n!}(-a)^n\dv[n]{x}\psi(x) \\[4pt]
	&= \sum_{n=0}^{\infty}\frac{1}{n!}\qty(-\frac{ia}{\hbar}\hat{p})^n\psi(x)
\end{align*}

where $\hat{p}$ is the momentum operator. This gives

\begin{equation}
	\hat{T}(a) = \exp\qty(-\frac{ia}{\hbar}\hat{p}) \,.
\end{equation}

If the Hamiltonian commutes with the translation operator (\textit{which it does if the potential is periodic}), then they must share the same eigenstates and the wavefunctions must satisfy

\begin{equation*}
	\mathcal{H}\psi(x) = E\psi(x) \quad\text{and}\quad \hat{T}(a)\psi(x) = \lambda\psi(x) \,.
\end{equation*}

It is easily shown that the translation operator is unitary ($\hat{T}^\dagger\hat{T}=\mathbb{I}$), its eigenvalues must have unit magnitude. This means we must take $\lambda = e^{i\phi}$ for some real constant $\phi$. For reasons of convention we take $\phi = -qa$. Then, by the above argument, any wavefunction for the periodic system must satisfy

\begin{equation*}
	\psi(x-a)=e^{-iqa}\psi(x) \,.
\end{equation*} 

We can write this differently by taking $\psi(x) = e^{iqx}u(x)$. Writing this equation in terms of $u(x)$, we obtain 

\begin{equation*}
	e^{iq(x-a)}u(x-a) = e^{iqa}e^{-iqa}e^{iqx}u(x) \,,
\end{equation*}

which simplifies to 

\begin{equation*}
	u(x-a) = u(x) \,.
\end{equation*}

Thus we have proven both forms of Bloch's theorem.

\hspace*{\fill}$\blacksquare$
\end{mdframed}

Bloch's theorem says that $\psi(x)$ is not periodic, but $|\psi(x)|^2$ is:

\begin{equation}
	|\psi(x-a)|^2 = |\psi(x)|^2 \,.
\end{equation}

No true solid has infinite size, and the surface terminates the periodicity. However, any macroscopic crystal is comprised of a large number of atoms (around $10^{28}$ per cubic metre), such that the electrons are in an effectively infinite crystal. We can apply periodic boundary conditions such that:

\begin{equation}
	\psi(x+Na) = \psi(x) \,.
\end{equation}

Bloch's theorem ensures that

\begin{equation}
	e^{iNqa}\psi(x) = \psi(x) \,,
\end{equation}

so $e^{iNqa}=1$ which means that $Nqa = 2\pi n$ and

\begin{equation}
	q = \frac{2\pi}{a} n \,, \quad n = 0,\,\pm 1,\,\pm 2\,,\ldots
\end{equation}

Due to Bloch's theorem, we only need to solve the Schrödinger equation in one unit cell (for $0<x<a$), and we can construct a solution for all $x$.

Suppose an electron in a solid experiences a potential given by:

%% Figure: Delta function potential

\begin{equation}
	V(x) = \alpha\sum_{j=0}^{N-1}\delta(x-ja) \,.
\end{equation}

This is a Dirac comb, like found in the Kronig-Penney model. We need to solve the Schrödinger equation for $0<x<a$. We obtain the oscillatory free-particle wavefunction 

\begin{equation}
	\psi(x) = A\sin(kx) + B\cos(kx) \,,
\end{equation}

where $k=\sqrt{2mE}/\hbar$ and $E$ is the electron energy. Using Bloch's theorem, the wavefunction for $-a<x<0$ is

\begin{equation}
	\psi(x) = e^{-iqa}\Bigl(A\sin\bigl(k(x+a)\bigr) + B\cos\bigl(k(x+a)\bigr)\Bigr) \,.
\end{equation}

At $x =0$, the wavefunction must be continuous, which gives

\begin{equation*}
	B = e^{-iqa}\Bigl[A\sin(ka) + B\cos(ka)\Bigr] \,.
\end{equation*}

The derivative, on the other hand, is \textit{not} continuous at $x=0$ due to the delta function potential. It can be shown that the difference in derivatives is

\begin{equation*}
	\Delta\qty(\dv{\psi}{x}) = \frac{2m\alpha}{\hbar^2}\psi(0) \,.
\end{equation*}

This condition gives

\begin{equation*}
	kA - e^{-iqa}k\biggl(A\cos(ka) - B\sin(ka)\Bigr) = \frac{2m\alpha}{\hbar^2k}B \,,
\end{equation*}

and solving for $A$ in terms of $B$, we get

\begin{equation*}
	A\sin(ka) = \biggl(e^{iqa}-\cos(ka)\biggr)B \,.
\end{equation*}

Substituting this into the earlier result (and canceling $kB$), we get

\begin{equation*}
	\biggl(e^{iqa} - \cos(ka)\biggr)\bigg(1-e^{-iqa}\cos(ka)\bigg) + e^{-iqa}\sin[2](ka) = \frac{2m\alpha}{\hbar^2k}\sin(ka) \,,
\end{equation*}

which may be simplified to

\begin{equation}
	\cos(qa) = \cos(ka) + \frac{m\alpha}{\hbar^2k}\sin(ka) \,.
\end{equation}

This equation determines the allowed wavevectors and energies that the electron can take. If we let $z=ka$ and $\beta=m\alpha a/\hbar^2$, we obtain

\begin{equation*}
	f(z) = \cos z + \beta\frac{\sin z}{z} \,.
\end{equation*}

Only $k$ values when $|f(z)|<1$ are acceptable, since there are no solutions to $\cos(qa) = f(z)$ otherwise.

%% Figure: Plot of f(z) and bands

%% Figure: Plot of energy bands

There are \textit{gaps} in $k$ representing forbidden energies, which are separated by \textit{bands} of allowed energies. Within a band, nearly every energy is allowed, since $qa = 2\pi n/N$ where $N$ is very large. 

Due to Pauli exclusion, $2N$ electrons can occupy a band; $N$ are spin up and $N$ are spin down. If there is one free electron per atom ($d=1$), the first band will be half-filled. If there are two free electrons per atom ($d=2$), the first band will be totally filled. In three dimensions, with more realistic potentials, the band structure is more complicated. However, energy bands exist as a result of the periodic potential.

If the topmost band is partly filled, it takes little energy to excite an electron to the next allowed level and the material will be a conductor, or metal. On the other hand, if the highest band is completely filled, an electron is required to overcome the energy difference in the band gap to enter an excited state. This material is an insulator, although if the band gap is small and the electrons have sufficient thermal energy, a select few may become excited and the material is a semiconductor.

In the Sommerfeld free electron model, all solids should be metals, as no band gaps exist. However, Bloch's theorem ensures that the presence of a periodic potential gives rise to band theory, which can account for the range of electrical conductivities of different solids.

\section{Time independent perturbation theory}
\subsection{Nondegernerate perturbation theory}
\subsubsection{General formulation}
Let us assume that er have solves the time-independent Schrödinger equation for some potential

\begin{equation}
	\mathcal{H}^0{\psi_n}^0 = {E_n}^0{\psi_n}^0
\end{equation}

which gives energy eigenvalues ${E_n}^0$ and an associated set of orthonormal eigenfunctions

\begin{equation}
	\bra{{\psi_m}^0}\ket{{\psi_n}^0} = \delta_{mn} \,.
\end{equation}

Now suppose we add a small perturbation to the potential. We want to determine the new eigenvalues and eigenfunctions,

\begin{equation}
\mathcal{H}\psi_n = E_n\psi_n \,.
\end{equation}

In the vast majority of cases, we will not be able to analytically solve the Schrödinger equation for the new potential.

\textbf{Perturbation theory} is a systematic procedure for obtaining \emph{approximate} solutions to the perturbed problem, by building on the known exact solutions to the unperturbed case. 

To begin, we write the new Hamiltonian as a sum of two terms,

\begin{equation}
	\mathcal{H} = \mathcal{H}^0 + \lambda\mathcal{H}^{\prime}
\end{equation}

where $\mathcal{H}^0$ is the unperturbed Hamiltonian and $\mathcal{H}^{\prime}$ is the perturbation. We expand $\psi_n$ and $E_n$ as a power series in $\lambda$,

\begin{align}
	\psi_n &= {\psi_n}^0 + \lambda{\psi_n}^{1} + \lambda^2{\psi_n}^2 + \ldots \\[4pt]
	E_n &= {E_n}^0 + \lambda{E_n}^{1} + \lambda^2{E_n}^2 + \ldots
\end{align}

where ${\psi_n}^{i}$ is the $i$\textsuperscript{th}-order correction to the $n$\textsuperscript{th} eigenfunction and ${E_n}^{i}$ is the $i$\textsuperscript{th}-order correction to its eigenvalue. Inserting this into the Schrödinger equation gives

\begin{align*}
	&\qty(\mathcal{H}^0+\lambda\mathcal{H}^{\prime})\qty[{\psi_n}^0 + \lambda{\psi_n}^{1} + \lambda^2{\psi_n}^{2} + \ldots] \\ &\qquad = \qty({E_n}^{0} + \lambda{E_n}^{1} + \lambda^2{E_n}^{2} + \ldots)\qty[{\psi_n}^0 + \lambda{\psi_n}^{1} + \lambda^2{\psi_n}^{2} + \ldots]
\end{align*}

and collecting powers of $\lambda$ we get

\begin{align*}
	&\mathcal{H}^0{\psi_n}^0 + \lambda\qty(\mathcal{H}^0{\psi_n}^1 + \mathcal{H}^{\prime}{\psi_n}^{0}) + \lambda^2\qty(\mathcal{H}^0{\psi_n}^2 + \mathcal{H}^{\prime}{\psi_n}^{1}) \\ &\qquad = {E_n}^0{\psi_n}^0 + \lambda\qty({E_n}^0{\psi_n}^1 + {E_n}^1{\psi_m}^0) + \lambda^2\qty({E_n}^0{\psi_n}^2 + {E_n}^1{\psi_n}^1 + {E_n}^2{\psi_n}^0) \,.
\end{align*}

To lowest order, we get nothing new, $\mathcal{H}^0{\psi_n}^0 = {E_n}^0{\psi_n}^0$. To first and second order, we get

\begin{align}
	\mathcal{H}^0{\psi_n}^1 + \mathcal{H}^{\prime}{\psi_n}^0 &= {E_n}^0{\psi_n}^1 + {E_n}^1{\psi_n}^0 \\[4pt]	\mathcal{H}^0{\psi_n}^2 + \mathcal{H}^{\prime}{\psi_n}^1 &= {E_n}^0{\psi_n}^2 + {E_n}^1{\psi_n}^1 + {E_n}^2{\psi_n}^0 \,.
\end{align}

We and we can continue to any order if necessary.

\subsubsection{First order}
The first order perturbation equation gives

\begin{equation*}
	\mathcal{H}^0\ket{{\psi_n}^1} + \mathcal{H}^{\prime}\ket{{\psi_n}^0} = {E_n}^0\ket{{\psi_n}^1} + {E_n}^1\ket{{\psi_n}^0} \,,
\end{equation*}

and multiplying by $\bra{{\psi_n}^0}$, we get

\begin{equation}
	\bra{{\psi_n}^0}\ket{\mathcal{H}^0{\psi_n}^1} + \bra{\mathcal{H}^0{\psi_n}^0}\ket{{\psi_n}^1} = {E_n}^0\bra{{\psi_n}^0}\ket{{\psi_n}^1} + {E_n}^{\prime}\bra{{\psi_n}^0}\ket{{\psi_n}^0}
\end{equation}

and since the Hamiltonian is Hermitian,

\begin{equation*}
	\bra{{\psi_n}^0}\ket{\mathcal{H}^0{\psi_n}^1} = \bra{\mathcal{H}^0{\psi_n}^0}\ket{{\psi_n}^1} = {E_n}^0\bra{{\psi_n}^0}\ket{{\psi_n}^1}
\end{equation*}

which gives

\begin{equation}
	\boxed{ {E_n}^{\prime} = \bra{{\psi_n}^0}\mathcal{H}^0\ket{{\psi_n}^0}\,. }
\end{equation}

This is the fundamental result of first-order perturbation theory, and one of the most important equations in quantum mechanics. The first order correction is the expectation value of the perturbation in the unperturbed state.

\begin{mdframed}[backgroundcolor=gray!20]
\paragraph*{Example:}
Consider an infinite square well of width $a$, where the eigenfunctions are 

\begin{equation*}
	{\psi_n}^0(x) = \sqrt{\frac{2}{a}}\sin\qty(\frac{n\pi}{a}x) \,,\quad {E_n} = \frac{\hbar^2\pi^2n^2}{2ma^2} \,.
\end{equation*}

Suppose we increase the potential at the bottom of the well from zero to $V_0$. What is the first order correction to the energies?

\paragraph*{Solution}:
In this case, $\mathcal{H}^{\prime} = V_0$, so the correction in the energy is

\begin{equation*}
	{E_n}^{\prime} = \bra{{\psi_n}^0}V_0\ket{{\psi_n}^0} = V_0\bra{{\psi_n}^0}\ket{{\psi_n}^0} = V_0 \,.
\end{equation*}

The corrected energy levels are, as expected,  $E_n \approx {E_n}^0 + V_0$. In this case, the first order perturbation is exact because all other terms vanish for constant $V_0$.

If instead the perturbation extends halfway through the well, $0<x<a/2$, we find

\begin{equation*}
	{E_n}^{\prime} = \frac{2V_0}{a}\int_{0}^{a/2}\sin^2\qty(\frac{n\pi}{a}x)\dd{x} = \frac{V_0}{2} \,.
\end{equation*}

In this case, this is not the exact shift in energy, but it is a reasonable approximation.
\end{mdframed}

Having found the first order correction to the energy, we now seek the first order correction to the wavefunction. We write

\begin{equation}
	\qty(\mathcal{H}^0 - {E_n}^0){\psi_n}^1 = -\qty(\mathcal{H}^{\prime} - {E_n}^1){\psi_n}^{0} \,.
\end{equation}

The unperturbed wavefunctions form a complete basis and we can rewrite ${\psi_n}^{1}$ in terms of the ${\psi_n}^0$:

\begin{equation}
{\psi_n}^{1} = \sum_{m \neq n}C_{m}^{(n)}{\psi_m}^0 \,.
\end{equation}

Inserting this into the last equation, we get

\begin{equation*}
	\sum_{m \neq n}\qty({E_m}^0 - {E_n}^0)C_{m}^{(n)}{\psi_m}^0 = -\qty(\mathcal{H}^\prime - {E_n}^1){\psi_n}^0 \,.
\end{equation*}

Taking the inner product with some state ${\psi_\ell}^0$, we obtain

\begin{equation*}
	\sum_{m \neq n}\qty({E_m}^0 - {E_n}^0)C_{m}^{(n)}\bra{{\psi_\ell}^0}\ket{{\psi_m}^0} = -\bra{{\psi_\ell}^0}\mathcal{H}^{\prime}\ket{{\psi_n}^0} + {E_n}^1\bra{{\psi_\ell}^0}\ket{{\psi_n}^0} \,.
\end{equation*}

If $\ell = n$ then the left hand size is zero and we recover the first order energy correction. If $\ell \neq n$, then

\begin{equation*}
	({E_\ell}^0 - {E_n}^0)C_{\ell}^{(n)} = -\bra{{\psi_\ell}^0}\mathcal{H}^\prime\ket{{\psi_n}^0} \,.
\end{equation*}

This means that the first order correction to the wavefunction is

\begin{equation}
	\boxed{{\psi_ n}^1 = \sum_{m \neq n}\frac{\bra{{\psi_m}^0}\mathcal{H}^\prime\ket{{\psi_n}^0}}{{E_n}^0 - {E_m}^0}}
\end{equation}

The denominator is nonzero so long as the unperturbed energy spectrum is non-degenerate. If two states have the same energy, we need to use degenerate.

Perturbation theory often yields surprisingly accurate energies, with ${E_n}^0 + {E_n}^\prime$ being close to $E_n$. However, the wavefunction corrections are notoriously poor.

\subsubsection{Second order}
Similar to before, we apply the inner product $\bra{{\psi_n}^0}$ to the second order equation,

\begin{equation*}
	\mathcal{H}^0{\psi_n}^2 + \mathcal{H}^{\prime}{\psi_n}^1 = {E_n}^0{\psi_n}^2 + {E_n}^1{\psi_n}^1 + {E_n}^2{\psi_n}^0 \,.
\end{equation*}

this yields

\begin{equation}
	\bra{{\psi_n}^0}\ket{\mathcal{H}^0{\psi_n}^2} + \bra{{\psi_n}^0}\ket{\mathcal{H}^\prime{\psi_n}^1} = {E_n}^0\bra{{\psi_n}^0}\ket{{\psi_n}^2} + {E_n}^1\bra{{\psi_n}^0}\ket{{\psi_n}^1} + {E_n}^2\bra{{\psi_n}^0}\ket{{\psi_n}^0}
\end{equation}

which, since the Hamiltonian is Hermitian, gives

\begin{equation*}
	{E_n}^2 = \bra{{\psi_n}^0}\mathcal{H}^\prime\ket{{\psi_n}^1} - {E_n}^1\bra{{\psi_n}^0}\ket{{\psi_n}^1} \,.
\end{equation*}

The second term drops out, since $\bra{{\psi_n}^0}\ket{{\psi_n}^1} = \sum\limits_{m \neq n} C_{m}^{(n)}\bra{{\psi_n}^0}\ket{{\psi_m}^0} = 0$. This means

\begin{equation*}
	{E_n}^2 = \bra{{\psi_n}^0}\mathcal{H}^\prime\ket{{\psi_n}^1} = \sum_{m \neq n}C_{m}^{(n)}\bra{{\psi_n}^0}\mathcal{H}^\prime\ket{{\psi_m}^0} \,,
\end{equation*}

or,

\begin{equation}
	\boxed{ {E_n}^2 = \sum_{m \neq n}\frac{\qty|\bra{{\psi_n}^0}\mathcal{H}^\prime\ket{{\psi_m}^0}|^2}{{E_n}^0 - {E_m}^0}}
\end{equation}

which is the fundamental result of second order perturbation theory. It is possible to determine the second order correction to the wavefunction, as well as additional energy corrections. However, a second order correction to the energy is all that is required.

\subsection{Degenerate perturbation theory}
If the unperturbed states are degenerate, two or more states ${\psi_n}^0$ share the same energy and the correction terms in perturbation theory blow up. We need to develop a proper theory to handle this case.

\subsubsection{Two-fold degeneracy}
Suppose that $\mathcal{H}^0{\psi_a}^0 = E^0{\psi_a}^0$ and $\mathcal{H}^0{\psi_b}^0 = E^0{\psi_b}^0$, where $\bra{{\psi_a}^0}\ket{{\psi_b}^0} = 0$. Any linear combination of these states is still an eigenstate of $\mathcal{H}^0$ with the same energy. Typically, adding a perturbation $\mathcal{H}^\prime$ will break the degeneracy and the unperturbed energy $E_0$ will split into two.

%% Figure: Lifting of degeneracy.

Turning off the perturbation, the upper and lower energies become orthogonal linear combinations of ${\psi_a}^0$ and ${\psi_b}^0$. However, we do not know what these \textit{good} states are, defined as the limit as the perturbation is turned off ($\lambda \rightarrow 0$). We write these in general form,

\begin{equation}
	\psi^0 = \alpha{\psi_a}^0 + \beta{\psi_b}^0 \,,
\end{equation}

where $\alpha$ and $\beta$ are adjustable. We wish to solve the Schrödinger equation, with $\mathcal{H} = \mathcal{H}^0 + \lambda\mathcal{H}^\prime$, and 

\begin{align}
\psi_n &= {\psi_n}^0 + \lambda{\psi_n}^{1} + \lambda^2{\psi_n}^2 + \ldots \\[4pt]
E_n &= {E_n}^0 + \lambda{E_n}^{1} + \lambda^2{E_n}^2 + \ldots
\end{align}

Inserting this into the Schrödinger equation and collecting powers of $\lambda$, we obtain 

\begin{equation*}
	\mathcal{H}^0\psi^0 = \lambda\qty(\mathcal{H}^\prime\psi^0 + \mathcal{H}^0\psi^1) + \ldots = E^0\psi^0 + \lambda\qty(E^1\psi^0 + E^0\psi^1) \,.
\end{equation*}

To first order in $\lambda$, we get

\begin{equation}
	\mathcal{H}^0\psi^1 + \mathcal{H}^1\psi^0 = E^0\psi^1 + E^1\psi^0 \,.
\end{equation}

Taking the inner product of this with ${\psi_a}^0$ gives

\begin{equation*}
	\bra{{\psi_a}^0}\ket{\mathcal{H}\psi^1} + \bra{{\psi_a}^0}\ket{\mathcal{H}^\prime\psi^0} = E^0\bra{{\psi_a}^1}\ket{\psi^1} + E^1\bra{{\psi_a}^0}\ket{\psi^0} \,,
\end{equation*}

and because the Hamiltonian is Hermitian, the first terms on the left and right cancel. Since $\psi^0$ is a linear combination of ${\psi_a}^0$ and ${\psi_b}^0$ which are orthogonal, we obtain

\begin{equation}
	\alpha\bra{{\psi_a}^0}\mathcal{H}\ket{{\psi_a}^0} + \beta\bra{{\psi_a}^0}\mathcal{H}\ket{{\psi_b}^0} \,, 
\end{equation}

and performing a similar analysis, taking the inner product with ${\psi_b}^0$ gives

\begin{equation}
\alpha\bra{{\psi_b}^0}\mathcal{H}\ket{{\psi_a}^0} + \beta\bra{{\psi_b}^0}\mathcal{H}\ket{{\psi_b}^0} \,. 
\end{equation}

We can write this in matrix form, 

\begin{equation}
	\begin{bmatrix}
	W_{aa} & W_{ab} \\ W_{ba} & W_{bb}
	\end{bmatrix}
	\begin{bmatrix}
	\alpha \\ \beta
	\end{bmatrix}
	= E^1
		\begin{bmatrix}
	\alpha \\ \beta
	\end{bmatrix} \,,
\end{equation}

where $W_{ij} = \bra{{\psi_i}^0}\mathcal{H}\ket{{\psi_j}^0}$ are known and depend of the perturbation and the good states. By diagonalizing the matrix we can determine the two energies. We find that:

\begin{equation}
	\boxed{{E_\pm}^1 = \frac{1}{2}\qty[W_{aa} + W_{bb} \pm \sqrt{\qty(W_{aa} - W_{bb})^2 - 4|W_{ab}|^2 \,\,}\,] }
\end{equation}

which is the fundamental result of degenerate perturbation theory. We can determine $\alpha$ and $\beta$ by noting that if the off-diagonal matrix elements are zero, then 

\begin{equation*}
	{E_+}^1 = W_{aa} = \bra{{\psi_a}^0}\mathcal{H}^\prime\ket{{\psi_a}^0} \quad\text{and}\quad {E_-}^1 = W_{bb} = \bra{{\psi_b}^0}\mathcal{H}^\prime\ket{{\psi_b}^0} \,.,
\end{equation*}

and we recover nondegenerate perturbation theory. Therefore, in the basis of degenerate states, the $W$-matrix is always diagonal and we can use nondegenerate perturbation theory. 

\begin{mdframed}
\paragraph*{Theorem.} Let $\mathcal{A}$ be a Hermitian operator that commutes with the unperturbed Hamiltonian $\mathcal{H}^0$ and the perturbation $\mathcal{H}^\prime$. If ${\psi_a}^0$ and ${\psi_b}^0$ are the degenerate eigenfunctions of $\mathcal{H}^0$ and are also eigenfunctions of $\mathcal{A}$ with distinct eigenvalues

\begin{equation*}
	\mathcal{A}{\psi_a}^0 = \mu{\psi_a}^0 \,, \quad \mathcal{A}{\psi_b}^0 = \nu{\psi_b}^0 \,\,\text{and}\,\, \mu\neq\nu
\end{equation*}

then ${\psi_a}^0$ and ${\psi_b}^0$ are \textit{good} states and off-diagonal matrix elements in the degenerate subspace are zero.

\paragraph*{Proof.} By assumption, $\comm{\mathcal{A}}{\mathcal{H}^\prime} = 0$ which means that

\begin{align*}
	\bra{{\psi_a}^0}\comm{\mathcal{A}}{\mathcal{H}^\prime}\ket{{\psi_b}^0} &= 0 \\[4pt]
	\bra{{\psi_a}^0}\ket{\mathcal{A}\mathcal{H}^\prime{\psi_b}^0} - \bra{{\psi_a}^0}\ket{\mathcal{H}^\prime\mathcal{A}{\psi_b}^0} &= 0 \\[4pt]
		\mu\bra{\mathcal{A}{\psi_a}^0}\ket{\mathcal{H}^\prime{\psi_b}^0} - \nu\bra{{\psi_a}^0}\ket{\mathcal{H}^\prime{\psi_b}^0} &= 0 \\[4pt]
		(\mu-\nu)\bra{{\psi_a}^0}\mathcal{H}^\prime\ket{{\psi_b}^0} &= 0 \\[4pt]
		(\mu-\nu)W_{ab} &= 0
\end{align*}

but $\mu\neq\nu$ so $W_{ab} = 0$. \\
\hspace*{\fill}$\blacksquare$
\end{mdframed}

If faced with degenerate states, find a Hermitian operator that commutes with $\mathcal{H}^0$ and $\mathcal{H}^\prime$. Then pick the unperturbed states that are simultaneously eigenfunctions of $\mathcal{H}^0$ and $\mathcal{A}$ to use nondegenerate perturbation theory.

\subsubsection{Higher-order degeneracy}
In the previous section we saw that we need to solve for the eigenvalues of 

\begin{equation*}
\begin{bmatrix}
W_{aa} & W_{ab} \\ W_{ba} & W_{bb}
\end{bmatrix}
\begin{bmatrix}
\alpha \\ \beta
\end{bmatrix}
= E^1
\begin{bmatrix}
\alpha \\ \beta
\end{bmatrix} \,,
\end{equation*}

where $E^1$ are the eigenvalues of $W$ and the eigenvectors represent \textit{good} linear combinations of the unperturbed states. For an $n$-fold degenerate case we need to solve the $n \times n$ eigenvalue problem of the form

\begin{equation}
	W\vb{c} = E^1\vb{c}
\end{equation}

where $W_{ij} = \bra{{\psi_i}^0}\mathcal{H}^\prime\ket{{\psi_j}^0}$ and the vector $\vb{c}$ contains all the Fourier coefficients in the expansion

\begin{equation*}
	\psi^0 = \sum_{k=1}^{n}c_i\psi_k \,.
\end{equation*}

These \textit{good} unperturbed states represent a basis in the degenerate subspace that diagonalizes the matrix $W$. 

Like before, if there exists an operator $\mathcal{A}$ that commutes with $\mathcal{H}^0$ and $\mathcal{H}^\prime$, and use the simultaneous eigenfunctions of $\mathcal{A}$ and $\mathcal{H}^0$, then the matrix $W$ will be diagonal.

\begin{mdframed}[backgroundcolor=gray!20]
\paragraph*{Example:}
Consider a three dimensional infinite well, where the potential ${V(x,y,z) = 0}$ if $0<x,y,z<a$ and is infinite otherwise. The wavefunctions are 

\begin{equation*}
	\psi_{n_{x}n_{y}n_{z}}(x,y,z) = \qty(\frac{2}{a})^{3/2}\sin\qty(\frac{n_x\pi}{a}x)\sin\qty(\frac{n_y\pi}{a}y)\sin\qty(\frac{n_z\pi}{a}z)
\end{equation*}

with energies

\begin{equation*}
	E_{n_{x}n_{y}n_{z}} = \frac{\hbar^2\pi^2}{2ma^2}\Bigl({n_x}^2  {n_y}^2 + {n_z}^2\Bigr) \,.
\end{equation*}

The ground state is non-degenerate, with $\psi_{111}$ having energy $E_0 = 3\hbar^2\pi^2/2ma^2$. However, the first excited state has a triplet degeneracy:

\begin{equation*}
	\psi_a = \psi_{112} \quad \psi_b = \psi_{121} \quad \psi_c = \psi_{211} \,,
\end{equation*}

with energy $E_{1} = 2E_0$. If we add a perturbation $\mathcal{H}^\prime = V_0$ for $0<x,y<a/2$ and zero otherwise, we raise the potential by $V_0$ in one quarter of the box. Use perturbation theory to determine the correction to the energies.

\paragraph*{Solution:}
The ground state is nondegenerate so the correction to the energy is given by

\begin{align*}
	{E_{0}}^1 &= \bra{\psi_{111}}\mathcal{H}^\prime\ket{\psi_{111}} \\[4pt]
	&= \qty(\frac{2}{a})^3V_0\int_{0}^{a/2}\sin^2\qty(\frac{\pi}{a}x)\dd{x}\int_{0}^{a/2}\sin^2\qty(\frac{\pi}{a}y)\dd{y}\int_{0}^{a/2}\sin^2\qty(\frac{\pi}{a}z)\dd{z} \\[4pt]
	&= \qty(\frac{2}{a})^3V_0\qty(\frac{a}{4})\qty(\frac{a}{4})\qty(\frac{a}{4}) \\[4pt]
	&= \frac{V_0}{4} \,.
\end{align*}

To determine the first order correction to the first excited states we need to use degenerate perturbation theory. We construct a matrix $W$ with matrix elements $W_{ij} = \bra{{\psi_i}^0}\mathcal{H}^\prime\ket{{\psi_j}^0}$ where $i$ and $j$ index over $a$, $b$ and $c$.

The integrals in the diagonal matrix elements take the same form as the correction to the ground state energy and it follows that $W_{aa} = W_{bb} = W_{cc} = V_0/4$. For the off-diagonal terms we obtain integrals of the form:

\begin{align*}
	&\int_{0}^{a}\sin\qty(\frac{n_1\pi}{a}x)\sin\qty(\frac{n_2\pi}{a}x)\dd{x} = \frac{a}{2}\delta_{n_{1}n_{2}} \\[4pt]
	&\int_{0}^{a/2}\sin\qty(\frac{n_1\pi}{a}x)\sin\qty(\frac{n_2\pi}{a}x)\dd{x} = \frac{a}{4}\delta_{n_{1}n_{n}} + \frac{2a}{3\pi}\biggl(1-\delta_{n_{1}n_{2}}\biggr) \,.
\end{align*}

Using this, we find that $W_{ab}$, $W_{ac}$, $W_{ba}$ and $W_{ca}$ are all zero. This leaves
\begin{align*}
	W_{bc} = {W_{cb}}^* &= \qty(\frac{2}{a})^3V_0\qty(\int_{0}^{a/2}\sin\qty(\frac{\pi}{a}u)\sin\qty(\frac{2\pi}{a}u)\dd{u} )^3 \\[4pt]
	&= \qty(\frac{2}{a})^3V_0\qty(\frac{2a}{3\pi})\qty(\frac{2a}{3\pi})\qty(\frac{a}{2}) \\[4pt]
	&= \qty(\frac{4}{3\pi})^2V_0 \,.
\end{align*}

This gives

\begin{equation*}
	W = \frac{V_0}{4}
	\begin{bmatrix}
	1 & 0 & 0 \\
	0 & 1 & K \\
	0 & K & 1 
	\end{bmatrix}
\end{equation*}
where $K = (8/3\pi)^2$. The eigenvalues of this matrix are $V_0/4$, $V_0(1+K)/4$ and $V_0(1-K)/4$. To first order, the perturbed energies are then

\begin{equation*}
	E_1 = \left\{
	\begin{array}{l}
	{E_1}^0 + V_0/4 \\
	{E_1}^0 + V_0(1+K)/4 \\
	{E_1}^0 + V_0(1-K)/4
	\end{array}
	\right. \,.
\end{equation*}

The perturbation lifts the degeneracy and splits ${E_1}^0$ into three distinct levels. If we had used nondegenerate perturbation theory, we would have found a correction of $V_0/4$ for all three states.

The \textit{good} linear combinations of states 

\begin{equation*}
	\psi^0 = \alpha\psi_a + \beta\psi_b + \gamma\psi_c
\end{equation*}

are the eigenvectors of 

\begin{equation*}
	\begin{bmatrix}
	1 & 0 & 0 \\
	0 & 1 & K \\
	0 & K & 1 
	\end{bmatrix}
	\begin{bmatrix}
	\alpha \\ \beta \\ \gamma
	\end{bmatrix}
	= w
		\begin{bmatrix}
	\alpha \\ \beta \\ \gamma
	\end{bmatrix}
\end{equation*}

where $w$ are the eigenvalues of the form given above. For $w=1$, $\alpha = V$ and $\beta=\gamma=0$. For $w = 1+K$, $\alpha = 0$ and $\beta = \gamma = 1/\sqrt{2}$. Finally, for $w = 1-K$, $\alpha = 0$ and $\beta = -\gamma = 1/\sqrt{2}$. This means that the \textit{good} states are

\begin{equation*}
	\psi^0 = \left\{
	\begin{array}{l}
	\psi_a \\[4pt]
	\frac{1}{\sqrt{2}}\bigl(\psi_b + \psi_c\bigr) \\[4pt]
	\frac{1}{\sqrt{2}}\bigl(\psi_b - \psi_c\bigr)
	\end{array}
	\right. \,.
\end{equation*}
\end{mdframed}

\subsection{Fine structure of hydrogen}
In studying the hydrogen atom, we solved the Hamiltonian

\begin{equation}
	\mathcal{H} = -\frac{\hbar^2}{2m}\laplacian{} - \frac{e^2}{4\pi\epsilon_0}\frac{1}{r} \,,
\end{equation}
 
however the \textbf{fine structure} brings corrections to the Bohr energies,

\begin{equation}
	E_n = -\qty[\frac{m}{2\hbar^2}\qty(\frac{e^2}{4\pi\epsilon_0})^2]\frac{1}{n^2} \,,
\end{equation}

through two effects: a \textbf{relativistic correction} and \textbf{spin-orbit coupling}. 

Compared to the Bohr energies fine structure is a small perturbation, smaller than a factor of $\alpha^2$, where

\begin{equation}
	\alpha = \frac{e^2}{4\pi\epsilon_0\hbar c} \approx \frac{1}{137}
\end{equation}

is the \textbf{fine structure constant}. Smaller corrections include the Lamb shift (associated with the quantization of the electric field), and the hyperfine structure (due to the magnetic interaction between the electron and proton).

\subsubsection{Relativistic correction}
The kinetic energy term in the hydrogen atom was written $K = p^2/2m$, with $\vb{p}\rightarrow -i\hbar\grad{}$, as is the standard prescription in quantum mechanics. However, the relativistic kinetic energy is given by

\begin{equation}
	K = \frac{mc^2}{\sqrt{1-\frac{v^2}{c^2}}} - mc^2 \,,
\end{equation}

and the momentum is

\begin{equation}
	p = \frac{mv}{\sqrt{1-\frac{v^2}{c^2}}} \,.
\end{equation}

By eliminating $v$ we find that

\begin{equation}
	K = \sqrt{p^2c^2 + m^2c^4} - mc^2 \,.
\end{equation}

In the limit that $p \ll mc$, this reduces to the classical result. Expanding in a Taylor series in powers of $p/mc$, we find 

\begin{equation}
	K = \frac{p^2}{2m} - \frac{p^4}{8m^3c^2} + \ldots
\end{equation}

The lowest order relativistic correction to the Hamiltonian is therefore

\begin{equation}
	\mathcal{H}_{\text{r}}^\prime = -\frac{p^4}{8m^3c^2} \,.
\end{equation}

The first order correction to the energy is therefore

\begin{align*}
	{E_{\text{r}}}^{1} &= \bra{{\psi_{n\ell m}}^0}{\mathcal{H}_{r}}^\prime\ket{{\psi_{n\ell m}}^0} \\[4pt]
	&= -\frac{1}{8m^3c^2}\bra{{\psi_{n\ell m}}^0}p^4\ket{{\psi_{n\ell m}}^0} \\[4pt]
	&= -\frac{1}{8m^3c^2}\bra{p^2{\psi_{n\ell m}}^0}\ket{p^2{\psi_{n\ell m}}^0} \,.
\end{align*}

Schrödinger's equation gives $p^2\psi = 2m(E-V)\psi$, such that

\begin{align*}
	{E_{\text{r}}}^{1} &= -\frac{\bra{{\psi_{n\ell m}}^0}(E-V)^2\ket{{\psi_{n\ell m}}^0}}{2mc^2} \\[4pt]
	&= -\frac{E^2 - 2E\exv{V} + \exv{V}^2}{2mc^2} \,.
\end{align*}

For hydrogen, there is a Coulomb potential, which gives

\begin{equation}
	{E_{\text{r}}}^{1} = -\frac{1}{2mc^2}\qty[{E_n}^2 + 2E_n\qty(\frac{e^2}{4\pi\epsilon_0})\exv{\frac{1}{r}} + \qty(\frac{e^2}{4\pi\epsilon_0})^2\exv{\frac{1}{r^2}}]
\end{equation}

where $E_n$ is the Bohr energy. It can be shown that for the unperturbed states of hydrogen,

\begin{align}
	\exv{\frac{1}{r}} &= \frac{1}{n^2a_0} \\[4pt]
	\exv{\frac{1}{r^2}} &= \frac{1}{(\ell+1/2)n^3{a_0}^2}
\end{align}

where $a_0$ is the Bohr radius. Inserting these, we get

\begin{equation*}
{E_{\text{r}}}^{1} = -\frac{1}{2mc^2}\qty[{E_n}^2 + 2E_n\qty(\frac{e^2}{4\pi\epsilon_0})\frac{1}{n^2a_0} + \qty(\frac{e^2}{4\pi\epsilon_0})^2\frac{1}{(\ell+1/2)n^3{a_0}^2}] \,.
\end{equation*}

Noting that the Bohr radius is $a_0 = 4\pi\epsilon_0\hbar^2/me^2$, we obtain

\begin{equation}
	\boxed{{E_{\text{r}}}^{1} = -\frac{(E_n)^2}{2mc^2}\qty[\frac{4n}{\ell+1/2}-3]}
\end{equation}

which is a factor of $E_n/mc^2\approx10^{-5}$ smaller than the Bohr energies.

\begin{mdframed}
	\paragraph*{Note:} 
	We used nondegenerate perturbation theory even though the hydrogenic states are degenerate in energy for a given $n$. However, we note that the perturbation ${\mathcal{H}_\text{r}}^\prime$ is spherically symmetric and therefore it commutes with the $L^2$ and $L_z$ operators. Furthermore, the eigenfunctions of $L^2$ and $L_z$ together yield distinct eigenvalues for all $n^2$ states with energy $E_n$. As a result, $\psi_{n\ell m}$ are \textit{good states} and nondegenerate perturbation theory can be used.
\end{mdframed}

\subsubsection{Spin-orbit coupling}
In the hydrogen atom the electron orbits the proton. However, from the electron's reference frame, it is the proton that circulates around it. The orbiting positive charge creates a magnetic field $\vb{B}$ that exerts a torque on the electron's magnetic dipole. The Hamiltonian is $\mathcal{H} = -\vb{\mu}\vdot\vb{B}$ and the magnetic field can be determined from the Biot-Savart law, assuming the proton orbit is a continuous current loop:

\begin{equation}
	B = \frac{\mu_0 I}{2r}
\end{equation}

where the effective current is $I = e/T$ and $T$ is the period of the orbit. The orbital angular momentum is then

\begin{equation}
	L = mvr = \frac{2\pi m r^2}{T} \,.
\end{equation}

using this (and that $c^2 = 1/\mu_0\epsilon_0$) we find that 

\begin{equation}
	\vb{B} = \frac{1}{4\pi\epsilon_0}\frac{e}{mc^2r^2}\,\vb{L} \,.
\end{equation}

From electrodynamics, the classical magnetic moment of a spinning charged sphere is 

\begin{equation}
	\mu = \frac{qr^2\omega}{5} \,,
\end{equation}

and its angular momentum $S = I\omega$ is

\begin{equation}
	S = \frac{2mr^2\omega}{5} \,.
\end{equation}

Combining these, we get 

\begin{equation}
	\vb*{\mu} = \frac{q}{2m}\vb{S} \,,
\end{equation}

where here the gyromagnetic ratio is $\gamma = -e/2m$ if this is an electron. Using relativistic quantum mechanics, Dirac showed that the moment is actually twice that of the classical value, giving $\gamma=-e/m$.

However, the electron's reference frame that we have considered is not inertial. The correction, \textbf{Thomas precession}, is a factor of $1/2$ which cancels Dirac's correction. This means that the spin-orbit Hamiltonian is

\begin{equation}
	\mathcal{H}_{\text{SO}}^\prime = \qty(\frac{e^2}{8\pi\epsilon_0})\frac{1}{m^2c^3r^3}\,\vb{L}\vdot\vb{S} \,.
\end{equation}

Physically, it is due to the torque applied on the electron's magnetic moment by the magnetic field generated by the proton.

With the spin-orbit interaction, the Hamiltonian no longer commutes with the orbital or angular momentum operators. As a result of Ehrenfest's theorem, spin and orbital angular momenta are not individually conserved. The spin-orbit Hamiltonian does commute with $L^2$ and $S^2$, as well as the total momentum,

\begin{equation}
	\vb{J} = \vb{L} + \vb{S} \,.
\end{equation}

Hence, these quantities are conserved. The eigenstates of $L_z$ and $S_z$ (spherical harmonics, and spin up/down) are not \textit{good} states to use in perturbation theory. However, the eigenstates of $L^2$, $S^2$ and $J^2$ are. We find that

\begin{equation}
	J^2 = (\vb{L}+\vb{S})\vdot(\vb{L}+\vb{S}) = L^2 + S^2 + 2\vb{L}\vdot\vb{S} \,,
\end{equation}

which allows us to write

\begin{equation}
	\vb{L}\vdot\vb{S} = \frac{1}{2}\Big(J^2 - L^2 - S^2\Bigr) \,.
\end{equation}

The eigenvalues of $\vb{L}\vdot\vb{S}$ are therefore

\begin{equation}
	(\vb{L}\vdot\vb{S})\ket{\Psi} = \frac{\hbar^2}{2}\Bigl[j(j+1) - \ell(\ell+1) - s(s+1)\Bigr]\ket{\Psi} \,,
\end{equation}

where $s=1/2$ since we are dealing with an electron. The first order correction to the Bohr energies is therefore

\begin{align*}
	{E_\text{SO}}^{1} &= \bra{\psi_{n\ell m}}\mathcal{H}_{\text{SO}}^\prime\ket{\psi_{n\ell m}} \\[4pt]
	&= \frac{e^2}{8m^2c^3\pi\epsilon_0}\bra{\psi_{n\ell m}}\frac{1}{r^3}\,\vb{L}\vdot\vb{S}\ket{\psi_{n\ell m}} \\[4pt]
	&= \frac{\hbar^2e^2}{16\pi\epsilon_0m^2c^3}\exv{\frac{1}{r^3}} \,,
\end{align*}

where it can be shown that

\begin{equation}
	\exv{\frac{1}{r^3}} = \frac{1}{\ell(\ell+1/2)(\ell+1)n^3{a_0}^3} \,.
\end{equation}

This gives

\begin{equation*}
	{E_\text{SO}}^{1} = \frac{\hbar^2e^2}{16\pi\epsilon_0m^2c^3}\qty[\frac{j(j+1)-\ell(\ell+1)-(s(s+1))}{\ell(\ell+1/2)(\ell+1)n^3{a_0}^3}] \,,
\end{equation*}

or

\begin{equation}
	\frac{(E_n)^2}{mc^2}\frac{n\Bigl[j(j+1) - \ell(\ell+1) - s(s+1)\Bigr]}{\ell(\ell+1/2)(\ell+1)} 
\end{equation}

in terms of the Bohr energies.

\begin{mdframed}
\paragraph*{Note:} Both the relativistic and spin-orbit corrections are of order $(E_n)^2/mc^2$. Adding both gives the fine structure formula:

\begin{equation}
	{E_\text{fs}}^{1} = \frac{(E_n)^2}{2mc^2}\qty[3 - \frac{4n}{\ell+1/2} + \frac{2n\Bigl(j(j+1) - \ell(\ell+1) - s(s+1)\Bigr)}{\ell(\ell+1/2)(\ell+1)}] \,.
\end{equation}

Cleaning this up, we can we can write the new hydrogenic energy levels as

\begin{equation}
	\boxed{E_{nj} = -\frac{\SI{1}{Ry}}{n^2}\qty[1 + \frac{\alpha^2}{n^2}\qty(\frac{n}{j+1/2} - \frac{3}{4})]}
\end{equation}

where $\SI{1}{Ry} = \SI{13.6}{\eV}$ is the ground state of the unperturbed hydrogen atom. 
\end{mdframed}

The fine structure corrections break the energy degeneracy in $\ell$. For a given $n$, the allowed $\ell$ states have different energies according to their total angular momentum.

%% Figure: energy levels

\subsection{The Zeeman effect}
When an external uniform magnetic field $\vb{B}_\text{ext}$ is applied to an atom, the energy levels are shifted, due to the Zeeman effect. This is similar to spin-orbit interaction and the Hamiltonian is

\begin{equation}
	\mathcal{H}_{\text{Z}}^\prime = -(\vb*{\mu}_\ell + \vb*{\mu}_s)\vdot\vb{B}_\text{ext} \,,
\end{equation}

where 

\begin{equation}
	\vb*{\mu}_s = -\frac{e}{m}\,\vb{S} \,,
\end{equation}

is the spin dipole moment (which uses the relativistic quantum mechanical gyromagnetic ratio), and

\begin{equation}
	\vb*{\mu}_\ell = -\frac{e}{m}\,\vb{L} \,.
\end{equation}

is the orbital dipole moment, (which uses the classical gyromagnetic ratio). We find that

\begin{equation}
	\mathcal{H}_{\text{Z}}^\prime = \frac{e}{2m}\Bigl(\vb{L} + 2\vb{S}\Bigr)\vdot\vb{B}_\text{ext} \,.
\end{equation}

The size of the Zeeman splitting depends on the magnitude of the external field, compared to the magnitude of the internal field created by spin-orbit coupling:

\begin{itemize}
	\item If $\vb{B}_\text{ext} \gg \vb{B}_\text{int}$ then the fine structure dominates and the Zeeman Hamiltonian can be treated as a perturbation.
	\item If $\vb{B}_\text{ext} \ll \vb{B}_\text{int}$ then the Zeeman effect dominates and the fine structure Hamiltonian is the perturbation.
	\item If $\vb{B}_\text{ext} \sim \vb{B}_\text{int}$ then both need to be treated with degenerate perturbation theory.
\end{itemize}

We will consider each case for the hydrogen atom.

\subsubsection{Weak-field Zeeman effect}
If $\vb{B}_\text{ext} \ll \vb{B}_\text{int}$ then fine structure dominates and \textit{good} quantum numbers are $n$, $\ell$, $j$ and $m_j$ (but not $m_\ell$ or $m_s$ since $\vb{L}$ and $\vb{S}$ are not individually conserved).

Perturbation theory says that the Zeeman energy correction is

\begin{equation}
	{E_\text{Z}}^1 = \bra{n,\ell,j,m_j} \mathcal{H}_\text{Z}^\prime \ket{{n,\ell,j,m_j}} = \frac{e}{2m}\,\vb{B}_\text{ext}\vdot\exv{\vb{L} + 2\vb{S}} \,.
\end{equation}

We eliminate $\vb{L}$ by writing $\vb{L} + 2\vb{S} = \vb{J} + \vb{S}$. We can easily calculate $\exv{\vb{J}}$, however to calculate $\exv{S}$ we consider that $\vb{L}$ and $\vb{S}$ precess rapidly around a fixed value of $\vb{J}$. 

%% Figure: Precession

This means that the time-averaged spin is simply its projection onto the total angular momentum:

\begin{equation}
	\vb{S}_\text{average} = \frac{\vb{S}\vdot\vb{J}}{J^2} \,\vb{J} \,.
\end{equation}

We write $\vb{L} = \vb{J}-\vb{S}$ in order to isolate $L^2 = J^2 +S^2 - 2\vb{S}\vdot\vb{J}$. This gives

\begin{equation}
	\vb{S}\vdot\vb{J} = \frac{1}{2}\bigl(J^2+S^2-L^2\bigr) = \frac{\hbar^2}{2}\Bigl[j(j+1) + s(s+1) - \ell(\ell+1)\Bigr] \,,
\end{equation}

from which we find 

\begin{equation}
	\exv{\vb{L} + 2\vb{S}} = \exv{\qty(1+\frac{\vb{S}\vdot\vb{J}}{J^2})\,\vb{J}} = \qty[1+\frac{j(j+1) - \ell(\ell+1) + s(s+1)}{2j(j+1)}]\exv{\vb{J}} \,.
\end{equation}

\begin{mdframed}
\paragraph*{Note:} We define the Landé g-factor, $g_J$ as 

\begin{equation}
	g_J \equiv \qty[1+\frac{j(j+1) - \ell(\ell+1) + s(s+1)}{2j(j+1)}] \,,
\end{equation}

which is a useful shorthand.
\end{mdframed}

If we choose the external field to be along the $z$-axis, then

\begin{align}
	{E_\text{Z}}^1 &= \frac{e}{2m}\,B_\text{ext} g_J \exv{J_z} \nonumber \\[4pt]
	&= \frac{e\hbar}{2m} B_\text{ext}g_Jm_j \nonumber \\[4pt]
	&= \mu_\text{B} g_J B_\text{ext}m_j
\end{align}

where $\mu_\text{B} = e\hbar/2m \approx \SI{5.8e-5}{\eV\per\tesla}$ is the \textbf{Bohr magneton}.

The total energy is the sum of the fine structure correction and the Zeeman contribution. For example, the ground state ($n=1$, $\ell=0$, $j=1/2$, $g_j=2$) splits in two:

\begin{equation*}
	{E_0}^1 = -\SI{1}{Ry}\qty(1+\frac{\alpha^2}{4}) \pm \mu_\text{B}B_\text{ext} \approx \,.
\end{equation*}

How small does $B_\text{ext}$ need to be in order to be treated as a perturbation? If we recall that

\begin{equation*}
	\vb{B} = \frac{1}{4\pi\epsilon_0}\frac{e}{mc^2r^2}\,\vb{L} \,,
\end{equation*}

and we assume that $L = \hbar$ and $r = a_0$, we find that $B_\text{int} = \SI{12}{\tesla}$, whereas the Earth's magnetic field is $\sim\SI{100}{\micro\tesla}$. Therefore, a Zeeman field is \emph{strong} if it is much larger than $\SI{12}{\tesla}$ and is \emph{weak} if it is much less than $\SI{12}{\tesla}$.

\begin{mdframed}[backgroundcolor=gray!20]
\paragraph*{Example:}
Find the energies of all eight $n=2$ states under weak-field Zeeman splitting.
\paragraph*{Solution:}
For $n=2$ we have $\ell = 0$ ($j=1/2$) and $\ell = 1$ ($j=1/2$ or $3/2$) states. These states are $\ket{n, \ell, j, m_j} $:

\begin{align*}
	\ket{1} &= \ket{2,\,0,\,1/2,\,1/2} \\
	\ket{2} &= \ket{2,\,0,\,1/2,\,-1/2} \\
	\ket{3} &= \ket{2,\,1,\,1/2,\,1/2} \\
	\ket{4} &= \ket{2,\,1,\,1/2,\,-1/2} \\
	\ket{5} &= \ket{2,\,1,\,3/2,\,3/2} \\
	\ket{6} &= \ket{2,\,1,\,3/2,\,1/2} \\
	\ket{7} &= \ket{2,\,1,\,3/2,\,-1/2} \\
	\ket{8} &= \ket{2,\,1,\,3/2,\,-3/2}
\end{align*}

where the Landé g-factor for each state is: 

\begin{equation*}
	g_{1} = g_{2} = 2 \qquad g_{3} = g_{4} = 2/3 \qquad g_{5} = g_{6} = g_{7} = g_{8} = 4/3 \,.
\end{equation*}

There are two sets of fine structure energies:

\begin{align*}
	E_{1} = E_{2} = E_{3} = E_{4} &= -\SI{3.4}{\eV}\qty(1+\frac{5}{16}\,\alpha^2) \,; \\[4pt]
		E_{5} = E_{6} = E_{7} = E_{8} &= -\SI{3.4}{\eV}\qty(1+\frac{1}{16}\,\alpha^2) \,,
\end{align*}

such that the degeneracy is broken and the energies are

\begin{align*}
	E_1 &= -\SI{3.4}{\eV}(1 + 5\alpha^2/16) + \mu_\text{B} B_\text{ext} \\[4pt]
	E_2 &= -\SI{3.4}{\eV}(1 + 5\alpha^2/16) - \mu_\text{B} B_\text{ext} \\[4pt]
	E_3 &= -\SI{3.4}{\eV}(1 + 5\alpha^2/16) + \frac{1}{3}\mu_\text{B} B_\text{ext} \\[4pt]
	E_4 &= -\SI{3.4}{\eV}(1 + 5\alpha^2/16) - \frac{1}{3}\mu_\text{B} B_\text{ext} \\[4pt]
	E_5 &= -\SI{3.4}{\eV}(1 + \alpha^2/16) + 2\mu_\text{B} B_\text{ext} \\[4pt]
	E_6 &= -\SI{3.4}{\eV}(1 + \alpha^2/16) + \frac{2}{3}\mu_\text{B} B_\text{ext} \\[4pt]
	E_7 &= -\SI{3.4}{\eV}(1 + \alpha^2/16) - \frac{2}{3}\mu_\text{B} B_\text{ext} \\[4pt]
	E_8 &= -\SI{3.4}{\eV}(1 + \alpha^2/16) - 2\mu_\text{B} B_\text{ext} \\[4pt]
\end{align*}

%% Figure: energies splitting
\end{mdframed}

\subsubsection{Strong-field Zeeman effect}
When $B_\text{ext} \gg B_\text{int}$, the Zeeman effect is dominant over fine structure. This means that we can apply perturbation theory in sequence $\mathcal{H}^0 \rightarrow \mathcal{H}_\text{Z}^\prime \rightarrow \mathcal{H}_\text{fs}^\prime$. 

If we consider only the external field $B_\text{ext}$ along $\vu{z}$, the \textit{good} quantum numbers are $n$, $\ell$, $m_\ell$ and $m_s$. Recall that a spin in a magnetic field is subject to Larmor precession, with $\exv{S_z}$ being independent of time whereas $\exv{S_x}$ and $\exv{S_y}$ are time dependent; a similar effect holds for orbital angular momentum. We then choose $m_\ell$ and $m_s$ as \textit{good} quantum numbers since their quantities are conserved. Since $\exv{\vb{L}}$ and $\exv{\vb{S}}$ precess at different frequencies, the total momentum $\exv{\vb{J}}$ is not conserved.

Both $L_z$ and $S_z$ commute with the Hamiltonian. The Zeeman Hamiltonian is

\begin{equation*}
	\mathcal{H}_\text{Z}^\prime = \frac{e}{2m}\,B_\text{ext}(L_z + 2S_z) \,.
\end{equation*}

Using the unperturbed hydrogen wavefunctions $\psi_{n \ell m}$, we find

\begin{align*}
	\mathcal{H}_\text{Z}^\prime \psi_{n \ell m} &= \frac{e}{2m}\,B_\text{ext}(L_z\psi_{n \ell m} + 2S_z\psi_{n \ell m}) \\[4pt]
	&= \frac{e}{2m}\,B_\text{ext}(\hbar m_\ell\psi_{n \ell m} + 2\hbar m_s\psi_{n \ell m}) \\[4pt]
	&= \mu_\text{B} B_\text{ext} (m_\ell + 2m_s) \psi_{n \ell m} \,,
\end{align*}

and we see that they are an eigenstate of this perturbation Hamiltonian. The energies with an external field but no fine structure are therefore

\begin{equation}
	E_{n, m_{\ell}, m_{s}} = -\frac{\SI{13.6}{\eV}}{n^2} + \mu_\text{B} B_\text{ext}(m_\ell + 2m_s) \,.
\end{equation}

These are the \textit{unperturbed} energies, to which we apply a fine structure perturbation. The first order correction to the energy is

\begin{equation}
	E_\text{fs}^\prime = \bra{\psi_{n \ell m_{\ell} m_{s}}}\mathcal{H}_\text{r}^\prime + \mathcal{H}_\text{SO}^\prime \ket{\psi_{n \ell m_{\ell} m_{s}}} \,.
\end{equation}

The relativistic correction ends up being the same as before,

\begin{equation*}
	E_\text{r}^\prime = -\frac{(E_n)^2}{2mc^2}\qty[\frac{4n}{\ell+1/2}-3] \,,
\end{equation*}

With spin-orbit coupling, we get a term 

\begin{equation*}
	E_\text{SO}^\prime = \frac{e^2}{8\pi\epsilon_0 m^2c^2} \exv{\vb{S}\vdot\vb{L}}\exv{\frac{1}{r^3}} \,,
\end{equation*}

which depends on both the angular and radial parts of the wavefunction. As before, 

\begin{equation*}
	\exv{\frac{1}{r^3}} = \frac{1}{\ell(\ell+1/2)(\ell+1) n^3 {a_0}^3}
\end{equation*}

and 

\begin{align}
	\exv{\vb{S}\vdot\vb{L}} &= \exv{S_xL_x} + \exv{S_yL_y} + \exv{S_zL_z} \nonumber \\[4pt]
	&= \exv{S_x}\exv{L_x} + \exv{S_y}\exv{L_y} + \exv{S_z}\exv{L_z} \,.
\end{align}

The $x$ and $y$ components of both angular momenta oscillate rapidly and therefore the time average is zero. Their expectation values will be zero. This gives

\begin{equation}
	\exv{\vb{S}\vdot\vb{L}} = \exv{S_z}\exv{L_z} = \hbar^2m_\ell m_s
\end{equation}

We then find that

\begin{equation}
	{E_\text{fs}}^\prime = -\frac{\SI{13.6}{\eV}}{n^2}\qty[1 - \frac{2\alpha^2}{4n^2} + \frac{\alpha^2}{n^2}\qty(\frac{\ell(\ell+1) - m_\ell m_s}{\ell(\ell+1/2)(\ell+1)})] + \mu_\text{B}B_\text{ext}(m_\ell + 2m_s) \,.
\end{equation}

\subsubsection*{Intermediate-field Zeeman effect}
When $B_\text{ext} \sim B_\text{int}$ then we have to treat both the fine structure and Zeeman Hamiltonians as perturbations to the Bohr Hamiltonian,

\begin{equation}
	\mathcal{H}^\prime = \mathcal{H}_\text{fs}^\prime + \mathcal{H}_\text{Z}^\prime \,.
\end{equation}

We will focus on the $n=2$ states as an example and we will use $\ell$, $j$ and $m_\ell$ as a \textit{good} basis for degenerate perturbation theory. 

\begin{mdframed}
\paragraph*{Note:} one can use $\ell$, $m_\ell$ and $m_s$ as a basis, which would make the matrix elements of $\mathcal{H}_\text{Z}^\prime$ simpler, but those of $\mathcal{H}_\text{fs}^\prime$ more complicated.
\end{mdframed}

We use the Clebsch-Gordan tables to express $\ket{j, m_j}$ in terms of the composite states $\ket{\ell,m_\ell}\ket{s,m_s}$:

\begin{align*}
	&\ell = 0 \quad
	\left\{ 
	\begin{array}{l}
	\psi_1 \equiv \ket{1/2,\, 1/2} = \ket{0,\, 0}\ket{1/2,\, 1/2} \\[6pt]
	\psi_2 \equiv \ket{1/2,\, -1/2} = \ket{0,\, 0}\ket{1/2,\, -1/2}
	\end{array}
	\right. \\[6pt]
	&\ell = 1 \quad
	\left\{ 
	\begin{array}{l}
	\psi_3 \equiv \ket{3/2,\, 3/2} = \ket{1,\, 1}\ket{1/2,\, 1/2} \\[6pt]
	\psi_4 \equiv \ket{3/2,\, -3/2} = \ket{1,\, -1}\ket{1/2,\, -1/2} \\[6pt]
	\psi_5 \equiv \ket{3/2,\, 1/2} = \sqrt{\frac{2}{3}}\ket{1,\, 0}\ket{1/2,\, 1/2} + \frac{1}{\sqrt{3}}\ket{1,\, 1}\ket{1/2,\, -1/2} \\[6pt]
	\psi_6 \equiv \ket{1/2,\, 1/2} = -\frac{1}{\sqrt{3}}\ket{1,\, 0}\ket{1/2,\, 1/2} + \sqrt{\frac{2}{3}}\ket{1,\, 1}\ket{1/2,\, -1/2} \\[6pt]
	\psi_7 \equiv \ket{3/2,\, -1/2} = \frac{1}{\sqrt{3}}\ket{1,\, -1}\ket{1/2,\, 1/2} + \sqrt{\frac{2}{3}}\ket{1,\, 0}\ket{1/2,\, -1/2} \\[6pt]
	\psi_8 \equiv \ket{1/2,\, -1/2} = -\sqrt{\frac{2}{3}}\ket{1,\, -1} + \frac{1}{\sqrt{3}} \ket{1,\, 0}\ket{1/2,\, -1/2}
	\end{array}
	\right. \,.
\end{align*}

Next, we calculate all matrix elements, $\bra{\psi_i}\mathcal{H}_\text{fs}^\prime + \mathcal{H}_\text{Z}^\prime\ket{\psi_k}$.

For the fine structure perturbation, the diagonal matrix elements are

\begin{equation*}
	{E_\text{fs}}^\prime = \frac{(E_n)^2}{2mc^2}\qty(3-\frac{4n}{j+1/2}) \,.
\end{equation*}

Noting that the $\{\psi_i\}$ either have $j=1/2$ or $3/2$ and that $E_n = \alpha^2mc^2/2n^2$, we find that 

\begin{align}
	\bra{\psi_i}\mathcal{H}_\text{fs}^\prime\ket{\psi_i} &= -5\gamma \, \quad i\in\{1, 2, 6, 8\} \\[4pt]
		\bra{\psi_i}\mathcal{H}_\text{fs}^\prime\ket{\psi_i} &= -\gamma\,  \quad i\in\{3, 4, 5, 7\}
\end{align}

where $\gamma \equiv (\alpha^2/8)(\SI{13.6}{\eV})$. We now need the off-diagonal elements. The relativistic part of the fine structure gives

\begin{align*}
	\bra{\psi_i}\mathcal{H}_\text{r}^\prime\ket{\psi_k} &= -\frac{1}{8m^3c^2}\bra{p^2\psi_i}\ket{p^2\psi_k} \\[4pt]
	&= -\frac{1}{8m^3c^2}\bra{\psi_i}[2m(E-V)]^2\ket{\psi_k} \\[4pt]
	&= -\frac{1}{2mc^2}\bra{\psi_i}(E-V(r))^2\ket{\psi_k}\bra{\psi_i}\ket{\psi_k}\bra{\chi_i}\ket{\chi_k}\,,
\end{align*}

where $\bra{\psi_k}\bra{\psi_i} = \delta_{{m_\ell}_i{m_\ell}_k}$ and $\bra{\chi_k}\ket{\chi_i} = \delta_{{m_s}_i {m_s}_k}$. With this, the only possible nonzero terms are $\bra{\psi_5}\mathcal{H}_\text{r}^\prime\ket{\psi_6}$ and $\bra{\psi_7}\mathcal{H}_\text{r}^\prime\ket{\psi_8}$, however it can be shown that these are also zero. The relativistic part of the fine structure only contributes diagonal matrix elements.

The spin-orbit Hamiltonian gives

\begin{align*}
	\bra{\psi_i}\mathcal{H}_\text{SO}^\prime\ket{\psi_k} &\propto \bra{\psi_i}\vb{L}\vdot\vb{S}\ket{\psi_k} \\[4pt]
	&= \frac{1}{2}\Bigl(\bra{\psi_i}J^2\ket{\psi_k} - \bra{\psi_i}L^2\ket{\psi_k} - \bra{\psi_i}S^2\ket{\psi_k}\Bigr)
\end{align*}

which requires either $\ell$, $j$ and $m_j$ or $\ell$, $m_\ell$ and $m_s$ to be the same. There are no states which satisfy these, so the spin-orbit part of the fine structure only contributes diagonal matrix elements. This means that fine structure only contributes diagonal elements; either $-\gamma$ or $-5\gamma$.

The easiest way to work out the matrix elements for the Zeeman Hamiltonian is to realize that the hydrogenic states $\ket{\ell, m_\ell, m_s}$ are eigenstates of that Hamiltonian with energies $\mu_\text{B}B_\text{ext}(m_\ell + 2m_s)$. For states $\psi_1$ to $\psi_4$ the composite states are only composed of one term and so 

\begin{align*}
	\bra{\psi_1}\mathcal{H}_\text{Z}^\prime\ket{\psi_1} &= \beta \\[4pt]
	\bra{\psi_2}\mathcal{H}_\text{Z}^\prime\ket{\psi_2} &= -\beta \\[4pt]
	\bra{\psi_3}\mathcal{H}_\text{Z}^\prime\ket{\psi_3} &= 2\beta \\[4pt]
	\bra{\psi_4}\mathcal{H}_\text{Z}^\prime\ket{\psi_4} &= -2\beta 
\end{align*}

where $\beta = \mu_\text{B}B_\text{ext}$. There are no off diagonal terms for these states. The states $\psi_5$ to $\psi_8$ are the sum of two terms. Applying the Hamiltonian gives

\begin{align*}
	\mathcal{H}_\text{Z}^\prime\ket{\psi_5} &= \beta
	\sqrt{\frac{2}{3}}\ket{1,\, 0}\ket{1/2,\, 1/2} \\[4pt]
	\mathcal{H}_\text{Z}^\prime\ket{\psi_6} &= \beta\frac{1}{\sqrt{3}}\ket{1,\, 0}\ket{1/2,\, 1/2} \\[4pt]
	\mathcal{H}_\text{Z}^\prime\ket{\psi_7} &= -\beta\sqrt{\frac{2}{3}}\ket{1,\, 0}\ket{1/2,\, -1/2} \\[4pt]
	\mathcal{H}_\text{Z}^\prime\ket{\psi_8} &= -\beta\frac{1}{\sqrt{3}}\ket{1,\, 0}\ket{1/2,\, -1/2}
\end{align*}

and from this we find

\begin{align*}
	\bra{\psi_5}\mathcal{H}_\text{Z}^\prime\ket{\psi_5} &= \frac{2}{3}\beta \\[4pt]
	\bra{\psi_6}\mathcal{H}_\text{Z}^\prime\ket{\psi_6} &= \frac{1}{3}\beta \\[4pt]
	\bra{\psi_7}\mathcal{H}_\text{Z}^\prime\ket{\psi_7} &= -\frac{2}{3}\beta \\[4pt]
	\bra{\psi_8}\mathcal{H}_\text{Z}^\prime\ket{\psi_8} &= -\frac{1}{3}\beta \\[4pt]
	\bra{\psi_5}\mathcal{H}_\text{Z}^\prime\ket{\psi_6} = \bra{\psi_6}\mathcal{H}_\text{Z}^\prime\ket{\psi_5}  &= -\frac{\sqrt{2}}{3}\beta \\[4pt]
	\bra{\psi_7}\mathcal{H}_\text{Z}^\prime\ket{\psi_8} = \bra{\psi_8}\mathcal{H}_\text{Z}^\prime\ket{\psi_7}  &= -\frac{\sqrt{2}}{3}\beta \,.
\end{align*}

We can now construct the Hamiltonian matrix in the subspace of degenerate states:

\begin{equation}
	W = \begin{bmatrix}
	\beta-5\gamma & 0 & 0 & 0 & 0 & 0 & 0 & 0 \\
	0 & -\beta-5\gamma & 0 & 0 & 0 & 0 & 0 & 0 \\
	0 & 0 & 2\beta-\gamma & 0 & 0 & 0 & 0 & 0 \\
	0 & 0 & 0 & -2\beta-\gamma & 0 & 0 & 0 & 0 \\
	0 & 0 & 0 & 0 & \frac{2}{3}\beta-\gamma & -\frac{\sqrt{2}}{3}\beta & 0 & 0 \\
	0 & 0 & 0 & 0 & -\frac{\sqrt{2}}{3}\beta & \frac{1}{3}\beta-5\gamma & 0 & 0 \\
	0 & 0 & 0 & 0 & 0 & 0 & -\frac{2}{3}\beta-\gamma & -\frac{\sqrt{2}}{3}\beta \\
	0 & 0 & 0 & 0 & 0 & 0 & -\frac{\sqrt{2}}{3}\beta & -\frac{1}{3}\beta-5\gamma \\
	\end{bmatrix}
\end{equation}

After solving for the eigenvalues, the corrections to ${E_2}^0 = -\SI{3.4}{\eV}$ are:

\begin{subequations}
\begin{align}
	E_{\psi_{1}}^\prime &= \beta-5\gamma \\[4pt]
	E_{\psi_{2}}^\prime &= -\beta-5\gamma \\[4pt]
	E_{\psi_{3}}^\prime &= 2\beta-\gamma \\[4pt]
	E_{\psi_{4}}^\prime &= -2\beta - \gamma \\[4pt]
	E_{\psi_{5}}^\prime &= \frac{1}{2}\beta - 3\gamma +\sqrt{4\gamma^2 + \frac{2}{3}\beta\gamma + \frac{1}{4}\beta^2}\\[4pt]
	E_{\psi_{6}}^\prime &= \frac{1}{2}\beta - 3\gamma - \sqrt{4\gamma^2 + \frac{2}{3}\beta\gamma + \frac{1}{4}\beta^2}\\[4pt]
	E_{\psi_{7}}^\prime &= -\frac{1}{2}\beta - 3\gamma + \sqrt{4\gamma^2 - \frac{2}{3}\beta\gamma + \frac{1}{4}\beta^2}\\[4pt]
	E_{\psi_{8}}^\prime &= -\frac{1}{2}\beta - 3\gamma - \sqrt{4\gamma^2 - \frac{2}{3}\beta\gamma + \frac{1}{4}\beta^2}
\end{align}
\end{subequations}

\subsection{Hyperfine splitting}
Hyperfine splitting arises from the interaction between the proton spin and the electron spin. Both result in a magnetic dipole moment:

\begin{equation}
\vb*{\mu}_p = \frac{g_p e}{2m_p}\,\vb{S}_p \quad;\quad \vb*{\mu}_e = -\frac{e}{2m}\,\vb{S}_e \,,
\end{equation}

where the proton g-factor is approximately 5.59. The electron moment (on the order of one Bohr magneton) is much larger than the proton moment, since the proton mass is much larger than that of an electron. From classical electrodynamics, a dipole $\vb*{\mu}$ creates a magnetic field,

\begin{equation}
\vb{B} = \frac{\mu_0}{4\pi r^3}\Bigl[3(\vb*{\mu}\vdot\vb{r})\vu{r} - \vb*{\mu}\Bigr] + \frac{2\mu_0}{3}\vb*{\mu}\,\delta^3(\vb{r}) \,.
\end{equation}

The Hamiltonian that describes the electron in a magnetic field produced by the proton spin is

\begin{equation}
\mathcal{H}_\text{hf}^\prime = -\vb*{\mu}_e\vdot\vb{B} = \frac{\mu_0g_pe^2}{8\pi m_p m_e}\frac{3(\vb{S}_p\vdot\vu{r})(\vb{S}_e\vdot\vu{r})-\vb{S}_p\vdot\vb{S}_e}{r^3} + \frac{\mu_0g_pe^2}{3m_p m_e}\,\vb{S}_p\vdot\vb{S}_e\,\delta^3(\vb{r}) \,.
\end{equation}

Using perturbation theory, the first order correction to the energy is

\begin{equation}
E_\text{hf}^\prime = \frac{mu_0g_pe^2}{8\pi m_p m_e}\exv{\frac{3(\vb{S}_p\vdot\vu{r})(\vb{S}_e\vdot\vu{r})-\vb{S}_p\vdot\vb{S}_e}{r^3}} + \frac{\mu_0 g_p e^2}{3 m_p m_e}\exv{\vb{S}_p\vdot\vb{S}_e}|\psi(0)|^2 \,.
\end{equation}

Here we will focus on the ground state of hydrogen. Since it is spherically symmetric, the first term in the energy vanishes. For any $\ell = 0$ state, we have 

\begin{equation*}
E_\text{hf}^\prime = \frac{\mu_0 g_p e^2}{3\pi m_p m_e {a_0}^3}\exv{\vb{S}_p\vdot\vb{S}_e} \,,
\end{equation*}

where we have used $|\psi_{100}(0)|^2 = 1/\pi{a_0}^3$. This is \textbf{spin-spin coupling}, similar to the spin-orbit coupling. With this coupling, the individual spin angular momenta are not conserved. The total spin is conserved and as a result, the \textit{good} states are eigenvectors of 

\begin{equation}
\vb{S} = \vb{S}_e + \vb{S}_p \,.
\end{equation}

We use the same approach as before, were $S^2 = {S_e}^2 + {S_p}^2 + 2\vb{S}_e\vdot\vb{S}_p$, or

\begin{equation}
\vb{S}_e\vdot\vb{S}_p = \frac{1}{2}\Bigl(S^2 - {S_e}^2 - {S_p}^2 \Bigr) = \,.
\end{equation}

Since electrons and protons are both $s=1/2$, we obtain

\begin{equation*}
E_\text{hf}^\prime = \frac{\mu_0 g_p e^2\hbar^2}{6\pi m_p m_e {a_0}^3}\qty[s(s+1) - \frac{3}{2}] \,,
\end{equation*}

where $s$ is the total spin quantum number; either 0 or 1. This gives overall a triplet or singlet state.

This is

\begin{subequations}
	\begin{align}
	E_\text{triplet}^\prime &= \frac{g_p\hbar^4}{3 m_p m_e c^2 {a_0}^4} \\[4pt]
	E_\text{singlet}^\prime &= -\frac{g_p\hbar^4}{ m_p m_e c^2 {a_0}^4}
	\end{align}
\end{subequations}

which breaks the degeneracy of the ground state, increasing the energy of the triplet state and lowering the energy of the singlet state. The energy gap is

\begin{equation*}
\Delta E = \frac{4 g_p \hbar^4}{3 m_p m_e c^2 {a_0}^4} \approx \SI{5.88}{\micro\eV} \,.
\end{equation*}

This energy difference corresponds to a frequency of \SI{1420}{\mega\hertz}, or a wavelength of \SI{21}{\centi\metre}. This famous 21 centimetre line is one of the most pervasive forms of radiation in the universe, created when the proton and electron in the ground state of hydrogen change their spin configuration. 

\section{The Variational Principle}
\subsection{Theory of variation}
Suppose we would like to calculate the ground state energy $E_\text{gs}$ for a system described by a Hamiltonian $\mathcal{H}$. If we are unable to solve the Schrödinger equation exactly, we can obtain an upper bound for this energy using the variational principle. Oftentimes an upper bound is all that is needed and this bound is sometimes close to the exact value.

\begin{mdframed}
\paragraph*{Remark:}
For any normalized function $\psi$ we have

\begin{equation}
	E_\text{gs} \leq \bra{\psi}\mathcal{H}\ket{\psi} \equiv \exv{\mathcal{H}} \,.
\end{equation}

The expectation value of the Hamiltonian with some trial function $\psi$ will overestimate the ground state energy. The ground state wavefunction $\psi_\text{gs}$ uniquely minimizes $\exv{\mathcal{H}}$ to give $E_\text{gs}$.

\paragraph{Proof:}
The (unknown) eigenfunctions of $\mathcal{H}$ form a complete basis $\ket{\phi_n}$ so we can express $\psi$ as a linear combination of these basis functions:

\begin{equation*}
	\ket{\psi} = \sum_n C_n\ket{\phi_n} \quad\text{with}\quad\mathcal{H}\ket{\phi_n} = E_n\ket{\phi_n}\,.
\end{equation*}

Since $\psi$ is normalized, 

\begin{align*}
	\bra{\psi}\ket{\psi} &= \bra{\sum_mC_m\ket\phi_n}\ket{\sum_n C_n\ket{\phi_n}} \\[4pt]
	&= \sum_m\sum_n {C_m}^*C_n\bra{\psi_m}\ket{\psi_n} \\[4pt]
	&= \sum_m\sum_n {C_m}^*C_n \delta_{mn} = \sum_n |C_n|^2 = 1 \,,
\end{align*}

meaning we can write

\begin{align*}
	\exv{\mathcal{H}} &= \bra{\sum_mC_m\ket\phi_n}\mathcal{H}\ket{\sum_n C_n\ket{\phi_n}} \\[4pt]
	&= \sum_m\sum_n {C_m}^*E_n C_n\bra{\psi_m}\ket{\psi_n} \\[4pt]
	&= \sum_n E_n|C_n|^2 \,.
\end{align*}

By definition, the lowest eigenvalue is $E_\text{gs}$, thus

\begin{equation*}
	\exv{\mathcal{H}} \geq E_\text{gs}\sum_n|C_n|^2 = E_\text{gs}
\end{equation*}
\hspace*{\fill}$\blacksquare$
\end{mdframed}

\begin{mdframed}[backgroundcolor=gray!20]
\paragraph*{Example:}
Use a variational approach to determine an estimate of the ground state energy of the harmonic oscillator. Use a Gaussian trial function.

\paragraph*{Solution:}
We know the Hamiltonian and the exact ground state energy:

\begin{equation*}
	\mathcal{H} = -\frac{\hbar^2}{2m}\dv[2]{x} + \frac{1}{2}m\omega^2x^2 \qquad E_\text{gs} = \frac{\hbar\omega}{2} \,.
\end{equation*}

We select a trial (normalized) wavefunction to be

\begin{equation*}
	\psi(x) = \qty(\frac{2b}{\pi})^{1/4}\exp(-bx^2) \,.
\end{equation*}

We write the Hamiltonian as a kinetic and potential term, which gives $\exv{\mathcal{H}} = \exv{K} + \exv{V}$. We find:

\begin{align*}
	\exv{K} &= -\frac{\hbar^2}{2m}\qty(\frac{2b}{\pi})^{1/2}\int_{-\infty}^{\infty}e^{-bx^2}\dv[2]{x}e^{-bx^2}\dd{x} = \frac{\hbar^2 b}{2m} \,; \\[4pt]
	\exv{V} &= \frac{1}{2}m\omega^2\qty(\frac{2b}{\pi})^{1/2}\int_{-\infty}^{\infty}x^2e^{-bx^2}\dd{x} = \frac{m\omega^2}{8b} \,; \\[4pt]
	\exv{\mathcal{H}} &= \frac{\hbar^2b}{2m} + \frac{m\omega^2}{8b} \,.
\end{align*}

This result is greater than or equal to the ground state energy, for any $b$. We minimize $\exv{\mathcal{H}}$ with respect to $b$ to determine the best upper bound:

\begin{equation}
	\dv[2]{\exv{\mathcal{H}}}{x} = \frac{\hbar^2}{2m} = 0 \quad\implies\quad b = \frac{m\omega}{2\hbar} \,.
\end{equation}

Using this value of $b$, we find

\begin{equation*}
	\exv{\exv{\mathcal{H}}}_\text{min} = \frac{\hbar\omega}{2} \,,
\end{equation*}

which is the exact result. We found this result because we picked a trial wavefunction of the same form of the true ground state wavefunction.

\textbf{A key part of the variational approach is to attempt to make a good choice for the trial wavefunction}.
\end{mdframed}

\begin{mdframed}[backgroundcolor=gray!20]
\paragraph*{Example:}
Use a Gaussian trial function to obtain an estimate on the ground state energy of an electron in a delta function potential (where the true energy is also known),

\begin{equation*}
	\mathcal{H} = -\frac{\hbar^2}{2m}\dv[2]{x} - \alpha\delta(x) \quad E_\text{gs} = -\frac{m\alpha^2}{2\hbar^2} \,.
\end{equation*}

\paragraph*{Solution:}
The expectation value of the kinetic term will be the same as in the previous example. The potential term gives

\begin{equation*}
	\exv{V} = -\alpha\qty(\frac{2b}{\pi}^{1/2})\int_{-\infty}^{\infty}e^{-bx^2}\delta(x)\dd{x} = -\alpha\sqrt{\frac{2b}{\pi}} \,.
\end{equation*}

The expectation of the Hamiltonian is therefore

\begin{equation*}
	\exv{\mathcal{H}} = \frac{\hbar^2b}{2m} - -\alpha\sqrt{\frac{2b}{\pi}}\,,
\end{equation*}

which has a minimum value at $b = 2m^2\alpha^2/\pi\hbar^4$, giving

\begin{equation*}
	\exv{\mathcal{H}}_\text{min} = -\frac{m\alpha^2}{\pi\hbar^4} \,,
\end{equation*}

which is larger than the exact ground state energy, since $\pi>2$.
\end{mdframed}

\subsection{The ground state of helium}
The helium atom has two electrons orbiting a nucleus of two protons. The Hamiltonian (neglecting small corrections) is 

\begin{equation}
	\mathcal{H} = -\frac{\hbar^2}{2m}\Bigl(\laplacian{}_1 + \laplacian{}_1\Bigr) - \frac{e^2}{4\pi\epsilon_0}\qty(\frac{2}{r_1} + \frac{2}{r_2} - \frac{1}{|\vb{r}_1 - \vb{r}_2|}) \,.
\end{equation}

%% Figure: Coordinates of electrons

We seek the ground state energy. The electron-electron interaction prevents us from solving the Schrödinger equation directly, as we saw earlier. Ignoring the electron-electron interaction gave a ground state that was the product of two hydrogenic wavefunctions,

\begin{equation*}
	\psi_0(\vb{r}_1,\vb{r}_2) = \psi_{100}(\vb{r}_1)\psi_{100}(\vb{r}_2) = \frac{8}{\pi{a_0}^3}\exp\qty(-\frac{2(r_2 + r_2)}{a_0} \,,)
\end{equation*}

with energy $8E_1 = -\SI{109}{\eV}$ whereas the experimental ground state energy is $-\SI{79}{\eV}$. We apply the variational principle using this as a trial wavefunction:

\begin{equation}
	\mathcal{H}\psi_0 = (8E_1 + V_{e-e})\psi_0 \,.
\end{equation}

This means that $\exv{\mathcal{H}} = 8E_1 + \exv{V_{e-e}}$, where

\begin{equation}
	\exv{V_{e-e}} = \frac{e^2}{4\pi\epsilon_0}\qty(\frac{8}{\pi {a_0}^3})^2\iint\frac{\exp\bigl(-4(r_1+r_2)/a\bigr)}{|\vb{r}_1 - \vb{r}_2|}\dd[3]{r_1}\dd[3]{r_2} \,.
\end{equation}

It can be shown that this integral converges to $5E_1/2 = -\SI{34}{\eV}$. This is the same as the correction obtained from first order perturbation theory.  This gives $\exv{\mathcal{H}} = -\SI{75}{\eV}$, which is a close upper bound to $-\SI{79}{\eV}$.

We can obtain a more accurate estimate by choosing a better trial wavefunction. Physically we might expect one electron to screen the other, and therefore each electron will experience a smaller effective nuclear charge $Z \leq 2$. This motivates us to use the wavefunction

\begin{equation}
	\psi_1(\vb{r}_1,\vb{r}_2) = \frac{Z^3}{\pi{a_0}^3}\exp\qty(-\frac{Z(r_1 + _2)}{a_0}) \,,
\end{equation}

where $Z$ is a variational parameter used to minimize the expectation value of the Hamiltonian. We can rewrite the Hamiltonian as

\begin{equation}
	\mathcal{H} = -\frac{\hbar^2}{2m}(\laplacian{}_1 + \laplacian{}_2) - \frac{e^2}{4\pi\epsilon_0}\qty(\frac{Z}{r_1} + \frac{Z}{r_2}) + \frac{e^2}{4\pi\epsilon_0}\qty[\frac{Z-2}{r_1} + \frac{Z-2}{r_2} + \frac{1}{|\vb{r}_1 - \vb{r}_2|}] \,,
\end{equation}

which gives

\begin{equation}
	\exv{\mathcal{H}} = 2Z^2E_1 + 2(Z-2)\qty(\frac{e^2}{4\pi\epsilon_0})\exv{\frac{1}{r}} + \exv{V_{e-e}} \,.
\end{equation}

The expectation value $\exv{1/r} = \bra{\psi_{100}}\frac{1}{r}\ket{\psi_{100}}$ is the expectation value for the single-particle hydrogen ground state, with nuclear charge $Z$. It can be shown that

\begin{equation}
	\exv{\frac{1}{r}} = \frac{Z}{a_0} \,.
\end{equation}

The expectation of the electron-electron term is the same as before, except with $Z$ instead of $2$:

\begin{equation}
	\exv{V_{e-e}} = \frac{5Z}{8a_0}\qty(\frac{e^2}{4\pi\epsilon_0}) = -\frac{5Z}{4}E_1\,.
\end{equation}

Putting this all together we find, 

\begin{equation}
	\exv{\mathcal{H}} = \qty[-2Z^2 + \frac{27}{4}Z] E_1 \,,
\end{equation}

which has minimum value at $Z = 27/16 \approx 1.69$. Using this optimal value of $Z$, we find that

\begin{equation*}
	\exv{\mathcal{H}} = \frac{1}{2}\qty(\frac{3}{2})^6 E_1 = -\SI{77.5}{\eV} \,.
\end{equation*}

With a clever, physically-motivated choice in the trial function, we have obtained an estimate which is within 2\% from the experimental ground state.

\subsection{The singly ionized hydrogen molecule}
Another classic example is the application of the variational principle to the hydrogen molecule ion, $\ce{H2+}$, which has one electron and two protons. The Hamiltonian is

\begin{equation}
	\mathcal{H} = -\frac{\hbar^2}{2m}\laplacian{} - \frac{e^2}{4\pi\epsilon_0}\qty(\frac{1}{r_1} + \frac{1}{r_2}) \,.
\end{equation}

%% Figure: geometry of coordinates

Our challenge is to formulate a good guess for the trial wavefunction. One reasonable guess is to assume that $\psi$ is the sum of the two hydrogen wavefunctions centered around each proton:

\begin{equation}
	\psi = A\Bigl[\psi_0(\vb{r}_1) + \psi_0(\vb{r}_2)\Bigr] \,
\end{equation}

where 

\begin{equation}
	\psi_0 = \frac{1}{\sqrt{\pi {a_0}^3}}\,\exp(-r/a_0) \,.
\end{equation}

This is known as the linear combination of atomic orbitals (LCAO) technique. First we need to normalize $\psi$:

\begin{align*}
	\iint|\psi|^2\dd[3]{r_1}\dd[3]{r_2} &= |A|^2\qty[\int|\psi_0(r_1)|^2\dd[3]{r_1} + \int|\psi_0(r_2)|^2\dd[3]{r_2} + 2\iint\psi_0(r_1)\psi_0(r_2)\dd[3]{r_1}\dd[3]{r_2}] \\[4pt]
	&= 2|A^2|\qty[1 + \iint\psi_0(r_1)\psi_0(r_2)\dd[3]{r_1}\dd[3]{r_2}] \\[4pt]
	&= 2|A|^2(1+I) = 1\,,
\end{align*}

where $I\equiv\bra{\psi_0(r_1)}\ket{\psi_0(r_2)}$ is given by 

\begin{equation*}
	I = \frac{1}{\pi{a_0}^3}\iint\exp(-\frac{r_1 + r_2}{a_0})\dd[3]{r_1}\dd[3]{r_2} \,.
\end{equation*}

This is an \textbf{overlap intergral}, which measures the amount by which $\psi_0(r)$ overlaps $\psi_0(r^\prime)$. It goes to zero for $R\rightarrow\infty$ and to unity for $R\rightarrow 0$. To evaluate this integral, we choose coordinates such that one proton is at the origin, and the other is on the $z$-axis at a point $R$.

%% Figure: geometry of the setup

This means that the distance between the first proton and the electron is $r_1 = r$ and the distance between the second proton and the electron is $r_2 = |\vb{r} - \vb{R}| = \sqrt{r^2 + R^2 - 2\vb{r}\vdot\vb{R}}$, given by the cosine rule. This gives

\begin{align*}
I &= \frac{1}{\pi{a_0}^3}\int_{r=0}^{\infty}\int_{\phi=0}^{2\pi}\int_{\theta=0}^{\pi}\exp\qty(-\frac{r + \sqrt{r^2 + R^2 - 2rR\cos\theta}}{a_0})r^2\sin\theta\dd{\theta}\dd{\phi}\dd{r} \\[4pt]
&= e^{-R/a_0}\qty[1 + \qty(\frac{R}{a_0} + \frac{1}{3}\qty(\frac{R}{a_0})^2)]\,.
\end{align*}

Now we can write

\begin{equation}
	A = \frac{1}{\sqrt{2 + 2e^{-R/a_0}\qty[1 + \qty(\frac{R}{a_0} + \frac{1}{3}\qty(\frac{R}{a_0})^2)]}} \,,
\end{equation}

which means that the Hamiltonian is

\begin{align}
	\mathcal{H}\psi &= A\qty[-\frac{\hbar^2}{2m}\laplacian{} - \frac{e^2}{4\pi\epsilon_0}\qty(\frac{1}{r_1} + \frac{1}{r_2})]\Bigl[\psi_0(\vb{r}_1) + \psi_0(\vb{r}_2)\Bigr] \nonumber \\[4pt] 
	&= E_1\psi - A\qty(\frac{e^2}{4\pi\epsilon_0})\qty[\frac{1}{r_2}\psi_0(r_1) + \frac{1}{r_1}\psi_0(r_2)] \,,
	\,,
\end{align}

meaning

\begin{equation}
	\exv{\mathcal{H}} = E_1 - 2|A|^2\qty(\frac{e^2}{4\pi\epsilon_0})\qty[\bra{\psi_0(r_1)}1/r_2\ket{\psi(r_1)} + \bra{\psi_0(r_1)}1/r_1\ket{\psi_0(r_2)}] \,.
\end{equation}

We define two integrals:

\begin{subequations}
\begin{align}
	D &\equiv a\bra{\psi_0(r_1)}1/r_2\ket{\psi(r_1)} \qquad\text{``Direct integral"} \\[4pt]
	X &\equiv a\bra{\psi_0(r_1)}1/r_1\ket{\psi_0(r_2)} \qquad\text{``Exchange integral"}
\end{align}
\end{subequations}

which work out to

\begin{subequations}
	\begin{align}
	D &= \frac{a_0}{R} - \qty(1 + \frac{a_0}{R})e^{-2R/a_0} \,; \\[4pt]
	X &= \qty(1 + \frac{R}{a_0})e^{-R/a} \,.
	\end{align}
\end{subequations}

Putting everything together, we find that 

\begin{equation}
	\exv{\mathcal{H}} = \qty[1 + 2\frac{D+X}{1+I}]E_1 \,.
\end{equation}

We also need to introduce the proton-proton potential energy,

\begin{equation*}
	V_{p-p} = \frac{e^2}{4\pi\epsilon_0}\frac{1}{R} = -\frac{2a_0}{R}\,E_1\,,
\end{equation*}

which is a simple Coulomb repulsion. The total energy of the system, in units of $-E_1$ is

\begin{equation}
	F(x) = -1 + \frac{2}{x}\qty[\frac{(1-\frac{2}{3}x^2)e^{-x} + (1+x)e^{-2x}}{1 + (1 + x + \frac{1}{3}x^2)e^{-x}}] \,,
\end{equation}

where $x \equiv R/a_0$. There is an optimal $R$ that minimizes $\exv{\mathcal{H}}$. 

%% Figure: plot of F(x)

Since the total energy is negative, we can infer that \ce{H2+} is stable, with both protons sharing the electron in a covalent bond. The bond length is $\SI{1.3}{\angstrom}$ (experiment is $\SI{1.06}{\angstrom}$) and the bond energy is $\SI{1.8}{\eV}$ (experiment is $\SI{2.8}{\eV}$).

The variational principle overestimates the ground state energy and hence underestimates the bond strength. However, we gain a lot of qualitative insight into the problem. Choosing a more sophisticated trial wavefunction would yield an even closer result.

\section{The WKB approximation}
The WKB (\textit{Wentzel, Kramers, Brillouin}) approximation technique for solving the the time-independent Schrödinger equation in one dimension. It is especially useful in calculating bound state energies and tunneling rates through potential barriers.

We assume that a particle of energy $E$ is moving through a region where the potential $V$ is constant. If the particle's energy is larger than the potential, the wavefunction is 

\begin{equation}
	\psi(x) = Ae^{\pm iK x} \quad, \qquad K \equiv \frac{1}{\hbar}\sqrt{2m(E-V)} \,.
\end{equation}

The wavefunction is oscillatory, with wavelength $\lambda = 2\pi/K$, and amplitude $A$.

Now let us suppose that the potential is not constant, but smoothly varying compared to $\lambda$, so that over a region containing many full wavelengths the potential is essentially constant. It is then reasonable to assume that $\psi$ will remain sinusoidal, except that the wavelength and amplitude change slowly with $x$. This is the idea behind the WKB approximation.

If we consider the case when the potential is higher than the particle energy, the wavefunction is exponential,

\begin{equation}
	\psi(x) = Ae^{\pm Kx} \quad, \qquad K \equiv \frac{1}{\hbar}\sqrt{2m(V-E)} \,.
\end{equation}

If the potential is not constant but is slowly varying compared to $1/K$, the wavefunction remains almost exponential, except that $A$ and $K$ are slowly varying functions of $x$.

\begin{mdframed}
	\paragraph*{Note:} when $E \approx V$, $\lambda\sim 1/K \rightarrow\infty$. Then $V(x)$ cannot be considered as slowly varying. These points are referred to as \textbf{turning points}. The problem can be solved but requires special treatment.
\end{mdframed}

\subsection{The classical region}
We start by rewriting the Schrödinger equation as

\begin{equation}
	\dv[2]{\psi}{x} = -\frac{p^2}{\hbar^2}\psi \,,
\end{equation} 

where

\begin{equation}
	p(x) = \sqrt{2m(E-V(x))} \,.
\end{equation}

Here we assume that $E>V(x)$, such that $p(x)$ is real. We call this the \textit{classical region}.

In general for some complex $\psi$ we can write

\begin{equation}
	\psi(x) = A(x)e^{i\phi(x)} ,,
\end{equation}

where $A$ and $\phi$ are real. We start by differentiating this expression to obtain

\begin{align}
	\dv{\psi}{x} &= \qty[\dv{A}{x} + iA\dv{\phi}{x}]e^{i\phi}\,; \\[4pt]
	\dv[2]{\psi}{x} &= \qty[\dv[2]{\psi}{x} + 2i\dv{A}{x}\dv{\phi}{x} + iA\dv{\phi}{x} - A\qty(\dv{\phi}{x})^2]e^{i\phi} \,.
\end{align}

Inserting this into the rearranged Schrödinger equation, we obtain

\begin{equation}
	\dv[2]{A}{x} + 2i\dv{A}{x}\dv{\phi}{x} + iA\dv[2]{\phi}{x} - A\qty(\dv{\phi}{x})^2 = -\qty(\frac{p^2}{\hbar^2})A \,.
\end{equation}

Taking the real and imaginary parts, we obtain two equations:

\begin{subequations}
\begin{align}
	&\dv[2]{A}{x} = A\qty[\qty(\dv{\phi}{x})^2 - \frac{p^2}{\hbar^2}]\,; \\[4pt]
	&2\dv{A}{x}\dv{\phi}{x} + A\dv[2]{\phi}{x} = \dv{x}\qty(A^2\dv{\phi}{x}) = 0\,,
\end{align}
\end{subequations}
which are equivalent to the Schrödinger equation. The second equation is easy to solve:

\begin{equation}
	A = \frac{C}{\sqrt{\dv{\phi}{x}}} \,,
\end{equation}

where $C$ is a real constant.

To solve the other equation, we make the approximation that $\dv*[2](A{x})$ is negligible, because the amplitude $A$ varies slowly. We have:

\begin{equation}
	\qty(\dv[2]{\phi}{x})^2 = \frac{p^2}{\hbar^2} \quad\text{or}\quad \dv{\phi}{x} = \pm\frac{p}{\hbar} \,.
\end{equation}

Therefore, 

\begin{equation}
	\phi(x) = \pm \frac{1}{\hbar}\int p(x)\dd{x}\,.
\end{equation}

It follows that

\begin{equation}
	\boxed{ \psi(x) \approxeq \frac{C}{\sqrt{p(x)}}\exp\qty(\pm\frac{i}{\hbar}\int p(x)\dd{x}) }
\end{equation}

and the general solution is a linear combination of the two ($\pm$) terms.

\begin{mdframed}
\paragraph*{Note:} the probability density is approximately given by

\begin{equation}
	|\psi(x)|^2 \approxeq \frac{C^2}{p(x)} \,,
\end{equation}

which says that the probability of finding a particle at $x$ is inversely proportional to its velocity. Intuitively, a particle moving rapidly in a region is less likely to be caught there.
\end{mdframed}

\begin{mdframed}[backgroundcolor=gray!20]
\paragraph*{Example:}
Apply the WKB approximation to the case of an infinite square well, with a bumpy potential at the bottom:

\begin{equation*}
	V(x) = \begin{cases}
	\text{something} & 0<x<a \\
	0 & \text{otherwise}
	\end{cases}
	\,.
\end{equation*}

\paragraph*{Solution:}
Inside the well, assuming that $E>V$, we have

\begin{equation*}
	\psi(x) \approxeq \frac{1}{\sqrt{p(x)}}\Bigl[C_+ e^{i\phi(x)} + C_- e^{-i\phi(x)}\Bigr] = \frac{1}{\sqrt{p(x)}}\Bigl[C_1\sin(\phi(x)) + C_2\cos(\phi(x))\Bigr] \,.
\end{equation*}

where

\begin{equation*}
	\phi(x) = \frac{1}{\hbar}\int_{0}^{x}p(x^\prime)\dd{x^\prime} \,/
\end{equation*}

Applying boundary conditions:

\begin{align*}
	\left\{
	\begin{array}{l}
	\psi(0) = 0 \quad\rightarrow\quad C_2 = 0 \\[4pt]
	\psi(a) = 0 \quad\rightarrow\quad \sin(\phi(a)) = 0
	\end{array}	
	\right. \\[4pt]
	\implies \phi(a) = n\pi\,\quad n=1,\, 2,\, 3,\,\ldots \\[4pt]
	\implies \int_{0}^{a}p(x)\dd{x} = n\pi\hbar \,.
\end{align*}
This condition gives quantized energy levels.

\paragraph*{Flat bottom:}
If the bottom is flat ($V(x) = 0$) we recover the original infinite square well. Then $p = \sqrt{2mE}$ and

\begin{equation*}
	\int_{0}^{a}p(x)\dd{x} = a\sqrt{2mE} = n\hbar\pi \quad\implies\quad E_n = \frac{\hbar^2\pi^2n^2}{2ma^2} \,,
\end{equation*}

which is the exact result (since $\dv*[2]{A}{x}$ is identically zero in this case).

\paragraph*{Shelf potential:}
Here we consider a potential

\begin{equation*}
	V(x) = \begin{cases}
	V_0 & 0<x<a/2 \\
	0 & a/2<x<a \\
	\infty & \text{otherwise}
	\end{cases} \,.
\end{equation*}

This gives

\begin{equation*}
	\int_{0}^{a}p(x)\dd{x} = \int_{0}^{a/2}\sqrt{2m(E-V_0)}\dd{x} + \int_{a/2}^{a}\sqrt{2mE}\dd{x} = \frac{a}{2}\sqrt{2m(E-V_0)} + \frac{a}{2}\sqrt{2mE} \,,
\end{equation*}

from which we find that 

\begin{equation*}
	E_n = \frac{\hbar^2\pi^2n^2}{2ma^2} + \frac{V_0}{2} + \frac{{V_0}^2ma^2}{8\hbar^2\pi^2n^2} \,.
\end{equation*}

When we used perturbation theory, we found $E = {E_n}^0 + V_0/2$, which is the same as this result for small perturbations.
\end{mdframed}

\subsection{Tunneling}
So far we have considered the the case where the particle energy is larger than the potential and $p(x)$ is real. Next, we study the nonclassical region where the potential is higher than the particle's energy and $p(x)$ is imaginary. The wavefunction is the same as before:

\begin{equation}
	\psi(x) \approxeq \frac{C}{\sqrt{|p(x)|}}\exp\qty(\pm\frac{1}{\hbar}\int |p(x)|\dd{x}) \,.
\end{equation}

For illustration, let us consider a rectangular barrier with a bumpy top. 

%% Figure: barrier

To the left of the barrier $(x<0)$, we have an oscillatory solution:

\begin{equation}
	\psi(x) = Ae^{ikx} + Be^{-ikx} \,,
\end{equation} 

where $A$ and $B$ are the incident and reflected amplitudes, and $k = \sqrt{2mE}/\hbar$. To the right of the barrier ($x>a$), we obtain an exponential solution,

\begin{equation}
	\psi(x) = Fe^{ikx} \,,
\end{equation}

where $F$ is the transmitted amplitude and the transmission probability is

\begin{equation}
	T = \frac{|F|^2}{|A|^2} \,.
\end{equation}

In the tunneling region $(0<x<a)$, the WKB approximation gives:

\begin{equation}
	\psi(x) \simeq \frac{C}{\sqrt{|p(x)|}}\exp\qty(\frac{1}{\hbar}\int_{0}^{x}|p(x^\prime)|)\dd{x^\prime} + \frac{D}{\sqrt{|p(x)|}}\exp\qty(-\frac{1}{\hbar}\int_{0}^{x}|p(x^\prime)|)\dd{x^\prime} \,.
\end{equation}

In the case of a high and/or wide barrier (\textit{i.e.} when the tunneling probability is small), then the momentum integral is large and the coefficient $C$ is small, due to the normalization condition. For a barrier of infinite width, $C=0$.

%% Figure: tunneling wavefunction

The ratio of the transmitted and incident amplitudes is proportional to the exponential decrease over the nonclassical region:

\begin{equation*}
	\frac{|F|}{|A|} \sim \exp\qty(-\frac{1}{\hbar}\int_{0}^{a}|p(x^\prime)|)\dd{x^\prime} = \exp\qty(-\frac{\sqrt{2m}}{\hbar}\int_{0}^{a}\sqrt{V(x) - E}\dd{x}) \,,
\end{equation*}

so that 

\begin{equation}
	T = e^{-2\gamma} \quad\text{with}\quad \gamma \equiv \frac{\sqrt{2m}}{\hbar}\int_{0}^{a}\sqrt{V(x) = E}\dd{x} \,.
\end{equation}

\begin{mdframed}[backgroundcolor=gray!20]
\paragraph*{Example:}
\textbf{Gamov's theory of alpha decay}. Alpha emission occurs when two protons ans two neutrons overcome the weak nuclear force and are expelled from a radioactive nucleus. As soon as the alpha particle escapes the nuclear force, it is repelled by the electrostatic force. Gamov approximated the potential between the nucleus and alpha particle as:

\begin{equation}
	V(r) = \begin{cases}
	-V_0 & 0 < < r_1 \\[4pt]
	\frac{(Ze)(4e)}{2\pi\epsilon_0 r} & r > r_1
	\end{cases}
\end{equation}

%% Figure: Gamov potential

We can understand the decay lifetimes by treating tunneling with the WKB approximation. If $E$ is the energy of the emitted alpha particle, then the position $r_2$ where the potential is equal to the energy is given by

\begin{equation}
	\frac{1}{4\pi\epsilon_0}\frac{2Ze^2}{r_2} = E \,.
\end{equation}

The region $r_1<r<r_2$ is the tunneling region. The exponent $\gamma$ is therefore 

\begin{equation*}
	\gamma = \frac{1}{\hbar}\int_{r_1}^{r^2}\sqrt{2m\qty(\frac{Ze^2}{2\pi\epsilon_0 r} - E)}\dd{r} = \frac{\sqrt{2mE}}{\hbar} \int_{r_1}^{r_2}\sqrt{\frac{r_2}{r} - 1} \dd{r} \,.
\end{equation*}

Using $r = r_2\sin^2 u$, the integral can be solved:

\begin{equation}
	\gamma = \frac{\sqrt{2mE}}{\hbar}\qty[r_2\qty(\frac{\pi}{2} - \arcsin\sqrt{\frac{r_1}{r_2}}) - \sqrt{r_1(r_2-r_1)}] \,.
\end{equation}

Typically $r_1 \ll r_2$ so we can expand this to first order and obtain

\begin{equation}
	\gamma \approx \frac{\sqrt{2mE}}{\hbar}\qty[\frac{\pi}{2}r_2 - 2\sqrt{r_1 r_2}] = K_1\frac{Z}{\sqrt{E}} - K_2 \sqrt{Z r_1} \,,
\end{equation}

where

\begin{subequations}
\begin{align}
	K_1 &\equiv \qty(\frac{e^2}{4\pi\epsilon_0})\frac{\pi\sqrt{2m}}{\hbar} = 1.980(\si{\mega\electronvolt})^{1/2}\,, \\[4pt]
	K_2 &\equiv \qty(\frac{e^2}{4\pi\epsilon_0})^{1/2}\frac{4\sqrt{m}}{\hbar} = 1.485(\si{\femto\metre})^{1/2} \,.
\end{align}
\end{subequations}

If we imagine the alpha particle classically inside the nucleus, sometimes hitting the wall, the escape probability is $e^{-2\gamma}$, with an average velocity $v$, and time between collisions $2r_1/v$. The average lifetime is therefore

\begin{equation}
	\tau = \qty(\frac{2r_1}{v})\exp\qty(\frac{2K_1Z}{\sqrt{E}} - 2K_2\sqrt{Zr_1}) \,.
\end{equation} 

The exponentially measured lifetimes show this exponential trend.
\end{mdframed}

\subsection{The connection formulae}
So far we have considered cases with vertical \textit{walls} to the potential well or barrier. Here we focus on when $E = V(x)$, where the classical region joins the nonclassical region. Here we focus on the bound state problem. 

Assuming that $E = V$ at $x=0$, the WKB wavefunction is

\begin{equation}
	\psi(x) \simeq 
	\begin{cases}
	\frac{1}{\sqrt{p(x)}}\qty[B\exp\qty(\frac{i}{\hbar}\int_{x}^{0}p(x^\prime)\dd{x^\prime}) + C\exp\qty(-\frac{i}{\hbar}\int_{x}^{0}p(x^\prime)\dd{x^\prime})] & x<0 \\[4pt]
	\frac{1}{\sqrt{|p(x)|}} D \exp\qty(-\frac{1}{\hbar}\int_{0}^{x}|p(x^\prime)|\dd{x^\prime}) & x>0
	\end{cases}
\end{equation}

%% Figure: wavefunction

where we have assumed that $V(x)>E$ for all $x>0$. 

We want to join both solutions at $x=0$, however $\psi\rightarrow\infty$ as $x\rightarrow 0$. Our strategy is to linearize $V(x)$ near $x=0$ and solve for the wavefunction in a \textit{patching region}. We can use this to stitch together the two wavefunctions.

To find the patching wavefunction $\psi_p$, we approximate the potential,

\begin{equation}
	V(x) \approx E + x\eval{\dv{V}{x}}_{x=0} \,.
\end{equation}

We need to solve:

\begin{equation*}
	-\frac{\hbar^2}{2m}\dv[2]{\psi_p}{x} + \qty[E + x\eval{\dv{V}{x}}_{x=0}\,]\psi_p = E\psi_p \,,
\end{equation*}

or

\begin{equation}
	\dv[2]{\psi_p}{x} = \alpha^3 x \psi_p \,,
\end{equation}

where

\begin{equation}
	\alpha \equiv \qty(\frac{2m}{\hbar^2}\eval{\dv{V}{x}}_{x=0}\,)^{1/3} \,.
\end{equation}

We define $ = \alpha x$, which gives

\begin{equation}
	\dv[2]{\psi_p}{z} = z\psi_p \,,
\end{equation}

which is known as \textbf{Airy's equation}, with a set of solutions known as \textbf{Airy functions}. There are two linearly independent solutions $\Ai(z)$ and $\Bi(z)$, with asymptotic forms

\begin{align*}
	\Ai(z) &\sim \frac{1}{2\sqrt{\pi}z^{1/4}}\exp\qty(-\frac{2}{3}z^{3/2}) \\[4pt]
	\Bi(z) &\sim \frac{1}{2\sqrt{\pi}z^{1/4}}\exp\qty(\frac{2}{3}z^{3/2})
\end{align*}

for $z \gg 0$, and

\begin{align*}
\Ai(z) &\sim \frac{1}{2\sqrt{\pi}(-z)^{1/4}}\sin\qty(\frac{2}{3}(-z)^{3/2} + \frac{\pi}{4}) \\[4pt]
\Bi(z) &\sim \frac{1}{2\sqrt{\pi}(-z)^{1/4}}\cos\qty(\frac{2}{3}(-z)^{3/2} + \frac{\pi}{4})
\end{align*}

for $Z \ll 0$.

The patching wavefunction is then

\begin{equation}
	\psi_p = a \Ai(\alpha x) + b \Bi(\alpha x) \,.
\end{equation}

We can match the patching wavefunction to the WKB solutions on both sides of $x=0$.

%% Figure: Patching region

The overlapping regions must be close enough to the turning point at $x=0$ that $V(x)$ can be safely linearized, but far enough that the WKB approximation is reliable. In the overlap region,

\begin{equation}
	p(x) \approx \sqrt{2m\qty(E - E - x\eval{\dv{V}{x}}_{x=0}\,)} = \hbar \alpha^{3/2} \sqrt{-x} \,.
\end{equation}  

For $x > 0$:

\begin{equation}
	\int_{0}^{x} |p(x^\prime)|\dd{x^\prime} = \hbar\alpha^{3/2}\int_{0}^{x}\sqrt{x^\prime}\dd{x^\prime} = \frac{2}{3}\hbar(\alpha x)^{3/2}
\end{equation}

and the WKB wavefunction becomes

\begin{equation}
	\psi(x)\simeq \frac{D}{\sqrt{\hbar}\alpha^{3/4}x^{1/4}}\exp\qty(-\frac{2}{3}(\alpha x)^{3/2}) \,.
\end{equation}

Using the Airy functions (for large $z$) in the positive overlap region, 

\begin{equation}
	\psi_p(x)\simeq \frac{a}{2\sqrt{\pi}(\alpha x)^{1/4}}\exp(-\frac{2}{3}(\alpha x)^{3/2}) + \frac{b}{\sqrt{\pi}(\alpha x)^{1/4}}\exp(\frac{2}{3}(\alpha x)^{3/2}) \,.
\end{equation}

Comparing the positive WKB wavefunction and the patching function, we find that

\begin{equation}
	a = \sqrt{\frac{4\pi}{\alpha\hbar}} D \quad\text{and}\quad b = 0 \,.
\end{equation}

Repeating the same procedure for the negative overlap region,

\begin{equation}
	\int_{x}^{0}p(x^\prime)\dd{x^\prime} = \frac{2}{3}\hbar(-\alpha x)^{3/2} \,,
\end{equation}

and the WKB wavefunction is 

\begin{equation}
	\psi(x)\simeq \frac{1}{\sqrt{\hbar}\alpha^{3/4}(-x)^{1/4}}\qty[B\exp\qty(\frac{2i}{3}(-\alpha x)^{3/2}) + C\exp\qty(-\frac{2i}{3}(-\alpha x)^{3/2})] \,.
\end{equation}

We set consider the large Airy functions ($x<0$) and set $b=0$. This gives

\begin{align}
	\psi_p(x)&\simeq\frac{a}{\sqrt{\pi}(-\alpha x)^{1/4}}\sin\qty(\frac{2}{3}(-\alpha x)^{3/2} + \frac{\pi}{4}) \nonumber \\[4pt]  &= \frac{a}{\sqrt{\pi}(-\alpha x)^{1/4}}\frac{1}{2i}\qty[e^{i\pi/4}\exp\qty(\frac{2i}{3}(-\alpha x)^{3/2}) - e^{-i\pi/4}\exp\qty(-\frac{2i}{3}(-\alpha x)^{3/2})] \,.
\end{align}

Comparing the WKB and patching wavefunctions, we find

\begin{equation}
	\frac{a}{2i\sqrt{\pi}}e^{i\pi/4} = \frac{B}{\sqrt{\hbar\alpha}} \quad\text{and}\quad -\frac{a}{2i\sqrt{\pi}}e^{-i\pi/4} = \frac{C}{\sqrt{\hbar\alpha}} \,.
\end{equation}

Recalling that $a = D\sqrt{4\pi/\alpha\hbar}$, we find

\begin{equation}
	\boxed{ B = -ie^{i\pi/4}D \quad\text{and}\quad C = ie^{-i\pi/4}D }
\end{equation}

These are the \textbf{connection formulae}, joining the WKB solutions at either side of the turning point. The WKB wavefunction for some turning point at $x_0$ is

\begin{equation}
	\boxed{\psi(x)\simeq \begin{cases}
	\frac{2D}{\sqrt{p(x)}}\sin\qty[\frac{1}{\hbar}\int_{x}^{x_0}p(x^\prime) \dd{x^\prime} + \frac{\pi}{4}] & x < x_0 \\
	\frac{D}{\sqrt{|p(x)|}}\exp\qty[-\frac{1}{\hbar}\int_{x_0}^{x}|p(x^\prime)|\dd{x^\prime}] & x>x_0
	\end{cases}}
\end{equation} 

\begin{mdframed}[backgroundcolor=gray!20]
\paragraph*{Example:}
Consider the \textit{half harmonic oscillator}, with potential given by

\begin{equation*}
	V(x) = \begin{cases}
	\frac{1}{2}m\omega^2 x^2 & x>0 \\[4pt]
	\infty &\text{otherwise}
	\end{cases} \,.
\end{equation*}

Use the WKB approximation to determine the energies of each bound state.

\paragraph*{Solution:}
In this case, $\psi(0) = 0$ and therefore,

\begin{equation*}
	\frac{1}{\hbar}\int_{0}^{x_0}p(x)\dd{x} + \frac{\pi}{4} = n\pi \,,
\end{equation*}

or

\begin{equation*}
	\int_{0}^{x_0}p(x)\dd{x} = (n-1/4)\pi\hbar \,.
\end{equation*}

We have

\begin{equation*}
	p(x) = \sqrt{2m\qty(E - \frac{1}{2}m\omega^2 x^2)} = m\omega\sqrt{{x_0}^2 - x^2} \,,
\end{equation*}

where 

\begin{equation*}
	x_0 = \frac{1}{\omega}\sqrt{\frac{2E}{m}}
\end{equation*}

is the turning point. This means that

\begin{equation*}
	\int_{0}^{x_0}p(x)\dd{x} = m\omega\int_{0}^{x_0}\sqrt{{x_0}^2 - x^2}\dd{x} = \frac{\pi}{4}m\omega{x_0}^2 = \frac{\pi E}{2\omega}\,.
\end{equation*}

The quantization condition gives

\begin{equation*}
	E_n = \qty(2n-\frac{1}{2})\hbar\omega \,.
\end{equation*}

These are the odd energies of the full harmonic oscillator, which happen to be the exact energies for this problem.
\end{mdframed}

We determined the connection formulae for a turning point where the potential slopes upward. We now consider a potential well with no vertical walls. In this case, the turning points slope downward and we find that

\begin{equation}
\boxed{\psi(x)\simeq \begin{cases}
	\frac{D^\prime}{\sqrt{|p(x)|}}\exp\qty[-\frac{1}{\hbar}\int_{x_0}^{x}|p(x^\prime)|\dd{x^\prime}] & x<x_0 \\
	\frac{2D^\prime}{\sqrt{p(x)}}\sin\qty[\frac{1}{\hbar}\int_{x}^{x_0}p(x^\prime) \dd{x^\prime} + \frac{\pi}{4}] & x > x_0 
	\end{cases}}
\end{equation} 

If we consider a potential well, the wavefunction in the classical region can be written as 

\begin{equation}
	\psi(x)\approxeq \frac{2D}{\sqrt{p(x)}}\sin\theta_2(x) \,, \quad\text{where}\quad \theta_2 = \frac{1}{\hbar}\int_{x}^{x_0}p(x^\prime)\dd{x^\prime} + \frac{\pi}{4}
\end{equation}

or 

\begin{equation}
	\psi(x)\approxeq -\frac{2D^\prime}{\sqrt{p(x)}}\sin\theta_1(x) \,, \quad\text{where}\quad \theta_1 = -\frac{1}{\hbar}\int_{x}^{x_0^\prime}p(x^\prime)\dd{x^\prime} - \frac{\pi}{4} \,.
\end{equation}

Since both expressions describe the same region, then $\theta_2 = \theta_1 + n\pi$:

\begin{equation*}
	\frac{1}{\hbar}\int_{x}^{x_0}p(x^\prime)\dd{x^\prime} + \frac{\pi}{4} = -\frac{1}{\hbar}\int_{x_0^\prime}^{x}p(x^\prime)\dd{x^\prime} - \frac{\pi}{4} + n\pi \,,
\end{equation*}

which means that

\begin{equation}
	\int_{x_0^\prime}^{x_0}p(x)\dd{x} = \qty(n-\frac{1}{2})\pi\hbar \,.
\end{equation}

This quantization condition gives the allowed energies for a typical potential well with two sloping sides. 

\begin{mdframed}
\paragraph*{Note:}
The result is different for two vertical walls $(n)$, one vertical wall $(n-1/4)$, and no vertical walls $(n-1/2)$. The WKB approximation works best in the semi-classical (large $n$) case, where the differences are minor.
\end{mdframed}

\begin{mdframed}
\paragraph*{Note:}
This approach is powerful because it eliminates the need to solve the Schrödinger equation or deal with the wavefunction. Allowed energies are obtained by evaluating one integral. 
\end{mdframed}

\section{Time dependent perturbation theory}
Here we consider problems where the potential is time varying, $V(\vb{r},t)$. This is interesting since there are important physical processes that are time dependent, (like the absorption of radiation), and allows for transitions between energy levels. In \textbf{quantum dynamics}, there are few problems that can be solved exactly and therefore we develop time dependent perturbation theory which provides approximate solutions when the time dependent part of the Hamiltonian is small compared to the time independent part.

\subsection{Two-level systems}
We begin by considering that there are two unperturbed states $\psi_a$ and $\psi_b$, which are eigenstates of the unperturbed Hamiltonian, $\mathcal{H}^0$:

\begin{equation*}
	\mathcal{H}^0\psi_a = E_a\psi_a \,; \quad \mathcal{H}^0\psi_a = E_a\psi_a \,; \quad \bra{\psi_a}\ket{\psi_b}=\delta_{ab} \implies \psi(0) = c_a\psi_a + c_b\psi_b\,.
\end{equation*}

In the absence of any perturbation, the time dependent wavefunction is

\begin{equation}
	\Psi(t) = C_a\psi_ae^{-i E_a t/\hbar} + C_b\psi_be^{-i E_b t/\hbar} \,,
\end{equation}

and $|C_a|^2$ and $|C_b|^2$ are the probabilities that a measurement of energy would yield the value $E_a$ or $E_b$. Of course, normalization requires $|C_a|^2$ and $|C_b|^2$.

\subsubsection{The perturbed system}
Now suppose we have a time dependent perturbation $\mathcal{H}^\prime(t)$. We wish to determine the time dependent coefficients $C_a$ and $C_b$. If, for example, the particle starts out in the state $\psi_a$ ($C_a(0) = 1$) and at some later time $_0$ the we find that $C_b(t_0) = 1$, the system will have undergone a transition $\psi_a \rightarrow \psi_b$.

We impose that the wavefunction $\Psi(t)$ must satisfy the time dependent Schrödinger equation,

\begin{equation}
	\mathcal{H}\Psi = i\hbar\pdv{\Psi}{t} \,, \quad\text{where}\quad \mathcal{H} = \mathcal{H}^0 + \mathcal{H}^\prime \,.
\end{equation}

Inserting the form of $\Psi$ into the Schrödinger equation gives

\begin{multline*}
	C_a\qty(\mathcal{H}^0\psi_a)e^{-iE_at/\hbar} + C_b\qty(\mathcal{H}^0\psi_b)e^{-iE_bt/\hbar} + C_a\qty(\mathcal{H}^\prime\psi_a)e^{-iE_at/\hbar} + C_b\qty(\mathcal{H}^\prime\psi_b)e^{-iE_bt/\hbar} =\\ i\hbar\qty[\dot{C}_a\psi_ae^{-iE_at/\hbar} + \dot{C}_b \psi_be^{-iE_bt/\hbar} + C_a\psi_a\qty(-\frac{iE_a}{\hbar})e^{-iE_a t/\hbar} + C_b\psi_b\qty(-\frac{iE_b}{\hbar})e^{-iE_b t/\hbar}]
\end{multline*}

for which the first two terms on the left cancel the lat two terms on the right, which gives

\begin{equation}
	C_a\qty(\mathcal{H}^\prime\psi_a)e^{-iE_at/\hbar} + C_b\qty(\mathcal{H}^\prime\psi_b)e^{-iE_bt/\hbar} = i\hbar\dot{C}_a\psi_ae^{-iE_at/\hbar} + i\hbar\dot{C}_b \psi_be^{-iE_bt/\hbar} \,.
\end{equation}

We isolate $\dot{C}_a$ and $\dot{C}_a$ by taking the inner product with $\psi_a$ and $\psi_b$:

\begin{subequations}
\begin{align}
	C_a\bra{\psi_a}\mathcal{H}^\prime\ket{\psi_a}e^{-iE_at/\hbar} + C_b\bra{\psi_a}\mathcal{H}^\prime\ket{\psi_b}e^{-iE_bt/\hbar} &= i\hbar\dot{C}_ae^{-E_at/\hbar} \\[4pt]
	C_a\bra{\psi_b}\mathcal{H}^\prime\ket{\psi_a}e^{-iE_at/\hbar} + C_b\bra{\psi_b}\mathcal{H}^\prime\ket{\psi_b}e^{-iE_bt/\hbar} &= i\hbar\dot{C}_be^{-E_bt/\hbar}
\end{align}
\end{subequations}

and defining the matrix elements $\mathcal{H}_{ij}^\prime = \bra{\psi_i}\mathcal{H}^\prime\ket{\psi_j}$, we have the effective Schrödinger equation:

\begin{subequations}
\begin{align}
	\dot{C}_a &= -\frac{i}{\hbar}\qty[C_a\mathcal{H}^\prime_{aa} + C_b\mathcal{H}^\prime_{ab}e^{-i(E_b-E_a)t/\hbar}] \,; \\[4pt]
	\dot{C}_b &= -\frac{i}{\hbar}\qty[C_b\mathcal{H}^\prime_{bb} + C_a\mathcal{H}^\prime_{ba}e^{-i(E_b-E_a)t/\hbar}] \,.
\end{align}
\end{subequations}

It is common that the diagonal matrix elements vanish, which gives

\begin{align*}
	\dot{C}_a &= -\frac{i}{\hbar} C_b\mathcal{H}^\prime_{ab}e^{-i\omega_0 t} \,; \\[4pt]
	\dot{C}_b &= -\frac{i}{\hbar}C_a\mathcal{H}^\prime_{ba}e^{-i\omega_0 t} \,,
\end{align*}

where $\omega = \Delta E/\hbar$. 

\subsubsection{Time dependent perturbation}
As we did with time independent perturbation theory, we expand the Hamiltonian in powers of the parameter $\lambda$, which we then set to unity to introduce the perturbation. We have

\begin{equation}
	\mathcal{H}^\prime(t) = \mathcal{H}^0(t) + \lambda\mathcal{H}^\prime(t)\,, \qquad C_n(t) = C_n^{(0)}(t) + \lambda C_n^{(1)}(t) + \lambda^2C_n^{(2)}(t) + \ldots
\end{equation}

which gives to zeroth order:

\begin{subequations}
\begin{align}
	\dot{C}_a^{(0)}(t) = 0 \,; \\[4pt]
	\dot{C}_a^{(0)}(t) = 0 \,,
\end{align}
\end{subequations}

to first order:

\begin{subequations}
\begin{align}
	\dot{C}_a^{(1)}(t) &= -\frac{i}{\hbar}\qty(C_a^{(0)}\mathcal{H}^\prime_{aa} + C_b^{(0)}\mathcal{H}^\prime_{ab}e^{-i\omega_0 t}) \,;\\[4pt]
	\dot{C}_b^{(1)}(t) &= -\frac{i}{\hbar}\qty(C_b^{(0)}\mathcal{H}^\prime_{bb} + C_a^{(0)}\mathcal{H}^\prime_{ba}e^{-i\omega_0 t})\,,
\end{align}
\end{subequations}

and to higher order:

\begin{subequations}
\begin{align}
\dot{C}_a^{(r)}(t) &= -\frac{i}{\hbar}\qty(C_{a}^{(r-1)}\mathcal{H}^\prime_{aa} + C_{b}^{(r-1)}\mathcal{H}^\prime_{ab}e^{i\omega_0 t})\,; \\[4pt]
\dot{C}_b^{(r)}(t) &= -\frac{i}{\hbar}\qty(C_{b}^{(r-1)}\mathcal{H}^\prime_{bb} + C_{a}^{(r-1)}\mathcal{H}^\prime_{ba}e^{i\omega_0 t})\,.
\end{align}
\end{subequations}

We have a recursive formula, where the next higher order energy is obtained from the current order. Let us determine the solutions where the particle is initially in state $\psi_a$, or $C_a(0) = 1$ and $C_b(0)=0$. To zeroth order:

\begin{align}
	&\dot{C}_a^{(0)} = 0 &&\dot{C}_b^{(0)} = 0 \,; \nonumber \\[4pt]
    &C_{a}^{(0)} = 1 &&C_{b}^{(0)} = 0 \,.
\end{align}

To first order:

\begin{align}
	&\dot{C}_a^{(1)} = -\frac{i}{\hbar}\mathcal{H}^\prime_{aa} &&\dot{C}_a^{(1)} = -\frac{i}{\hbar}\mathcal{H}^\prime_{aa}\,; \nonumber \\[4pt]
	&C_{a}^{(1)} = \frac{1}{i\hbar}\int_{0}^{t} \mathcal{H}^\prime_{aa}(\tau)\dd{\tau}  &&C_{b}^{(1)} = \frac{1}{i\hbar}\int_{0}^{t} \mathcal{H}^\prime_{ba}(\tau)e^{i\omega_0 \tau}\dd{\tau} 
\end{align}

With these results we can write the time dependent expansion coefficients to first order and determine an approximation of $\Psi(t)$.

From the coefficients $C_n(t)$, we can calculate the transition probability from an initial state $i$ to final state $f$. The definition is

\begin{equation}
	\mathscr{P}_{if} = |\bra{\psi_f}\ket{\Psi(t)}|^2 \,,
\end{equation}

where $\ket{\Psi(0)} = \ket{\psi_i}$. This gives

\begin{equation}
	\mathscr{P}_{if}(t) = \qty|C_f(t)|^2 \,,
\end{equation}

where $C_f(t) = C_{f}^{(0)}(t) + C_f^{(1)}(t) + \ldots$

We are interested when $\mathcal{H}^\prime(t)$ introduces transitions $i \rightarrow f$ (for $i \neq f$). We have $C_f^{0}(t) = 0$ and therefore to first order:

\begin{equation}
	\mathcal{P}_{if} = |C_f^{(1)}(t)|^2\,.
\end{equation}

Using the approximation we obtained,

\begin{equation}
	\mathscr{P}_{if}(t) = \frac{1}{\hbar^2}\qty|\int_{0}^{t}\mathcal{H}^\prime_{fi}(\tau)e^{i\omega_0 \tau}\dd{\tau}|^2 \,.
\end{equation}

This assumes that $\mathcal{H}^\prime(t<0) = 0$. The transition probability is proportional to the matrix element $\bra{\psi_f}\mathcal{H}^\prime\ket{\psi_i}$.

\subsubsection{Sinusoidal perturbations}
We assume that we have a perturbation with sinusoidal time dependence,

\begin{equation}
	\mathcal{H}^\prime(\vb{r}, t) = V(\vb{r})\cos\omega t \,quad (t>0)\
\end{equation}

and we wish to extract the transmission probability of going from state $a$ to $b$. We have

\begin{equation}
	\mathcal{H}^\prime_{ab} = V_{ab}\cos\omega t \,,
\end{equation}

where 

\begin{equation}
	V_{ab} = \bra{\psi_a}V\ket{\psi_b} \,.
\end{equation}

We have

\begin{align}
	C_{b}(t) &\simeq \frac{1}{i\hbar}\int_{0}^{t} \cos\omega\tau e^{i\omega_0\tau}\dd{\tau} \nonumber\\[4pt]
	&= \frac{V_{ba}}{2i\hbar}\frac{0}{t}\qty[e^{i(\omega_0 + \omega)\tau} + e^{i(\omega_0-\omega)\tau}]\dd{\tau} \nonumber \\[4pt]
	&= \frac{V_ba}{2\hbar}\qty[\frac{e^{i(\omega_0+\omega)t}-1}{\omega_0+\omega} + \frac{e^{i(\omega_0-\omega)t-1}}{\omega_0-\omega}] \,.
\end{align}

We focus on driving frequencies near the transition frequency $\omega_0$:

\begin{equation}
	\omega_0 + \omega \gg |\omega_0 - \omega| \,.
\end{equation}

This leaves us with

\begin{align}
	C_b(t) &\approx \frac{V_{ba}}{2\hbar}\frac{e^{i(\omega_0-\omega)t/2}}{\omega_0 - \omega}\Bigl[e^{i(\omega_0-\omega)t/2} - e^{-i(\omega_0-\omega)t/2}\Bigl] \nonumber \\[4pt]
	&= \frac{V_{ba}}{i\hbar}\frac{\sin\bigl((\omega_0 - \omega)t/2\bigr)}{\omega_0 - \omega}e^{i(\omega_0 - \omega)t/2} \,.
\end{align}

The transition probability $\psi_a \rightarrow \psi_b$ at time $t$ is

\begin{equation}
	\mathscr{P}_{ab}(t) = |C_b(t)|^2 \approx \frac{|V_{ab}|^2}{\hbar^2}\frac{\sin^2\bigl((\omega_0 - \omega)t/2\bigr)}{(\omega_0 - \omega)} \,.
\end{equation}

%% Figure: Transition probability function of t and omega

The transition probability oscillates sinusoidally. The probability of measuring a particle in a state depends on when it is observed. This is in agreement with the exact solution.

\subsubsection{Generalization to N-level system}
For an eigenspace with$N$ dimensions, any state may be written

\begin{equation}
	\Psi(t) = \sum_{n=1}^{N}C_n(t)\psi_ne^{-iE_n t/\hbar} ]\,,
\end{equation}

where the Fourier coefficients are governed by

\begin{equation}
	i\hbar \dot{C}_n(t) = \sum_{k=1}^{N}\mathcal{H}_{nk}^\prime e^{i\omega_{nk}t}C_{k}(t) \,,
\end{equation}

where $\omega_{nk} = (E_n - E_k)\hbar$. The corrections are:

\begin{align}
	& i\hbar\dot{C}_n^{(0)}(t) = 0 \\[4pt]
	& i\hbar\dot{C}_n^{(r)}(t) = \sum_{k=1}^{N}\mathcal{H}_{nk}(t)e^{i\omega_{nk}t}C_{k}^{(r-1)}(t) \,.
\end{align}

Assuming that the particle is in one unperturbed state $\psi_i$,

\begin{align}
	C_n^{(0)}(t) &= \delta_{ni} \\[4pt]
	C_n^{(1)}(t) &= \int_{0}^{t}\mathcal{H}^\prime_{ni}(\tau)e^{i\omega_{ni}\tau}\dd{\tau} \,.
\end{align}

The transmission probability is the same as before.

\subsection{Emission and Absorption of Radiation}
\subsubsection{Electromagnetic waves}
Electromagnetic waves have oscillating electric and magnetic fields.

%% Figure: EM radiation

If the wavelength of the light is long compared to the size of an atom, we can ignore the spatial variation of electric field, which we can define as:

\begin{equation}
	\vb{E} = E_0\cos\omega t \vu{z} \,.
\end{equation}

An atom is then perturbed by this electric field:

\begin{align}
	\mathcal{H}^\prime = -q E_0 z \cos\omega t \,;
	\mathcal{H}^\prime_{ba} = i pE_0 \cos\omega t\,,
\end{align}

where $p \equiv q\bra{\psi_b}z\ket{\psi_a}$ is an odd diagonal matrix element of the $z$-component of the electric dipole moment operator, $q\vb{r}$. This interaction is the same as the oscillatory perturbation we studied before,

\begin{equation}
	V_{ba} = -pE_0 \,.
\end{equation}

\subsubsection{Absorption, stimulated emission and spontaneous emission}
If an atom is in the \textit{lower} state $\psi_a$ and monochromatic polarized light is incident, the probability of transition to the \textit{upper} state $\psi_b$ is

\begin{equation}
	\mathscr{P}_{ab}(t) = \qty(\frac{|p|E_0}{\hbar})^2\frac{\sin^2\qty(\frac{1}{2}(\omega_0 - \omega)t)}{(\omega_0 - \omega)^2} \,.
\end{equation}

In this process, the electron absorbs the energy $E_b - E_a = \hbar\omega_0$ from the electric field; it absorbs a photon.

Doing the same calculation assuming the electron the electron starts in the upper state $\psi_b$, we find

\begin{equation}
	\mathscr{P}_{ba}(t) = \qty(\frac{|p|E_0}{\hbar})^2\frac{\sin^2\qty(\frac{1}{2}(\omega_0 - \omega)t)}{(\omega_0 - \omega)^2} \,.
\end{equation}

If the electron is in the upper state and light is shined on it, it can make a transition to the lower state and emit a photon of energy $\hbar\omega_0$. Interestingly, the probability is the same as absorbing a photon. This process was first predicted by Einstein and is called stimulated emission.

%% Figure: Stimulated emission

Stimulated emission gives rise to amplification. If there are many atoms with electrons in the upper state, a single incident photon can produce a second, and this process can continue. This is the principle behind lasers (\textit{light amplification by stimulated emission of radiation}).

Spontaneous emission is the down transition without an applied field. This is identical to stimulated emissionm since quantum mechanics states that there's always a non-zero field known as \textit{zero point} radiation.

\subsubsection{Incoherent perturbations}
The energy density of electromagnetic waves is given by:

\begin{equation}
	u = \frac{\epsilon_0}{2}\,{E_0}^2 \,,
\end{equation}

where $E_0$ is the amplitude of the electric field. Then the expression for the transition probabiluty becomes:

\begin{equation}
	\mathscr{P}_{ba}(t) = \frac{2u}{\epsilon_0\hbar^2}|p^2|\frac{\sin^2\qty[(\omega_0-\omega)t/2]}{(\omega_0 - \omega)^2} \,.
\end{equation}

This is for monochromatic light; a single $\omega$. In many cases, the system would be exposed to a range of frequencies. In this case, we replace $u \rightarrow \rho(\omega)\dd{\omega}$, which is the energy density in the frequency range $\dd{\omega}$. We then integrate over all frequencies to obtain:

\begin{equation*}
	\mathscr{P}_{ba}(t) = \frac{2}{\epsilon_0\hbar^2}|p^2|\int_{0}^{\infty}\rho(\omega)\,\frac{\sin^2\qty[(\omega_0-\omega)t/2]}{(\omega_0 - \omega)^2} \dd{\omega}\,,
\end{equation*}

which is sharply peaked about $\omega = \omega_0$. We can then approximate $\rho(\omega) \approx \rho(\omega_0)$ and obtain

\begin{align}
	\mathscr{P}_{ba}(t) &\cong \frac{2|p|^2\rho(\omega_0)}{\epsilon_0\hbar^2}\int_{0}^{\infty}\frac{\sin^2\qty[(\omega_0-\omega)t/2]}{(\omega_0 - \omega)^2} \dd{\omega}\,, \nonumber\\[4pt]
	&= \frac{\pi|p|^2}{\epsilon_0\hbar^2}\rho(\omega_0) t \,.
\end{align}

\begin{mdframed}
\paragraph*{Note:} In the above relation we are required to compute the integral

\begin{equation*}
	I = \int_{0}^{\infty}\frac{\sin^2\qty[(\omega_0-\omega)t/2]}{(\omega_0 - \omega)^2} \dd{\omega}\,.
\end{equation*}

We make the substitution $x = (\omega_0 - \omega)t/2$ such that $\dd{x} = -t\dd{\omega}/2$. We may extend the bounds to $x = \pm\infty$ since the integrand decays rapidly. This gives us the approximate value

\begin{equation*}
	I \cong \frac{t}{2}\int_{-\infty}^{\infty}\frac{\sin^2 x}{x^2} = \frac{t\pi}{2}\,.
\end{equation*}

This remaining integral is equal to $\pi$ and can be solved by complex analysis, Feynman's trick, or Fourier series via Parseval's theorem.
\end{mdframed}

Now the transition probability is proportional to $t$. The oscillating behaviour seen with monochromatic light gets washed out when incoherent light of many frequencies is used. Here the transition rate is constant:

\begin{equation*}
	R_{ba} = \dv{\mathscr{P}}{t} = \frac{\pi}{\epsilon_0\hbar^2}|p|^2 \rho(\omega_0) \,.
\end{equation*}

There is another correction we need to consider for incoherent perturbations. Before we assumed that the incident light is traveling along $y$with polarization along $z$. We are interested in the case when light is incident from all directions with all polarizations. For this we need to replace $|p|^2$ with the average of $|\vb{p}\vdot\vu{n}|^2$\, where $\vu{n}$ is the polarization direction, $\vb{p} = \bra{\psi_b}\vb{r}\ket{\psi_a}$ is the polarization matrix element, and the average is taken over all space.

To perform the angular average we select spherical coordinates and choose the direction of propagation to be along $x$ and $\vu{n}$ along $z$. This way, $\vb{p}$ defines the spherical angles $\theta$ and $\phi$.

%% Figure: Geometry of orientations

We have

\begin{equation}
	\vb{p}\vdot\vu{n} = p\cos\theta \,,
\end{equation}

and the average over all angles is

\begin{equation}
	\exv{|\vb{p}\vdot\vu{n}|^2} = \frac{1}{4\pi}\int_{0}^{2\pi}\int_{0}^{\pi}|p|^2\cos\theta\sin\theta\dd{\theta}\dd{\phi} = \frac{1}{3}|p|^2 \,.
\end{equation}

The transition rate for stimulated emission from state $b$ to state $a$, under the influence of incoherent, unpolarized light incident from all directions is therefore

\begin{equation}
	R_{ba} = \frac{\pi}{3\epsilon_0\hbar^2}|p|^2\rho(\omega_0) \,.
\end{equation}


\end{document}